\batchmode% ===> this file was generated automatically by noweave --- better not edit it
\makeatletter
\def\input@path{{/home/druffer/Dropbox/ColorForth/File/Docs//}}
\makeatother
\documentclass[10pt,english]{article}
\usepackage[T1]{fontenc}
\usepackage[latin1]{inputenc}
\usepackage{listings}
\usepackage[dvips]{geometry}
\geometry{verbose,lmargin=2cm,rmargin=2cm}
\usepackage{fancyhdr}
\pagestyle{fancy}
\setcounter{secnumdepth}{5}
\setcounter{tocdepth}{5}
\setlength{\parskip}{\smallskipamount}
\setlength{\parindent}{0pt}
\usepackage{babel}
\usepackage{prettyref}
\usepackage[dvips]{graphicx}
\usepackage[unicode=true,pdfusetitle,
 bookmarks=true,bookmarksnumbered=false,bookmarksopen=false,
 breaklinks=false,pdfborder={0 0 1},backref=false,colorlinks=false]
 {hyperref}
\usepackage{breakurl}

\makeatletter

%%%%%%%%%%%%%%%%%%%%%%%%%%%%%% LyX specific LaTeX commands.
\providecommand{\LyX}{\texorpdfstring%
  {L\kern-.1667em\lower.25em\hbox{Y}\kern-.125emX\@}
  {LyX}}
%% Because html converters don't know tabularnewline
\providecommand{\tabularnewline}{\\}

%%%%%%%%%%%%%%%%%%%%%%%%%%%%%% Textclass specific LaTeX commands.
\usepackage{noweb}

%%%%%%%%%%%%%%%%%%%%%%%%%%%%%% User specified LaTeX commands.
%Document generated by wvWare/wvWare version 1.2.9
%wvWare written by Caolan.McNamara@ul.ie
%http://wvware.sourceforge.net
%

\@ifundefined{definecolor}
 {\usepackage{color}}{}
\usepackage[dvips]{graphics}\usepackage{longtable}\usepackage{times}
\usepackage{comment}\usepackage{lastpage}
\usepackage{microtype}\DisableLigatures[-]{}

\usepackage[normalem]{ulem}
\newcommand{\suppress}[1]{}
\newcommand{\deleted}[1]{\xout{#1}}
\newcommand{\revised}[1]{\uline{#1}}
\newlength\wvtextpercent
\setlength{\wvtextpercent}{0.009\textwidth}

\newbox\strikebox
\def\strike#1{\setbox\strikebox \hbox{<#1>}\hbox{\raise0.5ex\hbox to 0pt{\vrule height 0.4pt width \wd\strikebox\hss}\copy\strikebox}}

\makeatother

\begin{document}

\title{DataBase Support System}


\author{Elizabeth Rather and Dennis Ruffer}


\date{08 July 2012}
\maketitle
\begin{abstract}
To describe a simple database support system. This system originated
as a loadable module in polyFORTH. It was converted to be compatible
with the ANS Forth standard with a goal to become an IEEE 1275 binding
for a database package. Later, it was put into the VentureForth development
system for various purposes, including the Timing Diagrams. Now, it
is being used again in the colorForth assembler/disassembler and converted
to be compatible with 64-bit gforth, using Literate Programing in
\LyX{}

\tableofcontents{}
\end{abstract}

\section{Overview \& License: }

This document specifies a proposal for a database package. 


\subsection{Revision History}

\begin{tabular}{|c|c|c|}
\hline 
Rev. &
Date &
Description\tabularnewline
\hline 
\hline 
0.1 &
4/26/05 &
First draft\tabularnewline
\hline 
0.2 &
5/5/05 &
First formatting pass, with {[}DaR{]} markers for where more work
is required\tabularnewline
\hline 
0.3 &
5/18/05 &
Test all examples and fix {[}DaR{]} markers\tabularnewline
\hline 
0.4 &
1/15/09 &
Clean up for release as a VentureForth plug in\tabularnewline
\hline 
0.5 &
8/5/12 &
Convert to Literate Programming style\tabularnewline
\hline 
\end{tabular}


\subsection{License }

Copyright (c) 2009 FORTH, Inc. Portions contributed by Dennis Ruffer.

Permission is hereby granted, free of charge, to any person obtaining
a copy of this software and associated documentation files (the \textquotedbl{}Software\textquotedbl{}),
to deal in the Software without restriction, including without limitation
the rights to use, copy, modify, merge, publish, distribute, sublicense,
and/or sell copies of the Software, and to permit persons to whom
the Software is furnished to do so, subject to the following conditions:

The above copyright notice and this permission notice shall be included
in all copies or substantial portions of the Software.

THE SOFTWARE IS PROVIDED \textquotedbl{}AS IS\textquotedbl{}, WITHOUT
WARRANTY OF ANY KIND, EXPRESS OR IMPLIED, INCLUDING BUT NOT LIMITED
TO THE WARRANTIES OF MERCHANTABILITY, FITNESS FOR A PARTICULAR PURPOSE
AND NONINFRINGEMENT. IN NO EVENT SHALL THE AUTHORS OR COPYRIGHT HOLDERS
BE LIABLE FOR ANY CLAIM, DAMAGES OR OTHER LIABILITY, WHETHER IN AN
ACTION OF CONTRACT, TORT OR OTHERWISE, ARISING FROM, OUT OF OR IN
CONNECTION WITH THE SOFTWARE OR THE USE OR OTHER DEALINGS IN THE SOFTWARE.


\section{DATA BASE SUPPORT }

The polyFORTH Data Base Support package is a set of tools with which
you can design efficient data base applications or components for
general applications.

The Data Base Support package includes:
\begin{itemize}
\item A simple memory-oriented file manager.
\item Commands for defining records within files and fields within records.
\item Tools for generating columnar reports.
\item Utilities for producing totals and subtotals.
\item Techniques for linking sub-files to main files and for chaining records
within files, and;
\item A set of words for creating ordered indexes (for keeping sorted lists). 
\end{itemize}
\nwfilename{pFDatabase.nw}\nwbegincode{1}\sublabel{NWtG4Lr-4ffLeu-1}\nwmargintag{{\nwtagstyle{}\subpageref{NWtG4Lr-4ffLeu-1}}}\moddef{support~{\nwtagstyle{}\subpageref{NWtG4Lr-4ffLeu-1}}}\endmoddef\nwstartdeflinemarkup\nwusesondefline{\\{NWtG4Lr-2V5MiK-1}}\nwenddeflinemarkup
\LA{}ordered~{\nwtagstyle{}\subpageref{NWtG4Lr-4Y6sNo-1}}\RA{}
\LA{}bind-fields~{\nwtagstyle{}\subpageref{NWtG4Lr-2eFuO6-1}}\RA{}
\LA{}user~{\nwtagstyle{}\subpageref{NWtG4Lr-1alVLr-1}}\RA{}
\LA{}working~{\nwtagstyle{}\subpageref{NWtG4Lr-nM8ar-1}}\RA{}
\LA{}save~{\nwtagstyle{}\subpageref{NWtG4Lr-4HTvUe-1}}\RA{}
\LA{}file~{\nwtagstyle{}\subpageref{NWtG4Lr-4EB9lH-1}}\RA{}
\LA{}block~{\nwtagstyle{}\subpageref{NWtG4Lr-4IiShX-1}}\RA{}
\LA{}read~{\nwtagstyle{}\subpageref{NWtG4Lr-WHmjl-1}}\RA{}
\LA{}address~{\nwtagstyle{}\subpageref{NWtG4Lr-1pHJON-1}}\RA{}
\LA{}indexed~{\nwtagstyle{}\subpageref{NWtG4Lr-36msie-1}}\RA{}
\LA{}origin~{\nwtagstyle{}\subpageref{NWtG4Lr-2aZT6J-1}}\RA{}
\LA{}initialize~{\nwtagstyle{}\subpageref{NWtG4Lr-p3Fax-1}}\RA{}
\LA{}available~{\nwtagstyle{}\subpageref{NWtG4Lr-2zefAp-1}}\RA{}
\LA{}slot~{\nwtagstyle{}\subpageref{NWtG4Lr-M1coC-1}}\RA{}
\LA{}records~{\nwtagstyle{}\subpageref{NWtG4Lr-1FNAaT-1}}\RA{}
\LA{}strings~{\nwtagstyle{}\subpageref{NWtG4Lr-jUK2q-1}}\RA{}
\LA{}numeric~{\nwtagstyle{}\subpageref{NWtG4Lr-44RunE-1}}\RA{}
\LA{}long~{\nwtagstyle{}\subpageref{NWtG4Lr-2qounD-1}}\RA{}
\LA{}float~{\nwtagstyle{}\subpageref{NWtG4Lr-2mGd9n-1}}\RA{}
\LA{}double~{\nwtagstyle{}\subpageref{NWtG4Lr-4Sesx5-1}}\RA{}
\LA{}bytes~{\nwtagstyle{}\subpageref{NWtG4Lr-4T6feO-1}}\RA{}
\LA{}byte~{\nwtagstyle{}\subpageref{NWtG4Lr-4CCogU-1}}\RA{}
\LA{}put~{\nwtagstyle{}\subpageref{NWtG4Lr-105aWa-1}}\RA{}
\LA{}ask~{\nwtagstyle{}\subpageref{NWtG4Lr-2k8QJQ-1}}\RA{}
\LA{}filler~{\nwtagstyle{}\subpageref{NWtG4Lr-4UfOWH-1}}\RA{}
\LA{}fields~{\nwtagstyle{}\subpageref{NWtG4Lr-7mIQU-1}}\RA{}
\LA{}copies~{\nwtagstyle{}\subpageref{NWtG4Lr-3NRM5h-1}}\RA{}
\LA{}entire~{\nwtagstyle{}\subpageref{NWtG4Lr-uRyxG-1}}\RA{}
\LA{}link~{\nwtagstyle{}\subpageref{NWtG4Lr-3NK3Qp-1}}\RA{}
\LA{}access~{\nwtagstyle{}\subpageref{NWtG4Lr-17LFpz-1}}\RA{}
\nwused{\\{NWtG4Lr-2V5MiK-1}}\nwendcode{}\nwbegindocs{2}\nwdocspar


\subsection{Overview \label{sub:Overview}}

polyFORTH presents its Data Base Support package in the form of a
``kit,'' leaving complete flexibility for you, the developer, to
create a data base design that reflects the natural organization of
the data itself.

Before you begin constructing a data base application, you must understand
a few simple premises that underlie the design of the Data Base Support
package. First, let's review common data base terminology.

A \textit{data base} is the complete set of organized data that is
available to the computer. A data base is divided into related groups
of data called \textit{files}. For example, a file might contain the
names, addresses, and phone numbers of all your clients.

A file, in turn, is divided into \textit{records}. A record might
contain the name, address and phone number for a single client. For
every client, there would be one record in the file.

A record is itself divided into a collection of \textit{fields}. For
instance, one field might be called ``STREET.''

In a data acquisition environment, a file might contain a set of readings
taken during one experiment. Each record could contain the set of
measure\-ments taken at a single point in time during the experiment;
each field could contain the reading of a different measurement. In
this case, you might have numerous files, each containing the data
obtained during one run of the experiment; however, the records in
each file would be laid out identically.

Many applications require multiple types of files that relate one
to another. Suppose you want to record all your invoices, using an
``Accounts Receivable'' program. In the course of your business,
you bill several invoices to the same client. Rather than duplicate
the name and address of the clients every time you bill them, it makes
sense to have one file for the client data, and another file for invoices.
Each invoice record can point to one of the clients in the client
file. In this way, one file can ``use'' another file. 


\subsubsection{Contiguous Files and Performance \label{sub:Contiguous-Files-and-Performance}}

A premise of the Data Base Support package is that you are a knowledgeable
programmer concerned about performance. Its approach allows you to
design the data base for optimum efficiency.

In contrast, typical data base packages are intended to simplify data
base construction for non-programmers. These packages do not require
that you think about how your data is organized. On the down side,
you lose the ability to structure your data base in the optimal way.
The price for greater generality is impaired performance and increased
size of compiled code.

In a disk-oriented data base application, the key determinants of
performance are:
\begin{enumerate}
\item How many physical disk accesses are required to access a logical data
item?
\item How much head motion occurs during normal operation?
\end{enumerate}
If you can minimize the number of physical accesses required and the
disk head motion, you can maximize performance.

\textbf{}
\begin{figure}
\textbf{\caption{\textbf{\label{fig:1}\protect\includegraphics{0_home_druffer_Dropbox_ColorForth_File_Docs_Disk_Memory.eps}}}
}
\end{figure}


In polyFORTH, a file occupies a contiguous range of blocks on disk.
A file may be any size (using whole blocks) up to the capacity of
the disk. There is no need for a file allocation table.

The polyFORTH Data Base Support package addresses both issues simultaneously
by imposing a single restriction: a file is a contiguous region of
data. This means that the data blocks are physically contiguous on
a disk supported by a native polyFORTH. Versions of polyFORTH that
are co-resident with another OS use its files to contain blocks, and
thus rely on the host OS to manage disk. The system does not automatically
``manage'' files by interweaving them on disk as they expand and
contract. Files are not fragmented across the disk, and there is no
need for a file allocation table to point to the fragments.

Instead, you specify the maximum size of each file when you create
the file, and assign it a contiguous range of blocks on the disk.
Although this requires some thought, there are several advantages:
\begin{enumerate}
\item Since files, and therefore records, are contiguous, the exact location
of any data element can be calculated. Thus any data element can be
accessed in a single physical disk access. In traditional operating
systems and file managers, an application cannot know from a record
number where that record lies physically. The location must either
be looked up in a directory, or found by following a set of chains.
Reportedly, one popular operating system requires up to six disk reads
to access a single logical record.
\item While accessing various elements within a single file, the disk's
magnetic head need only travel within the distance occupied by the
file. Head motion is minimized.
\item You have control over the arrangement of multiple files in relation
to one another. For instance, if one file serves as an index into
another file, you can place these two files adjacent to each other
on the disk. Again, head motion is minimized.
\end{enumerate}
These benefits assume that you are running on a native polyFORTH system.
If your polyFORTH runs under another operating system, performance
depends on the way that O/S treats the disk.

Just as files are contiguous and of fixed length, so too are records
and fields. Again, although variable-length fields require less thinking
on the part of the user, they necessarily degrade performance. Fixed-length
fields do not necessarily imply fixed-length amounts of data, because
a variable number of subordinate records may be chained together as
necessary. (We'll explore this technique further in \prettyref{sub:Chaining}).

Since the primary bottleneck in disk-based file systems is disk-access
time, minimizing this bottleneck achieves surprising efficiency. For
example, one company sells a data base system which uses the polyFORTH
Data Base Support package to handle 300 Mbytes of data and support
64 simultaneous users with under one-second response time even at
peak load, on a single 68000 microprocessor.


\subsubsection{Current Files and Records \label{sub:Current-Files-and-Records}}

Another concept that is fundamental to the Data Base Support package
is the notion that at any given moment, exactly one file is current
and one record is current. Let's first describe what is meant by a
file being current.

We mentioned that a file is simply a contiguous, fixed-length range
of blocks on the disk. There is no file allocation table on the disk,
nor is there any other indicator of which blocks constitute which
file. The knowledge of where each file begins and ends resides within
the application code, specifically in a small table that you define
for each file (using the defining word \texttt{\textbf{FILE}}, \prettyref{sub:File-Definition-and-Access}).
This table is called a \textit{File Definition Area} (FDA).

The name you give this table is the name of the file itself. The table
contains the starting block number, along with sufficient information
about the number and size of records for the Data Base Support package
to be able to calculate the absolute location of any record in that
file.

When you invoke the name of a file, the file definition places the
address of its parameter field in a user variable called \texttt{\textbf{F\#}}.
All record-accessing operations in the Data Base Support package use
this pointer to indicate the current FDA, which in turn points to
the blocks where the desired record resides.

Thus, at any given moment, one and only one file is current. Changing
files is a simple matter of invoking the file name, which places a
new address in \texttt{\textbf{F\#}}, taking only microseconds.

Contrast this with the process of ``opening'' and ``closing''
files in tradition\-al operating systems. In these systems, each
open and close operation requires noticeable disk activity to read
in the file directory and write it out again. For this reason, the
question of how many files can be open simultaneously is a concern
in such systems. This concern disappears with polyFORTH's Data Base
Support package.

\textbf{}
\begin{figure}
\textbf{\caption{\textbf{\label{fig:2}\protect\includegraphics{1_home_druffer_Dropbox_ColorForth_File_Docs_Program_Memory.eps}}}
}
\end{figure}


For each file on disk there is a file definition in the dictionary.
This definition contains four parameters describing the location and
dimensions of the corresponding file. In this figure, the second file
is ``current.''

We can summarize the above discussion by saying, ``Files are pointed
to, not opened.'' Analogously, we can say that ``Records are pointed
to, not read.''

Just as there is always a current file, so there is also a current
record, the number of the current record found in the user variable
\texttt{\textbf{R\#}}. All the data-access operators refer to specified
fields within the \textit{current record} in the \textit{current file}.

The polyFORTH Data Base Support package is once again unique in this
concept. Many data base packages actually read in an entire record
from the disk, then allow access to the fields within it. polyFORTH
merely makes a record current; disk access only occurs when a field
name is invoked in combination with a field access operator.

This design takes advantage of the behavior of \texttt{\textbf{BLOCK}}.
Whenever a single field is accessed, \texttt{\textbf{BLOCK}} reads
the entire block in which that field resides. If multiple fields in
the record are required in the same operation (such as displaying
all fields in the record for a report), it is unlikely that the block
buffer will be reused before all the fields can be displayed. (Should
this happen, \texttt{\textbf{BLOCK}} will automatically read the block
again.) Moreover, it is even likely that references to neighboring
records in the same block will also not require physical disk accesses.

An important advantage to not reading the record physically is the
certainty that at any given moment only one copy of each record exists.
Systems that read a record into memory face the problem of two users
accessing the same record, and having different copies of it. Solving
this conflict entails various ``lockout'' schemes, all of which
complicate the system and reduce performance.

The file and record pointers are entirely independent of each other.
Not only can you select records without re-selecting the file, you
can also change files without affecting \texttt{\textbf{R\#}}.

\nwenddocs{}\nwbegincode{3}\sublabel{NWtG4Lr-1alVLr-1}\nwmargintag{{\nwtagstyle{}\subpageref{NWtG4Lr-1alVLr-1}}}\moddef{user~{\nwtagstyle{}\subpageref{NWtG4Lr-1alVLr-1}}}\endmoddef\nwstartdeflinemarkup\nwusesondefline{\\{NWtG4Lr-4ffLeu-1}}\nwenddeflinemarkup
#USER CELL +USER R#         \\ holds the current record number.
      CELL +USER F#         \\ holds the current file pointer.
      CELL +USER DB#        \\ holds the current database pointer.
      CELL +USER FLD#       \\ holds the current field pointer.
TO #USER 
\nwused{\\{NWtG4Lr-4ffLeu-1}}\nwendcode{}\nwbegindocs{4}\nwdocspar

From time to time in your application you may want to leave your current
file and record temporarily (perhaps to examine or display a field
from a related file) and return. The following words enable you to
``remember'' \texttt{\textbf{F\#}} and \texttt{\textbf{R\#}} temporarily:
\begin{description}
\item [{Word}] \textbf{Action}
\item [{\texttt{SAVE}}] pushes \texttt{\textbf{R\#}} and \texttt{\textbf{F\#}}
onto the return stack.
\item [{\texttt{RESTORE}}] pops those items off the return stack and places
them in \texttt{\textbf{R\#}} and \texttt{\textbf{F\#}}.
\end{description}
Naturally, you must use \texttt{\textbf{SAVE}} and \texttt{\textbf{RESTORE}}
as a paired set within the same definition, just as you would use
\texttt{\textbf{>R}} and \texttt{\textbf{R>}}. Similarly, you must
use both words within or outside of any \texttt{\textbf{DO\ ...\ LOOP}}
structure in that definition. Therefore, these words can not tollerate
tail recursion enhancements, if your system uses those techniques.

Following a \texttt{\textbf{SAVE}}, \texttt{\textbf{R\#}} is on top
of the return stack; if you need a copy of it you may get it by using
\texttt{\textbf{R@}}.

\nwenddocs{}\nwbegincode{5}\sublabel{NWtG4Lr-4HTvUe-1}\nwmargintag{{\nwtagstyle{}\subpageref{NWtG4Lr-4HTvUe-1}}}\moddef{save~{\nwtagstyle{}\subpageref{NWtG4Lr-4HTvUe-1}}}\endmoddef\nwstartdeflinemarkup\nwusesondefline{\\{NWtG4Lr-4ffLeu-1}}\nwenddeflinemarkup
: SAVE ( -- ) ( R: -- f# db# r# )   R>  DB# @ >R F# @ >R R# @ >R  >R ;
\\ NO-TAIL-RECURSION
: RESTORE ( -- ) ( R: f# db# r# -- )   R>  R> R# ! R> F# ! R> DB# !  >R ;
\\ NO-TAIL-RECURSION
\nwused{\\{NWtG4Lr-4ffLeu-1}}\nwendcode{}\nwbegindocs{6}\nwdocspar


\subsubsection{How Data is Stored \label{sub:How-Data-is-Stored}}

The Data Base Support package allows storage of data in either numeric
or alphanumeric form. For instance, a U.S. telephone number, including
area code, requires 14 bytes when stored in alphanumeric form:

\texttt{\textbf{(213) 372-8493}}

This same phone number can be stored in only 6 bytes per record, if
it is recorded as a 16-bit area code and a 32-bit local number:

\texttt{\textbf{213 3728493}}

The appropriate punctuation symbols can easily be inserted when the
number is displayed, using pictured numeric output.

The contents of numeric fields travel between the data stack and the
disk; the contents of alphanumeric strings travel between the \texttt{\textbf{PAD
}}and the disk.

For instance, if we have a double-length field named \texttt{\textbf{SALARY}},
we can fetch the value of the field (from the current record in the
current file) by invoking the phrase:

\texttt{\textbf{SALARY D@}}

which places its value on the stack in the same way that the word
\texttt{\textbf{2@}} fetches a double-length value from an ordinary
variable. Similarly, the phrase,

\texttt{\textbf{SALARY D!}}

removes a double-length value from the stack and places it in the
current \texttt{\textbf{SALARY}} field.

Alternatively, the word \texttt{\textbf{B@}} fetches the contents
of an alphanumeric field, and copies it to the \texttt{\textbf{PAD}}.
The word \texttt{\textbf{B!}} stores an alphanumeric string at \texttt{\textbf{PAD}}
into a given field.

\begin{figure}
\caption{\label{fig:3}\protect\includegraphics{2_home_druffer_Dropbox_ColorForth_File_Docs_Task_Memory.eps}}


\end{figure}


Shows how data travels between disk and memory. Numeric data travels
between disk and the parameter stack, and text strings travel between
the disk and \texttt{\textbf{PAD}}.

\textbf{REFERENCES}

Field Reference Operators, \prettyref{sub:Field-Reference-Operators}


\subsubsection{Working Storage \label{sub:Working-Storage}}

Two features of the Data Base Support package make use of a region
of mem\-ory called \textit{working storage}. Working storage is allocated
at the beginning of a task partition, and serves as a place where
record data may remain which is easily accessible, but less volatile
than \texttt{\textbf{PAD}} or the parameter stack. Since each task
has a working storage area, tasks running concur\-rently may use
the same code referring to working storage without conflict.

One use of working storage appears in automatic totaling (\prettyref{sub:Totals-and-Subtotals}).
Here working storage holds the accumulating registers for each column
of data to be added as the report is generated.

The second use of working storage is as an ``image'' of a record.
The same relative positions are maintained both in the record on disk
and in working storage. For example, working storage is used to hold
the key during a binary search of an ordered index (\prettyref{sub:Ordered-Index-Files}),
in the field in which it will be found in records being searched.

The same field names that let you access fields on disk also may be
used to reference the corresponding fields in local working storage.
There is only one ``record'' in the working storage area. This technique
lets you map data items as though they were contained in records although
they are tempor\-arily in resident memory instead of on the disk.

\begin{figure}
\caption{\label{fig:4}\protect\includegraphics{3_home_druffer_Dropbox_ColorForth_File_Docs_Working_Storage.eps}}


\end{figure}


Field names may be used to reference either the individual field in
the current record (on disk), or the corresponding field in working
storage.

The double use of working storage for both subtotaling and key searches
rarely causes a conflict, since the two activities occur at separate
times. To be on the safe side, however, the statistics component uses
the word \texttt{\textbf{REGISTER}} to return the address of the accumulators,
which in turn is defined in terms of \texttt{\textbf{WORKING}}. If
you find that you will encounter a conflict, you may resolve it by
simply redefining \texttt{\textbf{REGISTER}} to point to some other
area.

How much working storage is necessary? If you are using the subtotaling
feature, the amount depends on the number of accumulating registers
you need. In total, you will need the sum of:
\begin{description}
\item [{16}] register area management
\item [{4}] header variables for registers
\item [{\#registers~{*}~8}] 8 bytes per accumulator
\item [{\_\_\_\_\_\_\_\_\_\_\_\_\_}]~
\item [{total}]~
\end{description}
For instance, three accumulators will require 44 bytes (16\ +\ 4\ +\ 3{*}8).
On 32-bit machines these sizes should be doubled and on 64-bit machines,
multiply times 4.

If you are using ordered indexes, working storage must be as large
as the largest record in any ordered index file.

Remember that any task, which performs an application that uses working
storage must have sufficient room allotted for it---including the
printer task.

\nwenddocs{}\nwbegincode{7}\sublabel{NWtG4Lr-nM8ar-1}\nwmargintag{{\nwtagstyle{}\subpageref{NWtG4Lr-nM8ar-1}}}\moddef{working~{\nwtagstyle{}\subpageref{NWtG4Lr-nM8ar-1}}}\endmoddef\nwstartdeflinemarkup\nwusesondefline{\\{NWtG4Lr-4ffLeu-1}}\nwenddeflinemarkup
#USER 1024 +USER WORKING  TO #USER
\nwused{\\{NWtG4Lr-4ffLeu-1}}\nwendcode{}\nwbegindocs{8}\nwdocspar

\textbf{REFERENCES}

Accessing Fields in Working Storage, \prettyref{sub:Access-to-Working-Storage}\\
Ordered Indexes, \prettyref{sub:Ordered-Index-Files}\\
Subtotaling, \prettyref{sub:Totals-and-Subtotals}


\subsubsection{Installing The Data Base Support Package in polyFORTH \label{sub:Installing-The-Package}}

Before you begin to use the Data Base Support package, you must first
decide whether you will be using it in your personal task only, or
whether other terminal tasks may need to use it simultaneously.

To load the package into a your private terminal task, list its load
block with the phrase,

\texttt{\textbf{FILES LIST}}

The constant FILES return the number of this load block. Make sure
that the block begins with the phrase:

\texttt{\textbf{EMPTY n ALLOT}}

where n is the amount of working storage required for your files application
(see \prettyref{sub:Working-Storage}).

At the end of the load block, a null definition of \texttt{\textbf{TASK}}
should appear. This word will be the last word in the dictionary when
file applications are loaded, and will mark the point at which overlays
will occur.

Now issue the command:

\texttt{\textbf{FILES LOAD}}

to load the Data Base Support package.

As you create data base applications, each of these should begin with
the phrase:

\texttt{\textbf{FORGET TASK : TASK ;}}

This makes each application an overlay, which will discard other over\-lays
that use this convention. For instance, if you have an accounts-receivables
application, its load block should begin with the above phrase, to
forget any other applications without forgetting the Data Base Support
package itself.

Finally, if you wish to output a report to your printer, you must
allot a working storage area in the printer task. This may be done
with the phrase:

\texttt{\textbf{n TYPIST H HIS +!}}

This phrase advances the dictionary pointer \texttt{\textbf{H}} for
\texttt{\textbf{TYPIST}}---the printer task---by the amount \textit{n}.
It is most convenient to edit this phrase into the \texttt{\textbf{FILES}}
load block.

Alternatively, if several terminals require use of the Data Base Support
package, Block 9 should load the package with the system electives.
In this case, remove the:

\texttt{\textbf{EMPTY n ALLOT}}

at the top of the \texttt{\textbf{FILES}} load block; also remove
the definition of \texttt{\textbf{TASK}} and the word \texttt{\textbf{EXIT}}
at the bottom (by placing parentheses around them). By allowing the
final word \texttt{\textbf{GILD}} to execute, the Data Base Support
definitions will become available to all tasks.

Edit the phrase:

\texttt{\textbf{FILES LOAD n TYPIST HIS +!}}

where \textit{n} is the amount of working storage required for your
files application (see \prettyref{sub:Working-Storage}) into the
last line of Block 9 (just above the \texttt{\textbf{EXIT}}). If you
have already loaded the electives before making this addition, type:

\texttt{\textbf{FLUSH RELOAD}}

then type:

\texttt{\textbf{HI}}

Using this approach, each files application must begin with the phrase:

\texttt{\textbf{EMPTY n ALLOT}}

instead of \texttt{\textbf{FORGET\ TASK}}, where \textit{n} is the
amount of working storage needed by that application.


\subsubsection{Installing The Data Base Support Package in gforth }

While the previous section described how to install the original Data
Base Support Package on the 16-bit polyFORTH block-based system where
it was written, the current package has been enhanced to work on 32-bit,
file-based systems. It has been tested, at various points in time,
on the 32-bit pF32-386/pMSD system using its Text File Support (\texttt{\textbf{TEXTFILE}})
Option, on SwiftForth 2.00.2, and on Gforth 0.7.9. However, since
each system has its own, unique dependencies, the current system is
configured to run on gforth, using an ANS Forth compatibility layer
(\texttt{\textbf{pFDatabase.fth}}). The primary compatibility issue
these routines solve is the issue of locating the support files in
a multi-file application package, such as this.

While this could be called an ANS Forth Application, you are not left
with an ANS Forth System after it loads, due to many naming conflicts.
So, it has ben placed it into a \texttt{\textbf{files-wordlist}} that
is made current by the word \texttt{\textbf{Files}}. Thus, all of
the naming conflicts, and most of the application usages are buried
away from normal user interaction. Only those words which are considered
to be application interface words should be exposed to the user. The
developer, however, will need to execute \texttt{\textbf{Files}} before
the words in this plug in can be used.

The following definitions are required by this system, but are not
considered part of it. They may all, eventually, be moved to the supporting
Forth system.

The Data Base Support Package can be compiled into gforth by loading
the following file:

\nwenddocs{}\nwbegincode{9}\sublabel{NWtG4Lr-2V5MiK-1}\nwmargintag{{\nwtagstyle{}\subpageref{NWtG4Lr-2V5MiK-1}}}\moddef{pFDatabase.fth~{\nwtagstyle{}\subpageref{NWtG4Lr-2V5MiK-1}}}\endmoddef\nwstartdeflinemarkup\nwenddeflinemarkup
[defined] 2+! 0= [IF] : 2+! ( d a -- )   DUP >R 2@ D+ R> 2! ; [THEN]

\LA{}append~{\nwtagstyle{}\subpageref{NWtG4Lr-2ZtF2B-1}}\RA{}
\LA{}place~{\nwtagstyle{}\subpageref{NWtG4Lr-24CRHn-1}}\RA{}
: prepend ( from len to -- )   dup count  dup 1+ allocate throw  dup >r place
    dup >r place  r> r@ count rot append  r> free throw ;

\LA{},string~{\nwtagstyle{}\subpageref{NWtG4Lr-4Hk4uw-1}}\RA{}
\LA{}string-c,~{\nwtagstyle{}\subpageref{NWtG4Lr-b9pLA-1}}\RA{}

: null? DUP 0= IF NIP DUP THEN ;

: left-parse-string ( str len char -- rstr rlen lstr llen ) \\ IEEE 1275 parser from left
    OVER IF
        >R 2DUP R> ROT ROT OVER + SWAP 2DUP = IF  2DROP  ELSE  DO
            DUP I C@ = IF
                DROP 2DUP + I 1+ SWAP OVER - null?  ( rstr rlen | 0 0 )
                2SWAP DROP I OVER - null?           ( lstr llen | 0 0 )
                UNLOOP EXIT  THEN
        LOOP THEN  DROP 0 0 2SWAP
    ELSE  DROP 0 0 2SWAP  THEN ;

\LA{}endians~{\nwtagstyle{}\subpageref{NWtG4Lr-1c5Q5W-1}}\RA{}
\LA{}links~{\nwtagstyle{}\subpageref{NWtG4Lr-aGu5T-1}}\RA{}

wordlist constant DBfiles-wordlist

: files   forth-wordlist  DBfiles-wordlist
    2 set-order  definitions ;  immediate
files

\LA{}spin~{\nwtagstyle{}\subpageref{NWtG4Lr-2Cbxl2-1}}\RA{}
\LA{}get~{\nwtagstyle{}\subpageref{NWtG4Lr-101V9w-1}}\RA{}
\LA{}text~{\nwtagstyle{}\subpageref{NWtG4Lr-KJlJC-1}}\RA{}
\LA{}reverse~{\nwtagstyle{}\subpageref{NWtG4Lr-3QgOwL-1}}\RA{}
\LA{}keep-closed~{\nwtagstyle{}\subpageref{NWtG4Lr-4MW32q-1}}\RA{}

: >MOVE< ( s d n -- )
    REVERSE @ IF
        2/ 0 ?DO
            >R  COUNT R@ 1+ C!  COUNT R@ C!  R> 2 +
        LOOP  2DROP
    ELSE  MOVE  THEN ;

: xTYPE ( a n -- )
    BOUNDS ?DO
        I C@ DUP BL < IF
            DROP SPACE
        ELSE  EMIT
    THEN  LOOP ;

\LA{}-text~{\nwtagstyle{}\subpageref{NWtG4Lr-4HDS1q-1}}\RA{}
\LA{}date-format~{\nwtagstyle{}\subpageref{NWtG4Lr-41VtOE-1}}\RA{}
\LA{}time~{\nwtagstyle{}\subpageref{NWtG4Lr-4JKPYw-1}}\RA{}
\LA{}title"~{\nwtagstyle{}\subpageref{NWtG4Lr-4SxyK6-1}}\RA{}
\LA{}company"~{\nwtagstyle{}\subpageref{NWtG4Lr-1Iq9I9-1}}\RA{}
\LA{}app"~{\nwtagstyle{}\subpageref{NWtG4Lr-1ZZNZF-1}}\RA{}
\LA{}number-user~{\nwtagstyle{}\subpageref{NWtG4Lr-2KZt7q-1}}\RA{}
\LA{}+user~{\nwtagstyle{}\subpageref{NWtG4Lr-Zsakx-1}}\RA{}

\LA{}support~{\nwtagstyle{}\subpageref{NWtG4Lr-4ffLeu-1}}\RA{}   \\ DataBase Support System
\LA{}reports~{\nwtagstyle{}\subpageref{NWtG4Lr-1HFlBt-1}}\RA{}   \\ Report Generator
\LA{}structure~{\nwtagstyle{}\subpageref{NWtG4Lr-2dJGUc-1}}\RA{} \\ Structured files
\LA{}memory~{\nwtagstyle{}\subpageref{NWtG4Lr-4K7kUG-1}}\RA{}    \\ Memory based data
\LA{}index~{\nwtagstyle{}\subpageref{NWtG4Lr-2dg69x-1}}\RA{}     \\ Ordered Index
\LA{}people~{\nwtagstyle{}\subpageref{NWtG4Lr-3Y3Oqi-1}}\RA{}    \\ People example
\LA{}personnel~{\nwtagstyle{}\subpageref{NWtG4Lr-1aUp39-1}}\RA{} \\ Personnel example
\LA{}customers~{\nwtagstyle{}\subpageref{NWtG4Lr-i0vIO-1}}\RA{} \\ Customers example
\LA{}accounts~{\nwtagstyle{}\subpageref{NWtG4Lr-34dcAv-1}}\RA{}  \\ Accounts example
\LA{}wines~{\nwtagstyle{}\subpageref{NWtG4Lr-1KI31P-1}}\RA{}     \\ Wine Inventory example
\LA{}glossary~{\nwtagstyle{}\subpageref{NWtG4Lr-2RYM6N-1}}\RA{}  \\ Sample FILE application for documenting Forth words

\LA{}test~{\nwtagstyle{}\subpageref{NWtG4Lr-Joiug-1}}\RA{}      \\ Test the examples

FORTH DEFINITIONS
\nwnotused{pFDatabase.fth}\nwendcode{}\nwbegindocs{10}\nwdocspar

Once loaded, the following, optional configuration settings may be
invoked to cause the system to behave in specific ways.

\texttt{\textbf{FALSE REVERSE !}}

Causes the system to not store strings with every other byte reversed.
This is described in \prettyref{sub:Index-File-Records} below, regarding
Indexed File Records. The default setting is \texttt{\textbf{TRUE}},
since most existing Data Base Support files have been constructed
that way. However, if you are not concerned with compatibility with
existing data files, you will most likely set this flag to \texttt{\textbf{FALSE}}.

\nwenddocs{}\nwbegincode{11}\sublabel{NWtG4Lr-3QgOwL-1}\nwmargintag{{\nwtagstyle{}\subpageref{NWtG4Lr-3QgOwL-1}}}\moddef{reverse~{\nwtagstyle{}\subpageref{NWtG4Lr-3QgOwL-1}}}\endmoddef\nwstartdeflinemarkup\nwusesondefline{\\{NWtG4Lr-2V5MiK-1}}\nwenddeflinemarkup
VARIABLE REVERSE   ( Default ) FALSE REVERSE !
\nwused{\\{NWtG4Lr-2V5MiK-1}}\nwendcode{}\nwbegindocs{12}\nwdocspar

\begin{description}
\item [{\texttt{KEEP-CLOSED}}] when false, file handles are never explicitely
closed unless a new file name is bound to a file. When true, file
handles are closed after each major operation.
\end{description}
\nwenddocs{}\nwbegincode{13}\sublabel{NWtG4Lr-4MW32q-1}\nwmargintag{{\nwtagstyle{}\subpageref{NWtG4Lr-4MW32q-1}}}\moddef{keep-closed~{\nwtagstyle{}\subpageref{NWtG4Lr-4MW32q-1}}}\endmoddef\nwstartdeflinemarkup\nwusesondefline{\\{NWtG4Lr-2V5MiK-1}}\nwenddeflinemarkup
VARIABLE KEEP-CLOSED   ( Default ) FALSE KEEP-CLOSED !
[defined] ForTimbre [IF]  TRUE KEEP-CLOSED !  [THEN]
\nwused{\\{NWtG4Lr-2V5MiK-1}}\nwendcode{}\nwbegindocs{14}\nwdocspar

\texttt{\textbf{LITTLE-ENDIAN}}\textbf{}\\
\texttt{\textbf{BIG-ENDIAN}}

These options establish the byte order of values stored in \texttt{\textbf{NUMERIC}},
\texttt{\textbf{LONG}} and \texttt{\textbf{DOUBLE}} fields. The default
setting is \texttt{\textbf{LITTLE-ENDIAN}}, but you may want to use
\texttt{\textbf{BIG-ENDIAN}} on systems where that is the native data
format. Neither one has any speed benefit, since both are coded using
byte fetch operators. However, if you are accessing data in \texttt{\textbf{WORKING}}
storage, as described in \prettyref{sub:Access-to-Working-Storage},
this setting can be essential.

\texttt{\textbf{' DD-MMM-YYYY DATE-FORMAT !}}\textbf{}\\
\texttt{\textbf{' MM/DD/YYYY DATE-FORMAT !}}

The format of a date can be adjusted, using this option. The default
setting is \texttt{\textbf{MM/DD/YYYY}}. The primary usage is in \texttt{\textbf{+PAGE}}
as is described in \prettyref{sub:The-Page-Banner}.

Modified Julian Dates in the range 01 Jan 1900 through 05 Jun 2079
may be input and output converted, although dates in the first two
months of 1900 are not consistent with the rest of the calendar because
1900 was not a leap year.

\nwenddocs{}\nwbegincode{15}\sublabel{NWtG4Lr-41VtOE-1}\nwmargintag{{\nwtagstyle{}\subpageref{NWtG4Lr-41VtOE-1}}}\moddef{date-format~{\nwtagstyle{}\subpageref{NWtG4Lr-41VtOE-1}}}\endmoddef\nwstartdeflinemarkup\nwusesondefline{\\{NWtG4Lr-2V5MiK-1}}\nwenddeflinemarkup
\LA{}dy~{\nwtagstyle{}\subpageref{NWtG4Lr-478QOO-1}}\RA{}
\LA{}days~{\nwtagstyle{}\subpageref{NWtG4Lr-3D4RV7-1}}\RA{}
\LA{}fetch-mth~{\nwtagstyle{}\subpageref{NWtG4Lr-2xfvjX-1}}\RA{}
\LA{}dmy~{\nwtagstyle{}\subpageref{NWtG4Lr-46kc96-1}}\RA{}
\LA{}a.d.~{\nwtagstyle{}\subpageref{NWtG4Lr-2AFXuz-1}}\RA{}
\LA{}mth~{\nwtagstyle{}\subpageref{NWtG4Lr-UYpfI-1}}\RA{}
\LA{}mdy~{\nwtagstyle{}\subpageref{NWtG4Lr-478cHs-1}}\RA{}
\LA{}y-dd-dm~{\nwtagstyle{}\subpageref{NWtG4Lr-2JA1Oz-1}}\RA{}
\LA{}dd-mmm-yyyy~{\nwtagstyle{}\subpageref{NWtG4Lr-YNC8v-1}}\RA{}
\LA{}mmddyyyy~{\nwtagstyle{}\subpageref{NWtG4Lr-3b82bI-1}}\RA{}

VARIABLE DATE-FORMAT   ' MM/DD/YYYY DATE-FORMAT !

\LA{}.date~{\nwtagstyle{}\subpageref{NWtG4Lr-3H8kGz-1}}\RA{}
\LA{}fetch-date~{\nwtagstyle{}\subpageref{NWtG4Lr-1MB56x-1}}\RA{}
\nwused{\\{NWtG4Lr-2V5MiK-1}}\nwendcode{}\nwbegindocs{16}\nwdocspar

\begin{description}
\item [{\texttt{D/Y}}] is the number of days per year for a four-year period.
\end{description}
\nwenddocs{}\nwbegincode{17}\sublabel{NWtG4Lr-478QOO-1}\nwmargintag{{\nwtagstyle{}\subpageref{NWtG4Lr-478QOO-1}}}\moddef{dy~{\nwtagstyle{}\subpageref{NWtG4Lr-478QOO-1}}}\endmoddef\nwstartdeflinemarkup\nwusesondefline{\\{NWtG4Lr-41VtOE-1}}\nwenddeflinemarkup
365 4 * 1+ CONSTANT D/Y
\nwused{\\{NWtG4Lr-41VtOE-1}}\nwendcode{}\nwbegindocs{18}\nwdocspar

\begin{description}
\item [{\texttt{DAYS}}] is the lookup table of total days in the year at
the start of each month.
\end{description}
\nwenddocs{}\nwbegincode{19}\sublabel{NWtG4Lr-3D4RV7-1}\nwmargintag{{\nwtagstyle{}\subpageref{NWtG4Lr-3D4RV7-1}}}\moddef{days~{\nwtagstyle{}\subpageref{NWtG4Lr-3D4RV7-1}}}\endmoddef\nwstartdeflinemarkup\nwusesondefline{\\{NWtG4Lr-41VtOE-1}}\nwenddeflinemarkup
CREATE DAYS   -1 ,  0 ,  31 ,  59 ,  90 ,  120 ,  151 ,
        181 ,  212 ,  243 ,  273 ,  304 ,  334 ,  367 ,
\nwused{\\{NWtG4Lr-41VtOE-1}}\nwendcode{}\nwbegindocs{20}\nwdocspar

\begin{description}
\item [{\texttt{@MTH}}] returns the value from days for the given month.
\end{description}
\nwenddocs{}\nwbegincode{21}\sublabel{NWtG4Lr-2xfvjX-1}\nwmargintag{{\nwtagstyle{}\subpageref{NWtG4Lr-2xfvjX-1}}}\moddef{fetch-mth~{\nwtagstyle{}\subpageref{NWtG4Lr-2xfvjX-1}}}\endmoddef\nwstartdeflinemarkup\nwusesondefline{\\{NWtG4Lr-41VtOE-1}}\nwenddeflinemarkup
: @MTH ( u1 -- u2 )   CELLS DAYS + @ ;
\nwused{\\{NWtG4Lr-41VtOE-1}}\nwendcode{}\nwbegindocs{22}\nwdocspar

\begin{description}
\item [{\texttt{D/M/Y}}] converts day, month, year into MJD.
\end{description}
\nwenddocs{}\nwbegincode{23}\sublabel{NWtG4Lr-46kc96-1}\nwmargintag{{\nwtagstyle{}\subpageref{NWtG4Lr-46kc96-1}}}\moddef{dmy~{\nwtagstyle{}\subpageref{NWtG4Lr-46kc96-1}}}\endmoddef\nwstartdeflinemarkup\nwusesondefline{\\{NWtG4Lr-41VtOE-1}}\nwenddeflinemarkup
: D-M-Y ( d m y -- u )   >R  @MTH
    58 OVER < IF  R@ 3 AND 0= - THEN + 1-
    R> 1900 -  D/Y UM*  4 UM/MOD SWAP 0<> - + ;
\nwused{\\{NWtG4Lr-41VtOE-1}}\nwendcode{}\nwbegindocs{24}\nwdocspar

\begin{description}
\item [{\texttt{A.D.}}] sets the current \texttt{\textbf{YEAR}}.
\end{description}
\nwenddocs{}\nwbegincode{25}\sublabel{NWtG4Lr-2AFXuz-1}\nwmargintag{{\nwtagstyle{}\subpageref{NWtG4Lr-2AFXuz-1}}}\moddef{a.d.~{\nwtagstyle{}\subpageref{NWtG4Lr-2AFXuz-1}}}\endmoddef\nwstartdeflinemarkup\nwusesondefline{\\{NWtG4Lr-41VtOE-1}}\nwenddeflinemarkup
VARIABLE YEAR

: A.D. ( n -- )   YEAR ! ;   TIME&DATE A.D. 2DROP 2DROP DROP
\nwused{\\{NWtG4Lr-41VtOE-1}}\nwendcode{}\nwbegindocs{26}\nwdocspar

\begin{description}
\item [{\texttt{MTH}}] defines words that convert date to internal form
in month of the same name using year in \texttt{\textbf{YEAR}}.
\end{description}
\nwenddocs{}\nwbegincode{27}\sublabel{NWtG4Lr-UYpfI-1}\nwmargintag{{\nwtagstyle{}\subpageref{NWtG4Lr-UYpfI-1}}}\moddef{mth~{\nwtagstyle{}\subpageref{NWtG4Lr-UYpfI-1}}}\endmoddef\nwstartdeflinemarkup\nwusesondefline{\\{NWtG4Lr-41VtOE-1}}\nwenddeflinemarkup
: MTH ( n _ -- )   CREATE , DOES> ( d -- n )   @ YEAR @ D-M-Y ;

1 MTH JAN       2 MTH FEB       3 MTH MAR       4 MTH APR
5 MTH MAY       6 MTH JUN       7 MTH JUL       8 MTH AUG
9 MTH SEP      10 MTH OCT      11 MTH NOV      12 MTH DEC
\nwused{\\{NWtG4Lr-41VtOE-1}}\nwendcode{}\nwbegindocs{28}\nwdocspar

\begin{description}
\item [{\texttt{M/D/Y}}] takes a double number mm/dd/yyyy and converts
it to MJD.
\end{description}
\nwenddocs{}\nwbegincode{29}\sublabel{NWtG4Lr-478cHs-1}\nwmargintag{{\nwtagstyle{}\subpageref{NWtG4Lr-478cHs-1}}}\moddef{mdy~{\nwtagstyle{}\subpageref{NWtG4Lr-478cHs-1}}}\endmoddef\nwstartdeflinemarkup\nwusesondefline{\\{NWtG4Lr-41VtOE-1}}\nwenddeflinemarkup
: M/D/Y ( ud -- u )   10000 UM/MOD  100 /MOD  ROT D-M-Y ;
\nwused{\\{NWtG4Lr-41VtOE-1}}\nwendcode{}\nwbegindocs{30}\nwdocspar

\begin{description}
\item [{\texttt{Y-DD~DM}}] split the serial number back into its components.
\end{description}
\nwenddocs{}\nwbegincode{31}\sublabel{NWtG4Lr-2JA1Oz-1}\nwmargintag{{\nwtagstyle{}\subpageref{NWtG4Lr-2JA1Oz-1}}}\moddef{y-dd-dm~{\nwtagstyle{}\subpageref{NWtG4Lr-2JA1Oz-1}}}\endmoddef\nwstartdeflinemarkup\nwusesondefline{\\{NWtG4Lr-41VtOE-1}}\nwenddeflinemarkup
: Y-DD ( u1 -- y u2 u3 )   4 UM* D/Y  UM/MOD 1900 +  SWAP 4 /MOD 1+
    DUP ROT 0= IF  DUP 60 > +  SWAP DUP 59 > +  THEN ;

: DM ( u1 u2 -- d m )   1 BEGIN  1+  2DUP @MTH > 0= UNTIL  1-
    SWAP DROP SWAP  OVER @MTH - SWAP ;
\nwused{\\{NWtG4Lr-41VtOE-1}}\nwendcode{}\nwbegindocs{32}\nwdocspar

\begin{description}
\item [{\texttt{DD-MMM-YYYY}}] converts internal form to text formatted
as dd mmm yyyy.
\end{description}
\nwenddocs{}\nwbegincode{33}\sublabel{NWtG4Lr-YNC8v-1}\nwmargintag{{\nwtagstyle{}\subpageref{NWtG4Lr-YNC8v-1}}}\moddef{dd-mmm-yyyy~{\nwtagstyle{}\subpageref{NWtG4Lr-YNC8v-1}}}\endmoddef\nwstartdeflinemarkup\nwusesondefline{\\{NWtG4Lr-41VtOE-1}}\nwenddeflinemarkup
CREATE MTHS   s" JanFebMarAprMayJunJulAugSepOctNovDec" $c,

: DD-MMM-YYYY ( n -- a n )   BASE @ >R  DECIMAL  Y-DD
    ROT 0 <#  # # # #  2DROP  BL HOLD  DM 1- 3 * MTHS +
    3 + 3 0 DO  1- DUP C@ HOLD  LOOP  DROP
    BL HOLD  0 # #  #>  R> BASE ! ;
\nwused{\\{NWtG4Lr-41VtOE-1}}\nwendcode{}\nwbegindocs{34}\nwdocspar

\begin{description}
\item [{\texttt{MM/DD/YYYY}}] converts internal form to text formatted
as mm/dd/yyyy.
\end{description}
\nwenddocs{}\nwbegincode{35}\sublabel{NWtG4Lr-3b82bI-1}\nwmargintag{{\nwtagstyle{}\subpageref{NWtG4Lr-3b82bI-1}}}\moddef{mmddyyyy~{\nwtagstyle{}\subpageref{NWtG4Lr-3b82bI-1}}}\endmoddef\nwstartdeflinemarkup\nwusesondefline{\\{NWtG4Lr-41VtOE-1}}\nwenddeflinemarkup
: MM/DD/YYYY ( u1 -- c-addr u2 )   BASE @ >R  DECIMAL  Y-DD
    ROT 0 <#  # # # #  2DROP  [char] / HOLD  DM SWAP
    0 # #  2DROP   [char] / HOLD  0 # #  #>  R> BASE ! ;
\nwused{\\{NWtG4Lr-41VtOE-1}}\nwendcode{}\nwbegindocs{36}\nwdocspar

\begin{description}
\item [{\texttt{.DATE}}] displays the system date (u) as mm/dd/yyyy.
\end{description}
\nwenddocs{}\nwbegincode{37}\sublabel{NWtG4Lr-3H8kGz-1}\nwmargintag{{\nwtagstyle{}\subpageref{NWtG4Lr-3H8kGz-1}}}\moddef{.date~{\nwtagstyle{}\subpageref{NWtG4Lr-3H8kGz-1}}}\endmoddef\nwstartdeflinemarkup\nwusesondefline{\\{NWtG4Lr-41VtOE-1}}\nwenddeflinemarkup
: .DATE ( u -- )   DATE-FORMAT @ EXECUTE TYPE SPACE ;
\nwused{\\{NWtG4Lr-41VtOE-1}}\nwendcode{}\nwbegindocs{38}\nwdocspar

\begin{description}
\item [{\texttt{@DATE}}] gets the current system date.
\end{description}
\nwenddocs{}\nwbegincode{39}\sublabel{NWtG4Lr-1MB56x-1}\nwmargintag{{\nwtagstyle{}\subpageref{NWtG4Lr-1MB56x-1}}}\moddef{fetch-date~{\nwtagstyle{}\subpageref{NWtG4Lr-1MB56x-1}}}\endmoddef\nwstartdeflinemarkup\nwusesondefline{\\{NWtG4Lr-41VtOE-1}}\nwenddeflinemarkup
: @DATE ( -- n )   TIME&DATE D-M-Y NIP NIP NIP ;

: DATE ( -- )   @DATE .DATE ;
\nwused{\\{NWtG4Lr-41VtOE-1}}\nwendcode{}\nwbegindocs{40}\nwdocspar
\nwenddocs{}\nwbegincode{41}\sublabel{NWtG4Lr-4JKPYw-1}\nwmargintag{{\nwtagstyle{}\subpageref{NWtG4Lr-4JKPYw-1}}}\moddef{time~{\nwtagstyle{}\subpageref{NWtG4Lr-4JKPYw-1}}}\endmoddef\nwstartdeflinemarkup\nwusesondefline{\\{NWtG4Lr-2V5MiK-1}}\nwenddeflinemarkup
: :00 ( ud1 -- ud2)   DECIMAL  #  6 BASE !  # [char] : HOLD ;
: (TIME) ( secs -- c-addr u)   BASE @ >R  0 <#  :00 :00
    DECIMAL # #  #>  R> BASE ! ;
: .TIME ( secs -- )   (TIME) TYPE SPACE ;

: @TIME ( -- secs )   TIME&DATE 2DROP DROP 60 * + 60 * + ;

: TIME ( -- )   @TIME .TIME ;
\nwused{\\{NWtG4Lr-2V5MiK-1}}\nwendcode{}\nwbegindocs{42}\nwdocspar

\begin{description}
\item [{\texttt{TITLE\textquotedbl{}}}] defines the string to be used for
listings. When its children are executed, they leave the string address
in \texttt{\textbf{'TITLE}}.
\end{description}
\nwenddocs{}\nwbegincode{43}\sublabel{NWtG4Lr-4SxyK6-1}\nwmargintag{{\nwtagstyle{}\subpageref{NWtG4Lr-4SxyK6-1}}}\moddef{title"~{\nwtagstyle{}\subpageref{NWtG4Lr-4SxyK6-1}}}\endmoddef\nwstartdeflinemarkup\nwusesondefline{\\{NWtG4Lr-2V5MiK-1}}\nwenddeflinemarkup
VARIABLE 'TITLE

: TITLE" ( -- )   CREATE  HERE 'TITLE !  ,"
    DOES> ( -- )   'TITLE ! ;
\nwused{\\{NWtG4Lr-2V5MiK-1}}\nwendcode{}\nwbegindocs{44}\nwdocspar

\begin{description}
\item [{\texttt{COMPANY\textquotedbl{}}}] sets the company name used for
report titles.
\end{description}
\nwenddocs{}\nwbegincode{45}\sublabel{NWtG4Lr-1Iq9I9-1}\nwmargintag{{\nwtagstyle{}\subpageref{NWtG4Lr-1Iq9I9-1}}}\moddef{company"~{\nwtagstyle{}\subpageref{NWtG4Lr-1Iq9I9-1}}}\endmoddef\nwstartdeflinemarkup\nwusesondefline{\\{NWtG4Lr-2V5MiK-1}}\nwenddeflinemarkup
VARIABLE 'COMPANY

: COMPANY" ( -- )   HERE 'COMPANY !  ," ;

COMPANY" Software Specialists"
\nwused{\\{NWtG4Lr-2V5MiK-1}}\nwendcode{}\nwbegindocs{46}\nwdocspar

\begin{description}
\item [{\texttt{APP\textquotedbl{}}}] sets the application name used for
screen titles.
\end{description}
\nwenddocs{}\nwbegincode{47}\sublabel{NWtG4Lr-1ZZNZF-1}\nwmargintag{{\nwtagstyle{}\subpageref{NWtG4Lr-1ZZNZF-1}}}\moddef{app"~{\nwtagstyle{}\subpageref{NWtG4Lr-1ZZNZF-1}}}\endmoddef\nwstartdeflinemarkup\nwusesondefline{\\{NWtG4Lr-2V5MiK-1}}\nwenddeflinemarkup
VARIABLE 'APP

: APP" ( -- )   HERE 'APP !  ," ;

APP"  "
\nwused{\\{NWtG4Lr-2V5MiK-1}}\nwendcode{}\nwbegindocs{48}\nwdocspar

\textbf{REFERENCES}

\texttt{\textbf{B!}}, \prettyref{sub:Field-Reference-Operators}\\
\texttt{\textbf{WORKING}}, \prettyref{sub:Access-to-Working-Storage}\\
\texttt{\textbf{+PAGE}}, \prettyref{sub:The-Page-Banner} 


\subsubsection{Installing The Data Base Support Package in Open Firmware }

This documentation, 1275 Binding for a Database Package, was never
actually published. It is the last time that I worked on the documentation
with Elizabeth Rather, with the hope to make this tool public, but
the effort died when Apple switched to Intel. Since the Data Base
Support Package was originally written to run on polyFORTH, there
are dependencies, above and beyond simple ANS Forth requirements.
These have all been bracketed with the following:

\texttt{\textbf{{[}DEFINED{]} <name> 0= {[}IF{]} <definition> {[}THEN{]}}}

Here is a list of the non-ANS Forth words that this system requires:
\begin{description}
\item [{Word}] \textbf{Description}
\item [{\texttt{GET}}] (\ a\ --\ ) Multi-tasking means to get ownership
of a facility.
\end{description}
\nwenddocs{}\nwbegincode{49}\sublabel{NWtG4Lr-101V9w-1}\nwmargintag{{\nwtagstyle{}\subpageref{NWtG4Lr-101V9w-1}}}\moddef{get~{\nwtagstyle{}\subpageref{NWtG4Lr-101V9w-1}}}\endmoddef\nwstartdeflinemarkup\nwusesondefline{\\{NWtG4Lr-2V5MiK-1}}\nwenddeflinemarkup
[defined] GET 0= [IF] : GET DROP ; [THEN]
\LA{}grab~{\nwtagstyle{}\subpageref{NWtG4Lr-1YVM30-1}}\RA{}
\LA{}release~{\nwtagstyle{}\subpageref{NWtG4Lr-qFr4D-1}}\RA{}
\nwused{\\{NWtG4Lr-2V5MiK-1}}\nwendcode{}\nwbegindocs{50}\nwdocspar

\begin{description}
\item [{\texttt{GRAB}}] (\ a\ --\ ) Multi-tasking means to get ownership
without pausing.
\end{description}
\nwenddocs{}\nwbegincode{51}\sublabel{NWtG4Lr-1YVM30-1}\nwmargintag{{\nwtagstyle{}\subpageref{NWtG4Lr-1YVM30-1}}}\moddef{grab~{\nwtagstyle{}\subpageref{NWtG4Lr-1YVM30-1}}}\endmoddef\nwstartdeflinemarkup\nwusesondefline{\\{NWtG4Lr-101V9w-1}}\nwenddeflinemarkup
[defined] GRAB 0= [IF] : GRAB DROP ; [THEN]
\nwused{\\{NWtG4Lr-101V9w-1}}\nwendcode{}\nwbegindocs{52}\nwdocspar

\begin{description}
\item [{\texttt{RELEASE}}] (\ a\ --\ ) Multi-tasking means to release
ownership of a facility.
\end{description}
\nwenddocs{}\nwbegincode{53}\sublabel{NWtG4Lr-qFr4D-1}\nwmargintag{{\nwtagstyle{}\subpageref{NWtG4Lr-qFr4D-1}}}\moddef{release~{\nwtagstyle{}\subpageref{NWtG4Lr-qFr4D-1}}}\endmoddef\nwstartdeflinemarkup\nwusesondefline{\\{NWtG4Lr-101V9w-1}}\nwenddeflinemarkup
[defined] RELEASE 0= [IF] : RELEASE DROP ; [THEN]
\nwused{\\{NWtG4Lr-101V9w-1}}\nwendcode{}\nwbegindocs{54}\nwdocspar

\begin{description}
\item [{\texttt{\#USER}}] (\ -- a\ ) Multi-tasking variable containing
the size of the user area.
\end{description}
\nwenddocs{}\nwbegincode{55}\sublabel{NWtG4Lr-2KZt7q-1}\nwmargintag{{\nwtagstyle{}\subpageref{NWtG4Lr-2KZt7q-1}}}\moddef{number-user~{\nwtagstyle{}\subpageref{NWtG4Lr-2KZt7q-1}}}\endmoddef\nwstartdeflinemarkup\nwusesondefline{\\{NWtG4Lr-2V5MiK-1}}\nwenddeflinemarkup
[defined] #USER 0= [IF]  0 VALUE #USER  [THEN]
\nwused{\\{NWtG4Lr-2V5MiK-1}}\nwendcode{}\nwbegindocs{56}\nwdocspar

\begin{description}
\item [{\texttt{+USER}}] (\ o n\ -- o'\ ) Multi-tasking means to define
new user variables.
\end{description}
\nwenddocs{}\nwbegincode{57}\sublabel{NWtG4Lr-Zsakx-1}\nwmargintag{{\nwtagstyle{}\subpageref{NWtG4Lr-Zsakx-1}}}\moddef{+user~{\nwtagstyle{}\subpageref{NWtG4Lr-Zsakx-1}}}\endmoddef\nwstartdeflinemarkup\nwusesondefline{\\{NWtG4Lr-2V5MiK-1}}\nwenddeflinemarkup
[defined] +USER 0= [IF]
    : +USER ( o n _ -- o+n )   CREATE DUP ALLOT + ;
[THEN]
\nwused{\\{NWtG4Lr-2V5MiK-1}}\nwendcode{}\nwbegindocs{58}\nwdocspar

The following are used to fetch data that is in known Endian format.
E.g. in file system structures or network packets. These words work
on un-aligned entities.
\begin{description}
\item [{\texttt{c@-le}}] (\ a n\ -- \ldots{}\ ) Little endian fetch
of a multi-byte value.
\end{description}
\nwenddocs{}\nwbegincode{59}\sublabel{NWtG4Lr-1zZfnn-1}\nwmargintag{{\nwtagstyle{}\subpageref{NWtG4Lr-1zZfnn-1}}}\moddef{c@-le~{\nwtagstyle{}\subpageref{NWtG4Lr-1zZfnn-1}}}\endmoddef\nwstartdeflinemarkup\nwusesondefline{\\{NWtG4Lr-1c5Q5W-1}}\nwenddeflinemarkup
[defined] c@-le 0= [IF]
: 1c@-le ( a n -- x )   0 SWAP BEGIN ?DUP WHILE
        1- ROT 2DUP +      C@ >R ROT 8 LSHIFT R> + ROT
    REPEAT NIP ;
: c@-le ( a n -- ... )
    CELL /MOD ROT OVER CELLS + -ROT SWAP ?DUP
    IF  ROT DUP ROT 1c@-le -ROT SWAP  THEN  ?DUP
    IF  0 DO  1 CELLS - DUP CELL 1c@-le SWAP  LOOP
    THEN  DROP ;
[THEN]
\nwused{\\{NWtG4Lr-1c5Q5W-1}}\nwendcode{}\nwbegindocs{60}\nwdocspar

\begin{description}
\item [{\texttt{c!-le}}] (\ \ldots{}a n\ --\ ) Little endian store
of a multi-byte value.
\end{description}
\nwenddocs{}\nwbegincode{61}\sublabel{NWtG4Lr-1FK9pC-1}\nwmargintag{{\nwtagstyle{}\subpageref{NWtG4Lr-1FK9pC-1}}}\moddef{c!-le~{\nwtagstyle{}\subpageref{NWtG4Lr-1FK9pC-1}}}\endmoddef\nwstartdeflinemarkup\nwusesondefline{\\{NWtG4Lr-1c5Q5W-1}}\nwenddeflinemarkup
[defined] c!-le 0= [IF]
: 1c!-le ( x a n -- )   BEGIN ?DUP WHILE
        1- ROT DUP 8 RSHIFT SWAP 2OVER DROP C! ROT 1+ ROT
    REPEAT 2DROP ;
: c!-le ( ... a n -- )   CELL /MOD SWAP >R  ?DUP
    IF  0 DO  SWAP OVER CELL 1c!-le CELL+ LOOP
    THEN  R> ?DUP IF  1c!-le  ELSE DROP THEN ;
[THEN]
\nwused{\\{NWtG4Lr-1c5Q5W-1}}\nwendcode{}\nwbegindocs{62}\nwdocspar

\begin{description}
\item [{\texttt{c@-be}}] (\ a n\ -- \ldots{}\ ) Big endian fetch of
a multi-byte value.
\end{description}
\nwenddocs{}\nwbegincode{63}\sublabel{NWtG4Lr-1zG2w7-1}\nwmargintag{{\nwtagstyle{}\subpageref{NWtG4Lr-1zG2w7-1}}}\moddef{c@-be~{\nwtagstyle{}\subpageref{NWtG4Lr-1zG2w7-1}}}\endmoddef\nwstartdeflinemarkup\nwusesondefline{\\{NWtG4Lr-1c5Q5W-1}}\nwenddeflinemarkup
[defined] c@-be 0= [IF]
: 1c@-be ( a n -- x )   0 SWAP BEGIN ?DUP WHILE
        1- ROT DUP 1+ SWAP C@ >R ROT 8 LSHIFT R> + ROT
    REPEAT NIP ;
: c@-be ( a n -- ... )
    CELL /MOD ROT OVER CELLS + -ROT SWAP ?DUP
    IF  ROT DUP ROT 1c@-be -ROT SWAP  THEN  ?DUP
    IF  0 DO  1 CELLS - DUP CELL 1c@-be SWAP  LOOP
    THEN  DROP ;
[THEN]
\nwused{\\{NWtG4Lr-1c5Q5W-1}}\nwendcode{}\nwbegindocs{64}\nwdocspar

\begin{description}
\item [{\texttt{c!-be}}] (\ \ldots{}a n\ --\ ) Big endian store of
a multi-byte value.
\end{description}
\nwenddocs{}\nwbegincode{65}\sublabel{NWtG4Lr-1FVKR8-1}\nwmargintag{{\nwtagstyle{}\subpageref{NWtG4Lr-1FVKR8-1}}}\moddef{c!-be~{\nwtagstyle{}\subpageref{NWtG4Lr-1FVKR8-1}}}\endmoddef\nwstartdeflinemarkup\nwusesondefline{\\{NWtG4Lr-1c5Q5W-1}}\nwenddeflinemarkup
[defined] c!-be 0= [IF]
: 1c!-be ( x a n -- )   BEGIN ?DUP WHILE
        1- ROT DUP 8 RSHIFT SWAP 2OVER +    C! ROT    ROT
    REPEAT 2DROP ;
: c!-be ( ... a n -- )   CELL /MOD SWAP >R  ?DUP
    IF  0 DO  SWAP OVER CELL 1c!-be CELL+ LOOP
    THEN  R> ?DUP IF  1c!-be  ELSE DROP THEN ;
[THEN]
\nwused{\\{NWtG4Lr-1c5Q5W-1}}\nwendcode{}\nwbegindocs{66}\nwdocspar
\nwenddocs{}\nwbegincode{67}\sublabel{NWtG4Lr-1c5Q5W-1}\nwmargintag{{\nwtagstyle{}\subpageref{NWtG4Lr-1c5Q5W-1}}}\moddef{endians~{\nwtagstyle{}\subpageref{NWtG4Lr-1c5Q5W-1}}}\endmoddef\nwstartdeflinemarkup\nwusesondefline{\\{NWtG4Lr-2V5MiK-1}}\nwenddeflinemarkup
\LA{}c!-le~{\nwtagstyle{}\subpageref{NWtG4Lr-1FK9pC-1}}\RA{}
\LA{}c!-be~{\nwtagstyle{}\subpageref{NWtG4Lr-1FVKR8-1}}\RA{}
\LA{}c@-le~{\nwtagstyle{}\subpageref{NWtG4Lr-1zZfnn-1}}\RA{}
\LA{}c@-be~{\nwtagstyle{}\subpageref{NWtG4Lr-1zG2w7-1}}\RA{}

VARIABLE c@-xx
VARIABLE c!-xx

: nC@ ( a n -- u|d )   c@-xx @ EXECUTE ;
: nC! ( u|d a n -- )   c!-xx @ EXECUTE ;

: LITTLE-ENDIAN ( -- )   ['] c@-le c@-xx !  ['] c!-le c!-xx ! ;
:    BIG-ENDIAN ( -- )   ['] c@-be c@-xx !  ['] c!-be c!-xx ! ;

: LITTLE-ENDIAN? ( -- flag )   ['] c@-le c@-xx @ = ;

BIG-ENDIAN \\ Default
\nwused{\\{NWtG4Lr-2V5MiK-1}}\nwendcode{}\nwbegindocs{68}\nwdocspar

\begin{description}
\item [{\texttt{@LINK~!LINK~,LINK}}] Do relocation translation, if needed.
\end{description}
\nwenddocs{}\nwbegincode{69}\sublabel{NWtG4Lr-41t9JJ-1}\nwmargintag{{\nwtagstyle{}\subpageref{NWtG4Lr-41t9JJ-1}}}\moddef{@!,link~{\nwtagstyle{}\subpageref{NWtG4Lr-41t9JJ-1}}}\endmoddef\nwstartdeflinemarkup\nwusesondefline{\\{NWtG4Lr-aGu5T-1}}\nwenddeflinemarkup
: @LINK ( a -- a )   [defined] @REL [IF]
        @REL  [ELSE]  @  [THEN] ;
: !LINK ( a1 a2 -- )   [defined] !REL [IF]
        !REL  [ELSE]  !  [THEN] ;
: ,LINK ( a -- )   [defined] ,REL [IF]
        ,REL  [ELSE]  ,  [THEN] ;
\nwused{\\{NWtG4Lr-aGu5T-1}}\nwendcode{}\nwbegindocs{70}\nwdocspar

\begin{description}
\item [{\texttt{LINKS}}] (\ a\ -- a'\ ) Searches the linked list until
it finds the last entry in the list (the one with a 0 link).
\end{description}
\nwenddocs{}\nwbegincode{71}\sublabel{NWtG4Lr-aGu5T-1}\nwmargintag{{\nwtagstyle{}\subpageref{NWtG4Lr-aGu5T-1}}}\moddef{links~{\nwtagstyle{}\subpageref{NWtG4Lr-aGu5T-1}}}\endmoddef\nwstartdeflinemarkup\nwusesondefline{\\{NWtG4Lr-2V5MiK-1}}\nwenddeflinemarkup
\LA{}@!,link~{\nwtagstyle{}\subpageref{NWtG4Lr-41t9JJ-1}}\RA{}

: LINKS ( a -- a' )   BEGIN  DUP @LINK ?DUP WHILE  NIP  REPEAT ;

\LA{}>link~{\nwtagstyle{}\subpageref{NWtG4Lr-1CisWZ-1}}\RA{}
\LA{}<link~{\nwtagstyle{}\subpageref{NWtG4Lr-r5XgZ-1}}\RA{}
\LA{}unlink~{\nwtagstyle{}\subpageref{NWtG4Lr-1nyRGB-1}}\RA{}
\LA{}calls~{\nwtagstyle{}\subpageref{NWtG4Lr-16qzin-1}}\RA{}
\nwused{\\{NWtG4Lr-2V5MiK-1}}\nwendcode{}\nwbegindocs{72}\nwdocspar

\begin{description}
\item [{\texttt{>LINK}}] (\ a\ --\ ) Adds the top of the dictionary
to the given linked list.
\end{description}
\nwenddocs{}\nwbegincode{73}\sublabel{NWtG4Lr-1CisWZ-1}\nwmargintag{{\nwtagstyle{}\subpageref{NWtG4Lr-1CisWZ-1}}}\moddef{>link~{\nwtagstyle{}\subpageref{NWtG4Lr-1CisWZ-1}}}\endmoddef\nwstartdeflinemarkup\nwusesondefline{\\{NWtG4Lr-aGu5T-1}}\nwenddeflinemarkup
: >LINK ( a -- )   ALIGN HERE  OVER @LINK ,LINK  SWAP !LINK ;
\nwused{\\{NWtG4Lr-aGu5T-1}}\nwendcode{}\nwbegindocs{74}\nwdocspar

\begin{description}
\item [{\texttt{<LINK}}] (\ a\ --\ ) Adds the top of the dictionary
to the end of the given linked list.
\end{description}
\nwenddocs{}\nwbegincode{75}\sublabel{NWtG4Lr-r5XgZ-1}\nwmargintag{{\nwtagstyle{}\subpageref{NWtG4Lr-r5XgZ-1}}}\moddef{<link~{\nwtagstyle{}\subpageref{NWtG4Lr-r5XgZ-1}}}\endmoddef\nwstartdeflinemarkup\nwusesondefline{\\{NWtG4Lr-aGu5T-1}}\nwenddeflinemarkup
: <LINK ( a -- )   LINKS  >LINK ;
\nwused{\\{NWtG4Lr-aGu5T-1}}\nwendcode{}\nwbegindocs{76}\nwdocspar

\begin{description}
\item [{\texttt{UNLINK}}] (\ a a'\ -- a\ ) Breaks the link of the given
entry, resetting it to the link pointed to by the element on top of
the stack.
\end{description}
\nwenddocs{}\nwbegincode{77}\sublabel{NWtG4Lr-1nyRGB-1}\nwmargintag{{\nwtagstyle{}\subpageref{NWtG4Lr-1nyRGB-1}}}\moddef{unlink~{\nwtagstyle{}\subpageref{NWtG4Lr-1nyRGB-1}}}\endmoddef\nwstartdeflinemarkup\nwusesondefline{\\{NWtG4Lr-aGu5T-1}}\nwenddeflinemarkup
: UNLINK ( a a' -- a )   @LINK OVER !LINK ;
\nwused{\\{NWtG4Lr-aGu5T-1}}\nwendcode{}\nwbegindocs{78}\nwdocspar

\begin{description}
\item [{\texttt{CALLS}}] (\ a\ --\ ) Runs down a linked list, executing
the high level code that follows each entry in the list.
\end{description}
\nwenddocs{}\nwbegincode{79}\sublabel{NWtG4Lr-16qzin-1}\nwmargintag{{\nwtagstyle{}\subpageref{NWtG4Lr-16qzin-1}}}\moddef{calls~{\nwtagstyle{}\subpageref{NWtG4Lr-16qzin-1}}}\endmoddef\nwstartdeflinemarkup\nwusesondefline{\\{NWtG4Lr-aGu5T-1}}\nwenddeflinemarkup
: CALLS ( a -- )
    BEGIN
        @LINK ?DUP WHILE
        DUP >R  1 CELLS + @LINK EXECUTE  R>
    REPEAT ;
\nwused{\\{NWtG4Lr-aGu5T-1}}\nwendcode{}\nwbegindocs{80}\nwdocspar

\begin{description}
\item [{\texttt{append}}] (\ a1 n a2\ --\ ) Add a string to a counted
string.
\end{description}
\nwenddocs{}\nwbegincode{81}\sublabel{NWtG4Lr-2ZtF2B-1}\nwmargintag{{\nwtagstyle{}\subpageref{NWtG4Lr-2ZtF2B-1}}}\moddef{append~{\nwtagstyle{}\subpageref{NWtG4Lr-2ZtF2B-1}}}\endmoddef\nwstartdeflinemarkup\nwusesondefline{\\{NWtG4Lr-2V5MiK-1}}\nwenddeflinemarkup
: append ( from len to -- )   2DUP >R >R  COUNT + SWAP MOVE
    R> R@ C@ + R> C! ;
\nwused{\\{NWtG4Lr-2V5MiK-1}}\nwendcode{}\nwbegindocs{82}\nwdocspar

\begin{description}
\item [{\texttt{place}}] (\ a1 n a2\ --\ ) Place a counted string.
\end{description}
\nwenddocs{}\nwbegincode{83}\sublabel{NWtG4Lr-24CRHn-1}\nwmargintag{{\nwtagstyle{}\subpageref{NWtG4Lr-24CRHn-1}}}\moddef{place~{\nwtagstyle{}\subpageref{NWtG4Lr-24CRHn-1}}}\endmoddef\nwstartdeflinemarkup\nwusesondefline{\\{NWtG4Lr-2V5MiK-1}}\nwenddeflinemarkup
: place ( from len to -- )   0 OVER C! SWAP 255 MIN SWAP APPEND ;
\nwused{\\{NWtG4Lr-2V5MiK-1}}\nwendcode{}\nwbegindocs{84}\nwdocspar

\begin{description}
\item [{\texttt{,string}}] (\ a n\ --\ ) Compile a counted string.
\end{description}
\nwenddocs{}\nwbegincode{85}\sublabel{NWtG4Lr-4Hk4uw-1}\nwmargintag{{\nwtagstyle{}\subpageref{NWtG4Lr-4Hk4uw-1}}}\moddef{,string~{\nwtagstyle{}\subpageref{NWtG4Lr-4Hk4uw-1}}}\endmoddef\nwstartdeflinemarkup\nwusesondefline{\\{NWtG4Lr-2V5MiK-1}}\nwenddeflinemarkup
: ,string ( str len -- a )   HERE DUP >R OVER 1+ ALLOT place R> ;
\nwused{\\{NWtG4Lr-2V5MiK-1}}\nwendcode{}\nwbegindocs{86}\nwdocspar

\begin{description}
\item [{\$c,}] (\ a n\ --\ ) Compile a string.
\end{description}
\nwenddocs{}\nwbegincode{87}\sublabel{NWtG4Lr-b9pLA-1}\nwmargintag{{\nwtagstyle{}\subpageref{NWtG4Lr-b9pLA-1}}}\moddef{string-c,~{\nwtagstyle{}\subpageref{NWtG4Lr-b9pLA-1}}}\endmoddef\nwstartdeflinemarkup\nwusesondefline{\\{NWtG4Lr-2V5MiK-1}}\nwenddeflinemarkup
: $c, ( str len -- )   OVER + SWAP ?DO  I C@ C,  LOOP ;
\nwused{\\{NWtG4Lr-2V5MiK-1}}\nwendcode{}\nwbegindocs{88}\nwdocspar

\begin{description}
\item [{spin}] (\ --\ ) Show activity.
\end{description}
\nwenddocs{}\nwbegincode{89}\sublabel{NWtG4Lr-2Cbxl2-1}\nwmargintag{{\nwtagstyle{}\subpageref{NWtG4Lr-2Cbxl2-1}}}\moddef{spin~{\nwtagstyle{}\subpageref{NWtG4Lr-2Cbxl2-1}}}\endmoddef\nwstartdeflinemarkup\nwusesondefline{\\{NWtG4Lr-2V5MiK-1}}\nwenddeflinemarkup
[defined] spin 0= [IF] : spin ; IMMEDIATE [THEN]
\LA{}-spin~{\nwtagstyle{}\subpageref{NWtG4Lr-2LU99s-1}}\RA{}
\nwused{\\{NWtG4Lr-2V5MiK-1}}\nwendcode{}\nwbegindocs{90}\nwdocspar

\begin{description}
\item [{-spin}] (\ --\ ) Stop showing activity.
\end{description}
\nwenddocs{}\nwbegincode{91}\sublabel{NWtG4Lr-2LU99s-1}\nwmargintag{{\nwtagstyle{}\subpageref{NWtG4Lr-2LU99s-1}}}\moddef{-spin~{\nwtagstyle{}\subpageref{NWtG4Lr-2LU99s-1}}}\endmoddef\nwstartdeflinemarkup\nwusesondefline{\\{NWtG4Lr-2Cbxl2-1}}\nwenddeflinemarkup
[defined] -spin 0= [IF] : -spin ; IMMEDIATE [THEN]
\nwused{\\{NWtG4Lr-2Cbxl2-1}}\nwendcode{}\nwbegindocs{92}\nwdocspar

\begin{description}
\item [{-TEXT}] takes the address of two strings and a length and returns
+1 if the 1st string has a higher ASCII value, -1 if it has a lower
value, and 0 (false) if they are equal. The string length must be
even. If the length is odd, it will be rounded down to the next even
value.
\end{description}
\nwenddocs{}\nwbegincode{93}\sublabel{NWtG4Lr-4HDS1q-1}\nwmargintag{{\nwtagstyle{}\subpageref{NWtG4Lr-4HDS1q-1}}}\moddef{-text~{\nwtagstyle{}\subpageref{NWtG4Lr-4HDS1q-1}}}\endmoddef\nwstartdeflinemarkup\nwusesondefline{\\{NWtG4Lr-2V5MiK-1}}\nwenddeflinemarkup
: -TEXT ( a1 n a2 -- flag )
    REVERSE @ IF
        0  ROT 2/ 0 DO
            ROT DUP 2 nC@ >R 2 +
            ROT DUP 2 nC@ >R 2 +
            ROT R> R> - ?DUP IF
                NIP 0< IF
                    1  ELSE  -1
        THEN  LEAVE  THEN  LOOP  NIP NIP
    ELSE  OVER COMPARE  THEN ;
\nwused{\\{NWtG4Lr-2V5MiK-1}}\nwendcode{}\nwbegindocs{94}\nwdocspar

In addition, Open Firmware does not presently contain the following
ANS Forth words:

\texttt{\textbf{SAVE-INPUT RESTORE-INPUT M+ ALLOCATE FREE RESIZE}}

Additionally, the \texttt{\textbf{BYTE-LOAD}} method is used to conditionally
compile support for Open Firmware specific features. For example,
the 1275 tokenizer does not directly support defining words, so extensions
to the tokenizer must be defined every time one of these words is
defined. The disk based file support words are also not directly supported,
so they are not loaded into the Open Firmware package.

The Data Base Support Package is compiled with the rest of the Open
Firmware kernel, so there may be other dependencies, beyond the 1275
specified words, which have not been identified at this point. The
package is optionally byte-loaded into the \texttt{\textbf{/packages}}
node and it optionally byte-loads the applications which use it. This
creates a compile time binding (i.e. early binding) between the kernel,
this package, and the applications, which is required at the present
time. Future work will be needed to remove this dependency.

Systems that use this package should compile their applications in
\texttt{\textbf{THE-FILES.of}}, following the compile of the \texttt{\textbf{database.of}}
file

The following modules rely on the Database operators:

\texttt{\textbf{-1 -1 tokenize-list OF/OFSources/Common/Database.of}}~\\
\texttt{\textbf{ Encode-Driver BuildResults/Build/Obj/Database.lzss}}~\\
\texttt{\textbf{ -1 -1 tokenize-list OF/OFSources/IO/smu/sdb-parser.of}}~\\
\texttt{\textbf{ Encode-Driver BuildResults/Build/Obj/sdb-parser.lzss}}

Add their applications to \texttt{\textbf{The-Drivers.fo}}:

\texttt{\textbf{LoadDriver Database}}~\\
\texttt{\textbf{LoadDriver sdb-parser}}

\texttt{\textbf{AddDriver Database}}~\\
\texttt{\textbf{AddDriver sdb-parser}}

And the byte load of their applications to \texttt{\textbf{Database.of}}:

\texttt{\textbf{\textquotedbl{} sdb-parser\textquotedbl{} {[}'{]}
byte-load-driver CATCH IF 2DROP THEN}}

Then, their application interfaces can be called as needed:

\texttt{\textbf{0 0 \textquotedbl{} database\textquotedbl{} \$open-package
?dup}}~\\
\texttt{\textbf{if >r {[}char{]} B \textquotedbl{} distribute-sdb-properties\textquotedbl{}
r@ \$call-method}}~\\
\texttt{\textbf{ r> close-package}}~\\
\texttt{\textbf{then }}


\subsection{Creating a Simple File \label{sub:Creating-a-Simple-File}}

This section introduces the procedures for creating a simple file
by way of an example, and provides a contextual framework for the
detailed sections that follow.

In this section we are assuming that we already know how to structure
our data; we are concerned here only with the mechanical aspects of
file creation and field layout. For a more general discussion of data
base design, see \prettyref{sub:Data-Base-Design}.

Our simple example will be a file of names and addresses. To avoid
extra detail, we will only use alphanumeric fields. No attempt will
be made at keeping a sorted file (ordered indexes are discussed in
\prettyref{sub:Ordered-Index-Files}).

\nwenddocs{}\nwbegincode{95}\sublabel{NWtG4Lr-3Y3Oqi-1}\nwmargintag{{\nwtagstyle{}\subpageref{NWtG4Lr-3Y3Oqi-1}}}\moddef{people~{\nwtagstyle{}\subpageref{NWtG4Lr-3Y3Oqi-1}}}\endmoddef\nwstartdeflinemarkup\nwusesondefline{\\{NWtG4Lr-2V5MiK-1}}\nwenddeflinemarkup
\LA{}people-fields~{\nwtagstyle{}\subpageref{NWtG4Lr-1tBGdQ-1}}\RA{}
\LA{}people-file~{\nwtagstyle{}\subpageref{NWtG4Lr-2XK40H-1}}\RA{}
\LA{}people-initialize~{\nwtagstyle{}\subpageref{NWtG4Lr-4XCuIG-1}}\RA{}
\LA{}enter-person~{\nwtagstyle{}\subpageref{NWtG4Lr-30BTWM-1}}\RA{}
\LA{}display-person~{\nwtagstyle{}\subpageref{NWtG4Lr-2HPrsv-1}}\RA{}
\LA{}display-everyone~{\nwtagstyle{}\subpageref{NWtG4Lr-2lRxl2-1}}\RA{}
\LA{}all-people~{\nwtagstyle{}\subpageref{NWtG4Lr-36NJG5-1}}\RA{}
\nwused{\\{NWtG4Lr-2V5MiK-1}}\nwendcode{}\nwbegindocs{96}\nwdocspar

\textbf{Step 1}

Define the fields:

\nwenddocs{}\nwbegincode{97}\sublabel{NWtG4Lr-1tBGdQ-1}\nwmargintag{{\nwtagstyle{}\subpageref{NWtG4Lr-1tBGdQ-1}}}\moddef{people-fields~{\nwtagstyle{}\subpageref{NWtG4Lr-1tBGdQ-1}}}\endmoddef\nwstartdeflinemarkup\nwusesondefline{\\{NWtG4Lr-3Y3Oqi-1}}\nwenddeflinemarkup
0  20 BYTES NAME   20 BYTES STREET
   14 BYTES CITY    2 BYTES STATE
    6 BYTES ZIP    14 BYTES PHONE   DROP 
\nwused{\\{NWtG4Lr-3Y3Oqi-1}}\nwendcode{}\nwbegindocs{98}\nwdocspar

In the above lines we have defined six Forth words, which will reference
the individual fields in each record. The initial zero is the relative
position within the record. The defining word \texttt{\textbf{BYTES}}
creates an alphanumeric field of the specified width (the width must
be an even number on cell-aligned processors). The final \texttt{\textbf{DROP}}
is necessary to discard the final relative position within the record
(see \prettyref{sub:Record-Description}).

In addition to \texttt{\textbf{BYTES}}, several other defining words
are available for creating different types of fields (see \prettyref{sub:Field-Definitions}).

\textbf{Step 2}

Determine how many records and blocks the file will need. Two words
that are not generally resident are available in the ``file initialization
block.''%
\footnote{In polyFORTH this block is not generally resident, because it is used
only in the initial creation of the data base. It may usually be found
at \texttt{\textbf{FILES 5 +}}.%
} \texttt{\textbf{\#R}} computes the number of records of given size
that would fit in a given number of blocks; \texttt{\textbf{\#B}}
computes the reverse: the number of blocks needed to hold a given
number of records of given size (see \prettyref{sub:The-FILE-Definition}).

\textbf{Step 3}

Define the file:

\nwenddocs{}\nwbegincode{99}\sublabel{NWtG4Lr-2XK40H-1}\nwmargintag{{\nwtagstyle{}\subpageref{NWtG4Lr-2XK40H-1}}}\moddef{people-file~{\nwtagstyle{}\subpageref{NWtG4Lr-2XK40H-1}}}\endmoddef\nwstartdeflinemarkup\nwusesondefline{\\{NWtG4Lr-3Y3Oqi-1}}\nwenddeflinemarkup
FILE People.dbf

: /People ( -- ) \\ Runtime file setup
    People.dbf  S" People.dbf" >FILE
    TRUE REVERSE !  LITTLE-ENDIAN ;

76 500 400 BLOCK-DATA PEOPLE
\nwused{\\{NWtG4Lr-3Y3Oqi-1}}\nwendcode{}\nwbegindocs{100}\nwdocspar

This statement defines a file called \texttt{\textbf{PEOPLE}}, which
contains records that are each 74 bytes in length. The file will hold
a maximum of 500 records. The starting block will be 400 (see \prettyref{sub:The-FILE-Definition}).

Invoking the filename \texttt{\textbf{PEOPLE}} makes this file current.

\textbf{Step 4}

Initialize the file. Load the file initialization block (if it's not
already loaded) and execute the phrase:

\nwenddocs{}\nwbegincode{101}\sublabel{NWtG4Lr-4XCuIG-1}\nwmargintag{{\nwtagstyle{}\subpageref{NWtG4Lr-4XCuIG-1}}}\moddef{people-initialize~{\nwtagstyle{}\subpageref{NWtG4Lr-4XCuIG-1}}}\endmoddef\nwstartdeflinemarkup\nwusesondefline{\\{NWtG4Lr-3Y3Oqi-1}}\nwenddeflinemarkup
\\ PEOPLE INITIALIZE
\nwused{\\{NWtG4Lr-3Y3Oqi-1}}\nwendcode{}\nwbegindocs{102}\nwdocspar

to fill all blocks in the file with zeroes (see \prettyref{sub:File-Initialization-Utility}).

\textbf{Step 5}

Enter data.

Here is the definition of a word that will allow data entry for a
single record (person):

\nwenddocs{}\nwbegincode{103}\sublabel{NWtG4Lr-30BTWM-1}\nwmargintag{{\nwtagstyle{}\subpageref{NWtG4Lr-30BTWM-1}}}\moddef{enter-person~{\nwtagstyle{}\subpageref{NWtG4Lr-30BTWM-1}}}\endmoddef\nwstartdeflinemarkup\nwusesondefline{\\{NWtG4Lr-3Y3Oqi-1}}\nwenddeflinemarkup
: enter-person   PEOPLE  SLOT READ
    CR ." Name? "     NAME ASK
    CR ." Address? "  STREET ASK
    CR ." City? "     CITY ASK
    CR ." State? "    STATE ASK
    CR ." Zip? "      ZIP ASK
    CR ." Phone? "    PHONE ASK ; 
\nwused{\\{NWtG4Lr-3Y3Oqi-1}}\nwendcode{}\nwbegindocs{104}\nwdocspar

By invoking \texttt{\textbf{PEOPLE}}, we select the \texttt{\textbf{PEOPLE}}
file as the current file.

The word \texttt{\textbf{SLOT}} allocates a new record in the current
file, and leaves its number on the stack (see \prettyref{sub:Record-Allocation/Deallocation-Operators}).
The word \texttt{\textbf{READ}} sets the current record according
to the number on the stack (see \prettyref{sub:Record-Selection}).

Next, the definition prompts the user to enter the ``name'' field.
The word \texttt{\textbf{ASK}} is like \texttt{\textbf{EXPECT}}, except
that it places the expected text in the given field. The same process
is followed for the remaining five fields.

\textbf{Step 6}

Display the data.

We define the following word to display the current record:

\nwenddocs{}\nwbegincode{105}\sublabel{NWtG4Lr-2HPrsv-1}\nwmargintag{{\nwtagstyle{}\subpageref{NWtG4Lr-2HPrsv-1}}}\moddef{display-person~{\nwtagstyle{}\subpageref{NWtG4Lr-2HPrsv-1}}}\endmoddef\nwstartdeflinemarkup\nwusesondefline{\\{NWtG4Lr-3Y3Oqi-1}}\nwenddeflinemarkup
: display-person   CR NAME B?  CR STREET B?  CR
    CITY B?  STATE B?  ZIP B?  PHONE B? ; 
\nwused{\\{NWtG4Lr-3Y3Oqi-1}}\nwendcode{}\nwbegindocs{106}\nwdocspar

The word \texttt{\textbf{B?}} displays the contents of the given \texttt{\textbf{BYTES}}
field (see \prettyref{sub:Field-Reference-Operators}).

To display the contents of all records that have been entered, we
define:

\nwenddocs{}\nwbegincode{107}\sublabel{NWtG4Lr-2lRxl2-1}\nwmargintag{{\nwtagstyle{}\subpageref{NWtG4Lr-2lRxl2-1}}}\moddef{display-everyone~{\nwtagstyle{}\subpageref{NWtG4Lr-2lRxl2-1}}}\endmoddef\nwstartdeflinemarkup\nwusesondefline{\\{NWtG4Lr-3Y3Oqi-1}}\nwenddeflinemarkup
: display-everyone   PEOPLE RECORDS DO CR
    I READ  display-person  LOOP ; 
\nwused{\\{NWtG4Lr-3Y3Oqi-1}}\nwendcode{}\nwbegindocs{108}\nwdocspar

Invoking \texttt{\textbf{PEOPLE}} makes the \texttt{\textbf{PEOPLE}}
file current. The word \texttt{\textbf{RECORDS}} returns the appropriate
arguments for a \texttt{\textbf{DO\ ...\ LOOP}}, including all records
that have been allocated by \texttt{\textbf{SLOT}} in the current
file (see \prettyref{sub:Record-Allocation/Deallocation-Operators}).

Within the \texttt{\textbf{DO}} loop, \texttt{\textbf{READ}} makes
each record current in turn, and \texttt{\textbf{display-person}}
displays the information for that record.

\noindent \begin{flushleft}
Here is a sample of the output of \texttt{\textbf{display-everyone}}:
\par\end{flushleft}

\noindent \begin{flushleft}
\begin{lstlisting}
Andrews, Carl 
1432 Morriston Ave. 
Parkerville PA 17214 (717) 555-9853

Boehning, Greg 
POB 41256 
Santa Cruz CA 95061 (408) 666-7891

Chapel, Doug 
75 Fleetwood Dr. 
Rockville MD 20852 (301) 777-1259

Cook, Dottie 
154 Sweet Rd. 
Grand Prairie TX 75050 (214) 642-0011 
\end{lstlisting}

\par\end{flushleft}

\noindent \begin{flushleft}
To produce columnar output, we would use the ``Report Generator''
(\prettyref{sub:Report-Generator}).
\par\end{flushleft}

For deleting records, we would use the word \texttt{\textbf{SCRATCH}}
(see \prettyref{sub:Record-Allocation/Deallocation-Operators}). 


\subsection{File Definition and Access\label{sub:File-Definition-and-Access}}

A polyFORTH file is a contiguous region of Forth blocks. On native
versions of polyFORTH this means that the file will be physically
contiguous, and that you can arrange for files that are accessed together
to be physically near one another. This can significantly speed up
a data base application.

Versions of polyFORTH that are co-resident with a ``host'' operating
system (such as MS-DOS or RSX) are identical from the point of view
of the programmer, but since allocation of physical disk space is
performed by the host operating system you haven't the actual level
of control you do on the native versions.

This section discusses how files are defined and referenced on all
polyFORTH systems. 


\subsubsection{The FILE Definition \label{sub:The-FILE-Definition}}

The word \texttt{\textbf{FILE}} is used to group a collection of Data
Bases under a single name. Each Data Base within the group will use
the same access operators. Presently, there are 2 types of access
operators, disk based (\texttt{\textbf{>FILE}}) and memory based (\texttt{\textbf{>MEMORY}}).
The disk based operators can not be used within Open Firmware, since
the creation of files has not been fully developed yet. However, in
gforth, they are used extensively by passing the pathname string to
\texttt{\textbf{>FILE}}. The word \texttt{\textbf{FILE=}} simply parses
the input stream for the pathname string before passing it to \texttt{\textbf{>FILE}}.

\nwenddocs{}\nwbegincode{109}\sublabel{NWtG4Lr-4EB9lH-1}\nwmargintag{{\nwtagstyle{}\subpageref{NWtG4Lr-4EB9lH-1}}}\moddef{file~{\nwtagstyle{}\subpageref{NWtG4Lr-4EB9lH-1}}}\endmoddef\nwstartdeflinemarkup\nwusesondefline{\\{NWtG4Lr-4ffLeu-1}}\nwenddeflinemarkup
: (FILE) ( o n -- o+n ) \\ Usage: (FILE) <name>
    CREATE  OVER , +  DOES> ( -- a )   @  F# @ + ;

0 VALUE #FILE                       \\ holds offset of next FCB field.

#FILE 2 CELLS (FILE) FILE-HANDLE    \\ holds OS/Memory handle for this file, if open.
        CELL  (FILE) FILE-INIT      \\ holds initialization routines for this file.
        CELL  (FILE) 'NAME-FILE     \\ holds routine to name the file
        CELL  (FILE) 'BIND-FILE     \\ holds routine to bind the file to a path.
        CELL  (FILE) 'UNBIND-FILE   \\ holds routine to unbind the file.
        CELL  (FILE) 'READ-RECORD   \\ holds routine to read a record from the file.
        CELL  (FILE) 'WRITE-RECORD  \\ holds routine to write a record to the file.
        CELL  (FILE) 'DESTROY-FILE  \\ holds routine to delete the file.
        256   (FILE) FILE-NAME      \\ holds path and name of the file.
TO #FILE

: SET-FILE ( addr -- )   F# ! ;
: SET-DATA ( addr -- )   DB# ! ;

: (NAME-FILE) ( str len -- )   FILE-NAME DUP 256 ERASE place ;

: NAME-FILE ( str len -- )   'NAME-FILE @ ?DUP IF  EXECUTE
    ELSE  2DROP  THEN ;

: READ-RECORD ( d a n -- n' ior )   'READ-RECORD @ ?DUP IF  EXECUTE
    ELSE  NIP NIP NIP 0  THEN ;

: WRITE-RECORD ( d a n -- ior )   'WRITE-RECORD @ ?DUP IF  EXECUTE
    ELSE  2DROP 2DROP 0  THEN ;

: DESTROY-FILE ( -- ior )   'DESTROY-FILE @ ?DUP IF EXECUTE ELSE 0 THEN ;

: BIND-FILE ( -- )   FILE-HANDLE @ 0=
    IF  'BIND-FILE @ ?DUP
        IF  EXECUTE
        THEN  FILE-HANDLE !
    THEN ;

: -FILE ( -- )   FILE-HANDLE @ ?DUP
    IF  'UNBIND-FILE @ ?DUP
        IF  EXECUTE
        THEN  0 FILE-HANDLE !
    THEN ;

: FILE-ALLOT ( -- )
    HERE DUP SET-FILE  #FILE DUP ALLOT ERASE
    ['] (NAME-FILE) 'NAME-FILE ! ;

: FILE ( -- ) \\ Usage: FILE <name>
    SAVE-INPUT  CREATE  RESTORE-INPUT THROW
    FILE-ALLOT  BL WORD COUNT NAME-FILE
    DOES> ( -- )   SET-FILE ;

: FILES-NAME ( str len -- )
    [defined] ?FileName [IF]  0 TO ?FileName  ,ObjName
    [THEN]  (NAME-FILE) ;

: FILES-BIND ( -- handle )
    FILE-NAME COUNT 2DUP R/W OPEN-FILE
    IF  DROP R/W CREATE-FILE THROW
    ELSE  -ROT 2DROP
    THEN ;

: FILES-UNBIND ( handle -- )   CLOSE-FILE THROW ;

: FILES-READ ( d a n -- n' ior )
    FILE-HANDLE @ 0= DUP >R IF  BIND-FILE  THEN
    2SWAP FILE-HANDLE @ REPOSITION-FILE ?DUP IF  ROT DROP
    ELSE  FILE-HANDLE @ READ-FILE
    THEN  R> IF  -FILE  THEN ;

: FILES-WRITE ( d a n -- ior )
    FILE-HANDLE @ 0= DUP >R IF  BIND-FILE  THEN
    2SWAP FILE-HANDLE @ REPOSITION-FILE ?DUP IF  -ROT 2DROP
    ELSE  FILE-HANDLE @ WRITE-FILE
    THEN  R> IF  -FILE  THEN ;

: FILES-DESTROY ( -- ior )   FILE-NAME COUNT DELETE-FILE ;

: >FILE ( str len -- )   -FILE
    ['] FILES-NAME 'NAME-FILE !
    ['] FILES-BIND 'BIND-FILE !
    ['] FILES-UNBIND 'UNBIND-FILE !
    ['] FILES-READ 'READ-RECORD !
    ['] FILES-WRITE 'WRITE-RECORD !
    ['] FILES-DESTROY 'DESTROY-FILE !
    NAME-FILE  BIND-FILE
    FILE-INIT CALLS
    KEEP-CLOSED @ IF
        -FILE
    THEN  ;

: FILE= ( -- )   BL WORD COUNT  DUP 1+ ALLOCATE THROW
    DUP >R place R@ COUNT >FILE  R> FREE THROW ;

: <FILE ( xt -- )   FILE-INIT <LINK ,LINK ;
\nwused{\\{NWtG4Lr-4ffLeu-1}}\nwendcode{}\nwbegindocs{110}\nwdocspar

In Open Firmware, the Data Base groups are typically created at compile
time and compiled into Open Firmware with \texttt{\textbf{encode-file}}.
Since this puts the data into the dictionary, the word \texttt{\textbf{HERE>MEMORY}}
is used to put the data into the buddy memory manager which can be
used by the \texttt{\textbf{>MEMORY}} access operators.

\nwenddocs{}\nwbegincode{111}\sublabel{NWtG4Lr-4K7kUG-1}\nwmargintag{{\nwtagstyle{}\subpageref{NWtG4Lr-4K7kUG-1}}}\moddef{memory~{\nwtagstyle{}\subpageref{NWtG4Lr-4K7kUG-1}}}\endmoddef\nwstartdeflinemarkup\nwusesondefline{\\{NWtG4Lr-2V5MiK-1}}\nwenddeflinemarkup
: MEMORY-BIND ( -- handle )   2 B/R @ #B 1024 * \\ 2 records minimum
    DUP FILE-HANDLE CELL+ !  DUP ALLOCATE THROW
    DUP ROT ERASE ;

: MEMORY-UNBIND ( handle -- )   FREE THROW ;

: MEMORY-READ ( d a n -- n' ior )   BIND-FILE
    2SWAP IF  DROP NIP -24 EXIT  THEN      \\ invalid numeric argument, file too large
    FILE-HANDLE CELL+ @ 2DUP >             \\ limit record position to file size
    IF  SWAP  THEN  DROP SWAP OVER +
    FILE-HANDLE CELL+ @ 2DUP >             \\ limit record size to file size
    IF  SWAP  THEN  DROP OVER -
    DUP >R  SWAP FILE-HANDLE @ +
    ROT ROT MOVE  R> 0 ;

: MEMORY-WRITE ( d a n -- ior )   BIND-FILE
    2SWAP IF  DROP 2DROP -24 EXIT  THEN    \\ invalid numeric argument, file too large
    SWAP OVER +  DUP FILE-HANDLE CELL+ @ > \\ record position + size > file size
    IF  FILE-HANDLE @ OVER RESIZE ?DUP
        IF  >R 2DROP 2DROP R> EXIT         \\ can't resize the memory
        THEN  FILE-HANDLE !
        DUP FILE-HANDLE CELL+ !
    THEN  OVER - SWAP FILE-HANDLE @ +
    SWAP MOVE  0 ;

: >MEMORY ( addr len -- )   -FILE
    ['] MEMORY-BIND 'BIND-FILE !
    ['] MEMORY-UNBIND 'UNBIND-FILE !
    ['] MEMORY-READ 'READ-RECORD !
    ['] MEMORY-WRITE 'WRITE-RECORD !
    SWAP FILE-HANDLE 2!  BIND-FILE
    FILE-INIT CALLS ;
\nwused{\\{NWtG4Lr-2V5MiK-1}}\nwendcode{}\nwbegindocs{112}\nwdocspar

The words \texttt{\textbf{BLOCK-DATA}} or \texttt{\textbf{STRUCTURE}}
are used to define files, given the attributes of the file. The format
is:

\texttt{\textbf{length limit origin BLOCK-DATA name}}

where:
\begin{description}
\item [{Word}] \textbf{Description}
\item [{\texttt{length}}] is the length of each record in bytes (maximum
1024 using \texttt{\textbf{BLOCK-DATA}});
\item [{\texttt{limit}}] is the maximum number of records (on 16-bit processors
the limit is 32767 records per file);
\item [{\texttt{origin}}] is the first block number, and
\item [{\texttt{name}}] is the user-assigned name of the file.
\end{description}
The defining words \texttt{\textbf{BLOCK-DATA}} and \texttt{\textbf{STRUCTURE}}
create a new name (dictionary entry) that, when invoked, will make
this file current. The dictionary entry contains the \textbf{F}ile
\textbf{D}efinition \textbf{A}rea (FDA) for the file being defined.
\texttt{\textbf{BLOCK-DATA}} will make sure that records do not overlap
across 1K boundaries, while \texttt{\textbf{STRUCTURE}} does not impose
this additional overhead.

\nwenddocs{}\nwbegincode{113}\sublabel{NWtG4Lr-4IiShX-1}\nwmargintag{{\nwtagstyle{}\subpageref{NWtG4Lr-4IiShX-1}}}\moddef{block~{\nwtagstyle{}\subpageref{NWtG4Lr-4IiShX-1}}}\endmoddef\nwstartdeflinemarkup\nwusesondefline{\\{NWtG4Lr-4ffLeu-1}}\nwenddeflinemarkup
: (DATA) ( o n -- o+n ) \\ Usage: (DATA) <name>
    CREATE  OVER , +  DOES> ( -- a )   @  DB# @ + ;

0 VALUE #DATA               \\ holds offset of next DCB field.

#DATA CELL (DATA) ORG       \\ holds offset to first record in the file.
      CELL (DATA) LIM       \\ holds number of records in file.
      CELL (DATA) B/B       \\ holds bytes used per block.
      CELL (DATA) B/R       \\ holds bytes per record.
      CELL (DATA) DATA-BFR  \\ holds address of record buffer.
      CELL (DATA) 'RECORD   \\ holds the vector for record access.
      CELL (DATA) 'TOUCH    \\ holds the vector for field update.
      CELL (DATA) #REC      \\ holds the record number last accessed in this file.
                            \\  This is used with BOUND-FIELDS.
      CELL (DATA) #INDEX    \\ holds index of COPIES field.
       256 (DATA) DATA-NAME \\ holds name of the database.
TO #DATA

: NAME-DATA ( str len -- )   DATA-NAME DUP 256 ERASE place ;

: BLOCK-POSITION ( n -- d )   DUP #REC !
    B/R @  B/B @ */MOD  1024 UM*  ROT M+
    ORG @ M+ ;

: (RECORD) ( d -- a )
    DATA-BFR @ B/R @ READ-RECORD IF
        DROP DATA-BFR @ B/R @ ERASE
    ELSE  B/R @ OVER - ?DUP IF
            DATA-BFR @ ROT + SWAP ERASE
        ELSE  DROP  THEN
    THEN  DATA-BFR @ ;

: (TOUCH) ( d -- )   DATA-BFR @ B/R @ WRITE-RECORD THROW ;

: BLOCK-RECORD ( n -- a )   BLOCK-POSITION (RECORD) ;
: BLOCK-TOUCH ( -- )   #REC @ BLOCK-POSITION (TOUCH) ;

: (BLOCK-DATA) ( b r o -- )
    HERE DUP SET-DATA  #DATA DUP ALLOT ERASE
    ORG !  LIM !  0 #REC !  0 #INDEX !
    1024 OVER / OVER *  B/B !  B/R !
    HERE DATA-BFR !  B/R @ ALLOT
    ['] BLOCK-RECORD 'RECORD !
    ['] BLOCK-TOUCH 'TOUCH ! ;

: BLOCK-DATA ( b r o _ -- ) \\ Usage: BLOCK-DATA <name>
    SAVE-INPUT  CREATE  RESTORE-INPUT THROW
    (BLOCK-DATA)  BL WORD COUNT NAME-DATA
    DOES> ( -- )   SET-DATA ;
\nwused{\\{NWtG4Lr-4ffLeu-1}}\nwendcode{}\nwbegindocs{114}\nwdocspar
\nwenddocs{}\nwbegincode{115}\sublabel{NWtG4Lr-2dJGUc-1}\nwmargintag{{\nwtagstyle{}\subpageref{NWtG4Lr-2dJGUc-1}}}\moddef{structure~{\nwtagstyle{}\subpageref{NWtG4Lr-2dJGUc-1}}}\endmoddef\nwstartdeflinemarkup\nwusesondefline{\\{NWtG4Lr-2V5MiK-1}}\nwenddeflinemarkup
: POSITION-FILE ( r -- d )   DUP #REC !  B/R @ UM*  ORG @ M+ ;

: STRUCTURE-RECORD ( r -- a )   POSITION-FILE (RECORD) ;
: STRUCTURE-TOUCH ( -- )   #REC @ POSITION-FILE (TOUCH) ;

: (STRUCTURE) ( b r o _ -- )
    HERE DUP SET-DATA  #DATA DUP ALLOT ERASE
    ORG !  LIM !  0 #REC !  0 #INDEX !
    DUP B/R !  HERE DATA-BFR !  ALLOT
    ['] STRUCTURE-RECORD 'RECORD !
    ['] STRUCTURE-TOUCH 'TOUCH ! ;

: STRUCTURE ( b r o _ -- ) \\ Usage: STRUCTURE <name>
    SAVE-INPUT  CREATE  RESTORE-INPUT THROW
    (STRUCTURE)  BL WORD COUNT NAME-DATA
    DOES> ( -- )   SET-DATA ;
\nwused{\\{NWtG4Lr-2V5MiK-1}}\nwendcode{}\nwbegindocs{116}\nwdocspar

We recommend that you define all your files in a single source area,
making it easy to see which ranges of blocks have been allocated for
other files. If a disk will contain source or other data along with
files, it's a good idea to indicate these other uses in comments on
the same block.

Here is an example of good file definition layout in a block:

\texttt{\textbf{( Bytes Records Origin Name)}}~\\
\texttt{\textbf{ 26 801 500 STRUCTURE (GLOSSARY)}}~\\
\texttt{\textbf{ 340 800 522 STRUCTURE GLOSSARY}}~\\
\texttt{\textbf{ 4 10 795 STRUCTURE HITS}}~\\
\texttt{\textbf{ 24 42 799 STRUCTURE SECURITY}}~\\
\texttt{\textbf{ 38 2600 800 STRUCTURE TESTS}}

Note that the number of blocks may be computed from the number of
bytes/ record and number of records. Generally you will choose an
appropriate maximum number of records, based on a reasonable estimate
of the needs of the application and allowing for expansion. You will
also have worked out the approximate size of each record based on
the width and type of fields needed. Then derive the number of blocks
from the number of records and size of each record. The word \texttt{\textbf{\#B}}
in the file initialization block is a helpful tool for computing the
number of blocks. After loading this block, type:

\texttt{\textbf{\#records \#bytes/record \#B .}}

For instance, if your application requires 2000 records, and each
record is 42 bytes wide, type:%
\footnote{Responses shown in light type.%
}

\texttt{\textbf{2000 42 \#B . }}\texttt{84 ok}

Alternatively you can compute the number of records based on the number
of blocks. The word \texttt{\textbf{\#R}} in the file initialization
block does the arithmetic. Type:

\texttt{\textbf{\#blocks \#bytes/record \#R .}}

For example:

\texttt{\textbf{84 42 \#R . }}\texttt{2016 ok}

This shows that you can actually fit an extra sixteen records in the
same number of blocks.

By using these tools, you can iterate on various sizes until you get
the optimal combination. Sometimes you can increase the size of a
record without increasing overall file size. For instance, if your
record width is 94 bytes, it takes 200 blocks to store the same number
of records; however 200 blocks will store 2000 records even when each
record is 102 bytes wide:

\texttt{\textbf{2000 94 \#B . }}\texttt{200}\textbf{}\\
\texttt{\textbf{2000 102 \#B}}\texttt{ . 200}

It's a good idea to leave extra space in records, in case you need
to add fields later. Beware, however, of grossly over sizing either
your record width or file length, as both of these will increase head
motion. Strive for generous but reasonable estimates. 

\nwenddocs{}\nwbegincode{117}\sublabel{NWtG4Lr-2aZT6J-1}\nwmargintag{{\nwtagstyle{}\subpageref{NWtG4Lr-2aZT6J-1}}}\moddef{origin~{\nwtagstyle{}\subpageref{NWtG4Lr-2aZT6J-1}}}\endmoddef\nwstartdeflinemarkup\nwusesondefline{\\{NWtG4Lr-4ffLeu-1}}\nwenddeflinemarkup
: #B ( nr b/r -- nb )   1024 SWAP /  DUP 1- ROT +  SWAP / ;

: #R ( nb b/r -- nr )   1024 SWAP / * ;

: +ORIGIN ( -- o )   LIM @ B/R @ B/B @ IF #B 1024 THEN * ORG @ + ;
\nwused{\\{NWtG4Lr-4ffLeu-1}}\nwendcode{}\nwbegindocs{118}\nwdocspar


\subsubsection{File Definition Area and Access \label{sub:File-Definition-Area}}

The word \texttt{\textbf{STRUCTURE}} establishes a File Definition
Area (FDA) for each file in the system. The user variable \texttt{\textbf{F\#}}
always points to the current FDA. Execution of the filename sets \texttt{\textbf{F\#}}
to address the associated FDA.

Each file's FDA contains four values to specify the file. Each of
these values may be accessed by the following names, each of which
returns the address of the associated value in the current FDA.

Name Description
\begin{description}
\item [{\texttt{ORG}}] Starting disk-block number of the first disk block
allo\-cated to the file (Forth logical block number as a double-precision
number).
\item [{\texttt{LIM}}] Number of records, of declared record length, that
the file can contain.
\item [{\texttt{B/B}}] Number of bytes used per block.
\item [{\texttt{B/R}}] Number of bytes per record.
\end{description}
While these words are used by the Data Base Support package, they
are rarely referenced directly in applications. \texttt{\textbf{ORG}}
and \texttt{\textbf{LIM}} can be useful in debugging, however. For
instance, the phrase:

\texttt{\textbf{ORG 2@ D.}}

indicates which file is current; in case of an abort, you can tell
which file you were in at the time. 


\subsubsection{File Initialization Utility \label{sub:File-Initialization-Utility}}

A file that has just been created must be initialized before it can
be used. A special utility is available for this purpose.

To initialize a file, type:

\texttt{\textbf{filename INITIALIZE}}

The word \texttt{\textbf{INITIALIZE}} performs the following functions:
\begin{enumerate}
\item Writes binary zeros throughout the entire file (including \texttt{\textbf{AVAILABLE}}).
\item Writes -1 in the entire data area of Record 1 of the file. This serves
as a ``stopper'' for the binary search in an index file. In other
kinds of files this has no effect.
\end{enumerate}
\nwenddocs{}\nwbegincode{119}\sublabel{NWtG4Lr-p3Fax-1}\nwmargintag{{\nwtagstyle{}\subpageref{NWtG4Lr-p3Fax-1}}}\moddef{initialize~{\nwtagstyle{}\subpageref{NWtG4Lr-p3Fax-1}}}\endmoddef\nwstartdeflinemarkup\nwusesondefline{\\{NWtG4Lr-4ffLeu-1}}\nwenddeflinemarkup
: STOPPER ( -- )   1 RECORD CELL+  B/R @ CELL -  -1 FILL  TOUCH ;

: INITIALIZE ( -- )
\\   CR ." Initializing "
    FILE-HANDLE @ 0= DUP >R IF  BIND-FILE  THEN
\\   DATA-NAME COUNT TYPE ." ..."
    LIM @ 0 DO
        I RECORD  B/R @  ERASE  TOUCH  spin
    LOOP  STOPPER  -spin
    R> IF  -FILE  THEN ;
\nwused{\\{NWtG4Lr-4ffLeu-1}}\nwendcode{}\nwbegindocs{120}\nwdocspar

Two other words also defined---\texttt{\textbf{\#R}} and \texttt{\textbf{\#B}}---are
useful when designing file layouts. 


\subsubsection{Shared Files \label{sub:Shared-Files}}

In polyFORTH files may be either shared or unshared. Shared files
are those that are defined in the common dictionary available to all
users (loaded by the electives load block). If a file is defined in
an overlay, it will be available only to the task or tasks in whose
partition it is defined.

As we discussed in \prettyref{sub:How-Data-is-Stored}, all users
may freely access the file without having to worry about simultaneous
access problems, as long as standard polyFORTH accessing methods are
used. This is because \texttt{\textbf{BLOCK}} ensures that there will
be only one copy of a record at a time, and each task does not have
its own private copy.

Certain situations require an extra measure of control. For example,
one terminal might delete a record that is needed for processing at
another terminal at a later point. In such as case, you may use a
``status'' byte in the record to control access.

\textbf{REFERENCES}

Installing the Data Base Support Package, \prettyref{sub:Installing-The-Package} 


\subsection{Record Management \label{sub:Record-Management}}

The process of record management includes selecting records, finding
the next free record when a new record is needed, and marking deleted
records as available for future use.

Not all applications require special record allocation techniques.
For instance, if a file contains 100 records and each record contains
information on a permanent piece of equipment, which is identified
by a two-digit num\-ber, there is no need to allocate or deallocate
records. You may just use the equipment number as the record number.
This is called ``direct access.''

In an application in which the number of active records changes dynamic\-ally,
it may be appropriate to use the record allocation techniques described
here. 


\subsubsection{Record Selection \label{sub:Record-Selection}}

Field reference operators (\prettyref{sub:Field-Reference-Operators})
access fields in the current record. The word \texttt{\textbf{READ}}
makes a record current.
\begin{description}
\item [{\texttt{READ}}] (\ n\ --\ ) Makes record n current, having verified
that \textit{n} is a valid record within the current file.
\end{description}
The name \texttt{\textbf{READ}} is slightly misleading, in that it
doesn't perform an actual disk operation, but merely sets a pointer
to the current record. \texttt{\textbf{READ}} checks that n is not
less than zero and not greater than the value of \texttt{\textbf{LIM}}.
If \textit{n} fails this range test, \texttt{\textbf{READ}} aborts.

\texttt{\textbf{READ}} stores the number of the current record in
the user variable \texttt{\textbf{R\#}}.

\nwenddocs{}\nwbegincode{121}\sublabel{NWtG4Lr-WHmjl-1}\nwmargintag{{\nwtagstyle{}\subpageref{NWtG4Lr-WHmjl-1}}}\moddef{read~{\nwtagstyle{}\subpageref{NWtG4Lr-WHmjl-1}}}\endmoddef\nwstartdeflinemarkup\nwusesondefline{\\{NWtG4Lr-4ffLeu-1}}\nwenddeflinemarkup
: SAFE ( n -- n )   DUP LIM @ U< 0= ABORT" Outside file " ;

: READ ( n -- )   SAFE DUP R# !  #REC ! ;
\nwused{\\{NWtG4Lr-4ffLeu-1}}\nwendcode{}\nwbegindocs{122}\nwdocspar

\textbf{REFERENCES}

Current Files and Records, \prettyref{sub:Current-Files-and-Records} 


\subsubsection{Available Records \label{sub:Available-Records}}

To distinguish allocated records from available records, the Data
Base Support package uses the convention that if the first four bytes
in a record contain binary zeroes, the record is available for use.
When a file is ini\-tialized, all records are filled with zeroes.

Thereafter, active records may keep any non-zero data in the first
four bytes; when a record is released, zero is stored in this area.

\textbf{}
\begin{figure}
\textbf{\caption{\textbf{\label{fig:5}\protect\includegraphics{4_home_druffer_Dropbox_ColorForth_File_Docs_Records.eps}}}
}
\end{figure}


The search for an available record performed by \texttt{\textbf{SLOT}}
``wraps around'' if necessary at the end of the file.

Record 0 of each file contains, in its first four bytes, the record
number of the most recently allocated record in that file. The word
\texttt{\textbf{AVAILABLE}} returns the address of this pointer. When
the file is initialized, \texttt{\textbf{AVAILABLE}} is zero.

To allocate a new record, the system begins with the record immediately
following the ``available'' record and searches forward for the
first free record.

If the search should reach the end of the file without finding a free
record , it ``wraps'' around to the beginning again, so that deleted
records will be used. For instance, in Fig. \prettyref{fig:5}, \texttt{\textbf{AVAILABLE}}
points to Record 196; however, there are no more free records between
there and the end of the file. But Record 5 is free. By ``wrapping
around'' to the beginning of the file, the search finds the available
record. 

\nwenddocs{}\nwbegincode{123}\sublabel{NWtG4Lr-2zefAp-1}\nwmargintag{{\nwtagstyle{}\subpageref{NWtG4Lr-2zefAp-1}}}\moddef{available~{\nwtagstyle{}\subpageref{NWtG4Lr-2zefAp-1}}}\endmoddef\nwstartdeflinemarkup\nwusesondefline{\\{NWtG4Lr-4ffLeu-1}}\nwenddeflinemarkup
: AVAILABLE ( -- a )   0 RECORD ;
\nwused{\\{NWtG4Lr-4ffLeu-1}}\nwendcode{}\nwbegindocs{124}\nwdocspar


\subsubsection{Record Allocation/Deallocation Operators \label{sub:Record-Allocation/Deallocation-Operators}}

Only two words are required for allocating and deallocating records:
\begin{description}
\item [{Word}] \textbf{Stack Description}
\item [{\texttt{SLOT}}] (\ --\ n) Allocates a new record in the current
file and returns the number of the allocated record.
\item [{\texttt{SCRATCH}}] (\ n) Deallocates record \textit{n} from the
current file, making it available.
\end{description}
\nwenddocs{}\nwbegincode{125}\sublabel{NWtG4Lr-M1coC-1}\nwmargintag{{\nwtagstyle{}\subpageref{NWtG4Lr-M1coC-1}}}\moddef{slot~{\nwtagstyle{}\subpageref{NWtG4Lr-M1coC-1}}}\endmoddef\nwstartdeflinemarkup\nwusesondefline{\\{NWtG4Lr-4ffLeu-1}}\nwenddeflinemarkup
: SLOT ( -- r )
    AVAILABLE 4 nC@ DUP  BEGIN
        1+  LIM @ MOD  2DUP = ABORT" File full"
        DUP RECORD  DUP @ WHILE  DROP
    REPEAT  DUP  B/R @ ERASE  -1 SWAP !  TOUCH
    DUP AVAILABLE 4 nC!  TOUCH  NIP ;

: SCRATCH ( r -- )   SAFE  RECORD  0 SWAP !  TOUCH ;
\nwused{\\{NWtG4Lr-4ffLeu-1}}\nwendcode{}\nwbegindocs{126}\nwdocspar

\texttt{\textbf{SLOT}} searches the file for the first free record,
starting with the record following the one pointed to by \texttt{\textbf{AVAILABLE}}.
If a free record is found, \texttt{\textbf{SLOT}} sets the file's
\texttt{\textbf{AVAILABLE}} to point to it. \texttt{\textbf{SLOT}}
then stores a -1 into the first cell of the record to indicate that
it is no longer free, and clears the remainder of the record to zeros.
If the file is full, an error message occurs and processing is terminated.

\texttt{\textbf{SLOT}} does not make the new record current, it only
returns the selected record number on the stack. The reason for this
factoring is that you often want to do something with the record number
before consuming it with \texttt{\textbf{READ}} (which makes this
new record the current record). For example, we may wish to cause
a link in the current record to point to the new record, as in the
phrase

\texttt{\textbf{SLOT DUP LINK L! READ}}\textbf{ }

Here the phrase \texttt{\textbf{LINK L!}} must come first because
after the \texttt{\textbf{READ}} we'll be in a different record.

\texttt{\textbf{SCRATCH}} does not change the contents of the record
beyond the first two bytes.

\textbf{REFERENCES}

\texttt{\textbf{N!}}\textbf{,} \prettyref{sub:Field-Reference-Operators}\\
\texttt{\textbf{READ}}, \prettyref{sub:Record-Selection} 


\subsubsection{Accessing Files Sequentially \label{sub:Accessing-Files-Sequentially}}

The following words return appropriate stack arguments for a loop,
which will access the records in a file sequentially:
\begin{description}
\item [{Word}] Stack Description
\item [{\texttt{RECORDS}}] (\ available+1\ 1) Typically used before \texttt{\textbf{?DO}},
returns the content of \texttt{\textbf{AVAILABLE}} (the record number
of the last record allocated) incremented by one and starting index
(1) for a file that has never wrapped around.
\item [{\texttt{WHOLE}}] (\ limit\ 1) Typically used before \texttt{\textbf{?DO}},
returns the content of \texttt{\textbf{LIM}} and starting index (1)
for the entire file.
\end{description}
\nwenddocs{}\nwbegincode{127}\sublabel{NWtG4Lr-1FNAaT-1}\nwmargintag{{\nwtagstyle{}\subpageref{NWtG4Lr-1FNAaT-1}}}\moddef{records~{\nwtagstyle{}\subpageref{NWtG4Lr-1FNAaT-1}}}\endmoddef\nwstartdeflinemarkup\nwusesondefline{\\{NWtG4Lr-4ffLeu-1}}\nwenddeflinemarkup
: RECORDS ( -- l f )   AVAILABLE 4 nC@ 1+  1 ;

: WHOLE ( -- l f )   LIM @ 1 ;
\nwused{\\{NWtG4Lr-4ffLeu-1}}\nwendcode{}\nwbegindocs{128}\nwdocspar

Since these words return the parameters for the \textit{current file},
it's a good habit to invoke the name of the file just before them,
as in \texttt{\textbf{PEOPLE RECORDS}}.

When using \texttt{\textbf{WHOLE}}, you will probably want to check
inside the loop whether each record is currently active. This is normally
done by the phrase:

\texttt{\textbf{LINK L@ IF ...}}

where \texttt{\textbf{LINK}} is the generic long field comprising
the first four bytes of each record. If these bytes contain zero,
the record is available for use.

\textbf{REFERENCES}

Available Records, \prettyref{sub:Available-Records}\\
\texttt{\textbf{LIM}}, \prettyref{sub:File-Definition-Area}


\subsection{Field Definition and Access }

A record description is the list of defined fields that appear in
the record. Each field is an entry in the Forth dictionary, containing
the displacement of the field from the beginning of the record in
its parameter field.

A record description is not formally associated with any particular
file (unless \texttt{\textbf{BIND-FIELDS}} contains \texttt{\textbf{TRUE}}).
It is more like a mask, which is used whenever it is appropriate to
access data.

\nwenddocs{}\nwbegincode{129}\sublabel{NWtG4Lr-2eFuO6-1}\nwmargintag{{\nwtagstyle{}\subpageref{NWtG4Lr-2eFuO6-1}}}\moddef{bind-fields~{\nwtagstyle{}\subpageref{NWtG4Lr-2eFuO6-1}}}\endmoddef\nwstartdeflinemarkup\nwusesondefline{\\{NWtG4Lr-4ffLeu-1}}\nwenddeflinemarkup
VARIABLE BIND-FIELDS   FALSE BIND-FIELDS !
\nwused{\\{NWtG4Lr-4ffLeu-1}}\nwendcode{}\nwbegindocs{130}\nwdocspar

There are several kinds of fields: numbers of various sizes and byte
strings of specified length. The following sections discuss the various
types of fields and the related operators that are used to access
the data stored in them. 


\subsubsection{Record Description \label{sub:Record-Description}}

A record description defines the fields that make up each record in
a file. A record description has the following general format:

\texttt{\textbf{0 field-type field-name}}~\\
\texttt{\textbf{ field-type field-name}}~\\
\texttt{\textbf{ ...}}~\\
\texttt{\textbf{ DROP}}

The various field types are described in \prettyref{sub:Field-Definitions}.

A value is carried on the stack throughout the above process to give
the relative displacement of the beginning of a record. This value
is initial\-ized by the zero at the beginning of the record description,
incremented appropriately by each field definition, and finally discarded
at the end.

In a complex application the fields in a record description may be
defined vertically, like this:

\texttt{\textbf{( PEOPLE file records)}}

\texttt{\textbf{24 BYTES NAME ( Last name first)}}

\texttt{\textbf{24 BYTES STREET ( Street address)}}

\texttt{\textbf{10 BYTES CITY,STATE}}

\texttt{\textbf{DOUBLE ZIP ( zip code, US only)}}

\texttt{\textbf{NUMERIC AREA ( area code)}}

\texttt{\textbf{DOUBLE PHONE ( phone number )}}

\texttt{\textbf{NUMERIC >DETAIL ( link to DETAIL file)}}

\texttt{\textbf{( For employees:)}}

\texttt{\textbf{DOUBLE SS\# ( Social sec. number)}}

\texttt{\textbf{DOUBLE FICA}}

\texttt{\textbf{DOUBLE GROSS ( Gross income ytd)}}

\texttt{\textbf{DROP}}

As a quick check to verify that the number of bytes used for each
record matches the expected value (as specified in the file definition),
replace \texttt{\textbf{DROP}} with \textbf{.} (``dot'') or use
it to define a \texttt{\textbf{CONSTANT}}. This format allows you
to use the shadow block for a general discussion of the file and record.

The field-names defined in the example above---\texttt{\textbf{NAME}},
\texttt{\textbf{STREET}}, etc.---are now entries in the Forth dictionary.
When executed, these words return an address, except for field-names
defined with \texttt{\textbf{BYTES}}, which return a count and address
(see \prettyref{sub:Field-Definitions}).

A record description is not formally attached to a particular file
and has no name. Use of a field name references the relative location
given by that field name in the current record of the current file.

A record in \texttt{\textbf{EMPLOYEES}}:

\begin{figure}
\caption{\label{fig:6}\protect\includegraphics{5_home_druffer_Dropbox_ColorForth_File_Docs_Employees.eps}}
\end{figure}


Thus it is possible to use the same field names for two different
files, even if the record size used in each file varies. For instance,
all the above-defined field names could be used with a file called
\textbf{EMPLOYEES}, while the first six could be used with another
file called \textbf{CUSTOMERS}.

A record in \texttt{\textbf{CUSTOMERS}}:

\begin{figure}
\caption{\label{fig:7}\protect\includegraphics{6_home_druffer_Dropbox_ColorForth_File_Docs_Customers.eps}}
\end{figure}


\textit{NOTE}: Due to the Data Base Support package's record-allocation
scheme, the first field of an active record may never contain a zero
in its first four bytes. In our example, this is not a problem because
the first field is alphanumeric (even blanks are stored as decimal
32's). Otherwise, we would have to rearrange the order of the fields
so that one, which will never contain zero, is first.

\textbf{REFERENCES}

Available Records, \prettyref{sub:Available-Records}\\
Field Types, \prettyref{sub:Field-Definitions} 


\subsubsection{Field Definitions \label{sub:Field-Definitions}}

The following field types are defined:
\begin{description}
\item [{Word}] \textbf{Description}
\item [{\texttt{1BYTE}}] This field is for an 8-bit value (range 0-255).
On processors that do not tolerate odd byte addresses (such as the
PDP-11 and 68000), \texttt{\textbf{1BYTE}} fields must be used in
pairs to avoid wasting space.
\end{description}
Example: \texttt{\textbf{1BYTE AGE}}

Words that are defined by \texttt{\textbf{1BYTE}} return an address,
suitable for use with the one-byte memory access operators \texttt{\textbf{1@}},
\texttt{\textbf{1!}}, \texttt{\textbf{1?}}, and \texttt{\textbf{?1}}.
\begin{description}
\item [{\texttt{NUMERIC}}] These fields occupy two bytes of storage (on
32-bit systems also).
\end{description}
Example: \texttt{\textbf{NUMERIC\ WEIGHT}}

Words defined by \texttt{\textbf{NUMERIC}} return an address, suit\-able
for use with the numeric field access operators \texttt{\textbf{N@}},
\texttt{\textbf{N!}}, \texttt{\textbf{N?}}, and \texttt{\textbf{?N}}.
\begin{description}
\item [{\texttt{LONG}}] These fields occupy four bytes of storage (only
available on 32-bit systems).
\end{description}
Example: \texttt{\textbf{LONG\ WEIGHT}}

Words defined by \texttt{\textbf{LONG}} return an address, suit\-able
for use with the numeric field access operators \texttt{\textbf{L@}},
\texttt{\textbf{L!}}, \texttt{\textbf{L?}}, and \texttt{\textbf{?L}}.
\begin{description}
\item [{\texttt{DOUBLE}}] This field is for a 64-bit (8-byte) value.
\end{description}
Example: \texttt{\textbf{DOUBLE SALARY}}

Words defined by \texttt{\textbf{DOUBLE}} return an address, suitable
for use with the double field access operators \texttt{\textbf{D@}},
\texttt{\textbf{D!}}, \texttt{\textbf{D?}}, and \texttt{\textbf{?D}}
.
\begin{description}
\item [{\texttt{BYTES}}] This field is for alphanumeric text. A count is
required to specify the number of bytes in the field.
\end{description}
Example: \texttt{\textbf{24 BYTES NAME}}

Words defined by \texttt{\textbf{BYTES}} return a length and address,
suitable for use with the byte field access operators \texttt{\textbf{B@}},
\texttt{\textbf{B!}}, \texttt{\textbf{B?}}, and \texttt{\textbf{?B}}.
The width of a \texttt{\textbf{BYTES}} field must be even.

\nwenddocs{}\nwbegincode{131}\sublabel{NWtG4Lr-7mIQU-1}\nwmargintag{{\nwtagstyle{}\subpageref{NWtG4Lr-7mIQU-1}}}\moddef{fields~{\nwtagstyle{}\subpageref{NWtG4Lr-7mIQU-1}}}\endmoddef\nwstartdeflinemarkup\nwusesondefline{\\{NWtG4Lr-4ffLeu-1}}\nwenddeflinemarkup
: (FIELD) ( o n -- o+n )   CREATE  OVER , +  DOES> ( a -- a' )   @  FLD# @ + ;

0 VALUE #FIELD

#FIELD CELL (FIELD) FIELD-OFFSET \\ holds offset within the record.
       CELL (FIELD) FIELD-SIZE   \\ holds the size of the field.
       CELL (FIELD) FIELD-COPIES \\ holds numbers of times this field is repeated.
       CELL (FIELD) BOUND-FILE   \\ holds addr of FCB this record is bound to, or 0.
TO #FIELD

: CREATE-FIELD ( o n -- o+n )
    HERE DUP FLD# !  #FIELD DUP ALLOT  ERASE
    2DUP FIELD-SIZE !  FIELD-OFFSET !  +
    0 FIELD-COPIES !  BIND-FIELDS @ IF
        DB# @ BOUND-FILE !
    ELSE  0 BOUND-FILE !
    THEN ;

: FIELD-DOES ( a -- a' )
    FLD# !  WORKING FIELD-OFFSET @ +
    BOUND-FILE @ ?DUP IF
        SET-DATA  #REC @ R# !
    THEN  FIELD-COPIES @ IF
        #INDEX @  FIELD-SIZE @ * +
    THEN ;

: 1BYTE ( o _ -- o+1 ) \\ 1BYTE fields occupy 8 bits.
    CREATE 1 CREATE-FIELD  DOES> ( -- a )   FIELD-DOES ;

: NUMERIC ( o _ -- o+2 ) \\ NUMERIC fields occupy 16 bits.
    CREATE 2 CREATE-FIELD  DOES> ( -- a )   FIELD-DOES ;

: LONG ( o _ -- o+4 ) \\ LONG fields occupy 32 bits.
    CREATE 4 CREATE-FIELD  DOES> ( -- a )   FIELD-DOES ;

[DEFINED] SFALIGN [IF]
: FLOAT ( o _ -- o+4 ) \\ FLOAT fields occupy 32 bits.
    CREATE 4 CREATE-FIELD  DOES> ( -- a )   FIELD-DOES ;
[THEN]

: DOUBLE ( o _ -- o+4 ) \\ DOUBLE fields occupy 64 bits.
    CREATE 8 CREATE-FIELD  DOES> ( -- a )   FIELD-DOES ;

: BYTES ( o n _ -- o+n ) \\ BYTES fields occupy a specified number of bytes.
    CREATE CREATE-FIELD  DOES> ( -- n a )   FIELD-DOES  FIELD-SIZE @ SWAP ;
\nwused{\\{NWtG4Lr-4ffLeu-1}}\nwendcode{}\nwbegindocs{132}\nwdocspar

\begin{description}
\item [{\texttt{FILLER}}] This field reserves space in the record, typically
used for future expansion or to skip regions of a record that are
to be accessed by other means. \texttt{\textbf{FILLER}} requires the
number of bytes to be reserved.
\end{description}
\nwenddocs{}\nwbegincode{133}\sublabel{NWtG4Lr-4UfOWH-1}\nwmargintag{{\nwtagstyle{}\subpageref{NWtG4Lr-4UfOWH-1}}}\moddef{filler~{\nwtagstyle{}\subpageref{NWtG4Lr-4UfOWH-1}}}\endmoddef\nwstartdeflinemarkup\nwusesondefline{\\{NWtG4Lr-4ffLeu-1}}\nwenddeflinemarkup
[defined] -warning [IF] -warning [THEN]
: FILLER ( o n _ -- o+n )   BL WORD DROP + ;
[defined] +warning [IF] +warning [THEN] 
\nwused{\\{NWtG4Lr-4ffLeu-1}}\nwendcode{}\nwbegindocs{134}\nwdocspar

Example: \texttt{\textbf{6 FILLER <any name>}}

\texttt{\textbf{FILLER}} creates no dictionary entry, but simply discards
the word that follows.

At compile time, the numeric field defining words (\texttt{\textbf{1BYTE}},
\texttt{\textbf{NUMERIC}}, \texttt{\textbf{DOUBLE}}) expect the current
displacement in the record on the stack. A copy of the displacement
is compiled in the parameter field of the definition, and its value
on the stack is incremented by the size of the field in bytes. A \texttt{\textbf{BYTES}}
field also expects the size of the field on the stack. This value
is compiled along with the displacement, and used to increment the
displacement accordingly.

When a field-name defined by one of these words is \textit{executed},
it pushes onto the stack the address of working storage, incremented
by the displacement of the field to give the address of the field
in the record image in working storage. In the case of \texttt{\textbf{BYTES}}
fields, the size of the field is beneath the address on the stack.
The working storage address (and size, in the case of \texttt{\textbf{BYTES}}
fields) is the appropriate input to the field access operators described
in the next section.
\begin{description}
\item [{\texttt{INDEX}}] sets the current field index. Used to select one
of the fields in a \texttt{\textbf{COPIES}} field.
\end{description}
\nwenddocs{}\nwbegincode{135}\sublabel{NWtG4Lr-36msie-1}\nwmargintag{{\nwtagstyle{}\subpageref{NWtG4Lr-36msie-1}}}\moddef{indexed~{\nwtagstyle{}\subpageref{NWtG4Lr-36msie-1}}}\endmoddef\nwstartdeflinemarkup\nwusesondefline{\\{NWtG4Lr-4ffLeu-1}}\nwenddeflinemarkup
: INDEX ( n -- )   #INDEX ! ;
: INDEXED ( -- n )   #INDEX @ ;
\nwused{\\{NWtG4Lr-4ffLeu-1}}\nwendcode{}\nwbegindocs{136}\nwdocspar

\begin{description}
\item [{\texttt{COPIES}}] makes an array of the previous field. Use \texttt{\textbf{INDEX}}
to access the individual elements.
\end{description}
\nwenddocs{}\nwbegincode{137}\sublabel{NWtG4Lr-3NRM5h-1}\nwmargintag{{\nwtagstyle{}\subpageref{NWtG4Lr-3NRM5h-1}}}\moddef{copies~{\nwtagstyle{}\subpageref{NWtG4Lr-3NRM5h-1}}}\endmoddef\nwstartdeflinemarkup\nwusesondefline{\\{NWtG4Lr-4ffLeu-1}}\nwenddeflinemarkup
: COPIES ( o n -- o' )
    FIELD-COPIES @ ABORT" Invalid COPIES"  DUP
    FIELD-COPIES !  1- FIELD-SIZE @ * + ; 
\nwused{\\{NWtG4Lr-4ffLeu-1}}\nwendcode{}\nwbegindocs{138}\nwdocspar

\textbf{REFERENCES}

Access to the Record Image in Working Storage, \prettyref{sub:Access-to-Working-Storage}\\
Available Records, \prettyref{sub:Available-Records}\\
Direct Access to Fields, \prettyref{sub:Direct-Access-to-Fields}\\
Field Reference Operators, \prettyref{sub:Field-Reference-Operators}\\
Working Storage, \prettyref{sub:Working-Storage} 


\subsubsection{Field Reference Operators \label{sub:Field-Reference-Operators}}

Fields in files are referenced with special words. The following operators
assume that the desired file and record have been selected. They refer
to fields in the current record (as indicated by the value of user
variable \texttt{\textbf{R\#}}; see \prettyref{sub:Record-Selection}).
In all cases, the name of the field precedes the operator; the field-name
returns the appropriate address (and length, in the case of \texttt{\textbf{BYTES}}
fields) to be used by the access operator.
\begin{description}
\item [{Word}] \textbf{Stack Action}
\item [{\texttt{1@}}] (\ a\ -\ b) Fetches the contents of a \texttt{\textbf{1BYTE}}
field to the top of the stack.
\item [{\texttt{1!}}] (\ b\ a\ -\ ) Stores a byte into the \texttt{\textbf{1BYTE}}
field whose address is on top of the stack.
\end{description}
\nwenddocs{}\nwbegincode{139}\sublabel{NWtG4Lr-4CCogU-1}\nwmargintag{{\nwtagstyle{}\subpageref{NWtG4Lr-4CCogU-1}}}\moddef{byte~{\nwtagstyle{}\subpageref{NWtG4Lr-4CCogU-1}}}\endmoddef\nwstartdeflinemarkup\nwusesondefline{\\{NWtG4Lr-4ffLeu-1}}\nwenddeflinemarkup
: 1@ ( a -- n )   ADDRESS C@ ;
: 1! ( n a -- )   ADDRESS C! TOUCH ; 
\nwused{\\{NWtG4Lr-4ffLeu-1}}\nwendcode{}\nwbegindocs{140}\nwdocspar

\begin{description}
\item [{\texttt{1?}}] (\ a) Fetches and displays the contents of a \texttt{\textbf{1BYTE}}
field.
\item [{\texttt{?1}}] (\ a) As for \texttt{\textbf{1?}}, except the results
are right-justified by the report generator.
\item [{\texttt{N@}}] (\ a\ -\ n) Fetches the contents of a \texttt{\textbf{NUMERIC}}
field to the top of the stack.
\item [{\texttt{N!}}] (\ n\ a\ -\ ) Stores a number into the \texttt{\textbf{NUMERIC}}
field whose address is on top of the stack.
\end{description}
\nwenddocs{}\nwbegincode{141}\sublabel{NWtG4Lr-44RunE-1}\nwmargintag{{\nwtagstyle{}\subpageref{NWtG4Lr-44RunE-1}}}\moddef{numeric~{\nwtagstyle{}\subpageref{NWtG4Lr-44RunE-1}}}\endmoddef\nwstartdeflinemarkup\nwusesondefline{\\{NWtG4Lr-4ffLeu-1}}\nwenddeflinemarkup
: NU@ ( a -- n )   ADDRESS 2 nC@ ;
: N@ ( a -- n )   ADDRESS 2 nC@
    DUP 32768 AND IF  -65536 OR  THEN ;
: N! ( n a -- )   ADDRESS 2 nC! TOUCH ; 
\nwused{\\{NWtG4Lr-4ffLeu-1}}\nwendcode{}\nwbegindocs{142}\nwdocspar

\begin{description}
\item [{\texttt{N?}}] (\ a) Fetches and displays the contents of a \texttt{\textbf{NUMERIC}}
field.
\item [{\texttt{?N}}] (\ a) As for \texttt{\textbf{N?}}, except the results
are right-justified by the report generator.
\item [{\texttt{L@}}] (\ a\ -\ n) Fetches the contents of a \texttt{\textbf{LONG}}
field to the top of the stack.
\item [{\texttt{L!}}] (\ n\ a\ -\ ) Stores a number into the \texttt{\textbf{LONG}}
field whose address is on top of the stack.
\end{description}
\nwenddocs{}\nwbegincode{143}\sublabel{NWtG4Lr-2qounD-1}\nwmargintag{{\nwtagstyle{}\subpageref{NWtG4Lr-2qounD-1}}}\moddef{long~{\nwtagstyle{}\subpageref{NWtG4Lr-2qounD-1}}}\endmoddef\nwstartdeflinemarkup\nwusesondefline{\\{NWtG4Lr-4ffLeu-1}}\nwenddeflinemarkup
: LU@ ( a -- n )   ADDRESS 4 nC@ ;
: L@ ( a -- n )   ADDRESS 4 nC@
    DUP 2147483648 AND IF  -4294967295 OR  THEN ;
: L! ( n a -- )   ADDRESS 4 nC! TOUCH ; 
\nwused{\\{NWtG4Lr-4ffLeu-1}}\nwendcode{}\nwbegindocs{144}\nwdocspar

\begin{description}
\item [{\texttt{L?}}] (\ a) Fetches and displays the contents of a \texttt{\textbf{LONG}}
field.
\item [{\texttt{?L}}] (\ a) As for \texttt{\textbf{L?}}, except the results
are right-justified by the report generator.
\item [{\texttt{FL@~FL!}}] move 32-bit floats between the stack and the
file.
\end{description}
\nwenddocs{}\nwbegincode{145}\sublabel{NWtG4Lr-2mGd9n-1}\nwmargintag{{\nwtagstyle{}\subpageref{NWtG4Lr-2mGd9n-1}}}\moddef{float~{\nwtagstyle{}\subpageref{NWtG4Lr-2mGd9n-1}}}\endmoddef\nwstartdeflinemarkup\nwusesondefline{\\{NWtG4Lr-4ffLeu-1}}\nwenddeflinemarkup
[DEFINED] SFALIGN [IF]

SFALIGN HERE 1 SFLOATS ALLOT CONSTANT DBFLOAT

: FL@ ( a -- r )   ADDRESS 4 nC@ DBFLOAT ! DBFLOAT SF@ ;
: FL! ( r a -- )   DBFLOAT SF! DBFLOAT @ SWAP ADDRESS 4 nC! TOUCH ;

: FL@-le ( a -- r )   ADDRESS 4 c@-le DBFLOAT ! DBFLOAT SF@ ;
: FL!-le ( r a -- )   DBFLOAT SF! DBFLOAT @ SWAP ADDRESS 4 c!-le TOUCH ;

: FL@-be ( a -- r )   ADDRESS 4 c@-be DBFLOAT ! DBFLOAT SF@ ;
: FL!-be ( r a -- )   DBFLOAT SF! DBFLOAT @ SWAP ADDRESS 4 c!-be TOUCH ;

: FL@+ ( a -- r )   ADDRESS SF@ ;   \\ These are fast, but endian dependant
: FL!+ ( r a -- )   ADDRESS SF! TOUCH ;
[THEN]
\nwused{\\{NWtG4Lr-4ffLeu-1}}\nwendcode{}\nwbegindocs{146}\nwdocspar

\begin{description}
\item [{\texttt{D@}}] (\ a\ -\ d) Fetches the contents of a \texttt{\textbf{DOUBLE}}
field to the top of the stack (two cells).
\item [{\texttt{D!}}] (\ d\ a\ -\ ) Stores two cells into the \texttt{\textbf{DOUBLE}}
field whose address is on top of the stack.
\end{description}
\nwenddocs{}\nwbegincode{147}\sublabel{NWtG4Lr-4Sesx5-1}\nwmargintag{{\nwtagstyle{}\subpageref{NWtG4Lr-4Sesx5-1}}}\moddef{double~{\nwtagstyle{}\subpageref{NWtG4Lr-4Sesx5-1}}}\endmoddef\nwstartdeflinemarkup\nwusesondefline{\\{NWtG4Lr-4ffLeu-1}}\nwenddeflinemarkup
: D@ ( a -- d )   ADDRESS  CELL CASE
        2 OF  4 nC@  ENDOF
        4 OF  8 nC@  ENDOF
        8 OF  8 nC@ 4294967296 /MOD SWAP  ENDOF
        DUP OF  1 ABORT" CELL size not supported"  ENDOF
    ENDCASE ;
: D! ( d a -- )   ADDRESS  CELL CASE
        2 OF  4 nC!  ENDOF
        4 OF  8 nC!  ENDOF
        8 OF  SWAP 4294967296 + 8 nC!  ENDOF
        DUP OF  1 ABORT" CELL size not supported"  ENDOF
    ENDCASE  TOUCH ;
\nwused{\\{NWtG4Lr-4ffLeu-1}}\nwendcode{}\nwbegindocs{148}\nwdocspar

\begin{description}
\item [{\texttt{D?}}] (\ a) Fetches and displays the contents of a \texttt{\textbf{DOUBLE}}
field.
\item [{\texttt{?D}}] (\ a) As for \texttt{\textbf{D?}}, except the results
are right-justified by the report generator.
\item [{\texttt{B@}}] (\ n\ a) Reads a \texttt{\textbf{BYTES}} field,
according to the declared length, into \texttt{\textbf{PAD}}.
\item [{\texttt{B!}}] (\ n\ a) Stores a \texttt{\textbf{BYTES}} field,
according to the declared length, from \texttt{\textbf{PAD}}.
\end{description}
\nwenddocs{}\nwbegincode{149}\sublabel{NWtG4Lr-4T6feO-1}\nwmargintag{{\nwtagstyle{}\subpageref{NWtG4Lr-4T6feO-1}}}\moddef{bytes~{\nwtagstyle{}\subpageref{NWtG4Lr-4T6feO-1}}}\endmoddef\nwstartdeflinemarkup\nwusesondefline{\\{NWtG4Lr-4ffLeu-1}}\nwenddeflinemarkup
: B@ ( n a -- )   ADDRESS S@ ;
: B! ( n a -- )   ADDRESS S! TOUCH ;
\nwused{\\{NWtG4Lr-4ffLeu-1}}\nwendcode{}\nwbegindocs{150}\nwdocspar

\begin{description}
\item [{\texttt{B?}}] (\ n\ a) Fetches and displays the contents of a
\texttt{\textbf{BYTES}} field, according to the declared length. \texttt{\textbf{PAD}}
is used as intermediate storage of the field data.
\item [{\texttt{?B}}] (\ n\ a) As for \texttt{\textbf{B?}}, except the
results are right-justified by the report generator.
\end{description}
Example of usage:
\begin{description}
\item [{\texttt{GROSS\ D@}}] Fetches the contents of the \texttt{\textbf{DOUBLE}}
field \texttt{\textbf{GROSS}} onto the stack.
\end{description}
\nwenddocs{}\nwbegincode{151}\sublabel{NWtG4Lr-17LFpz-1}\nwmargintag{{\nwtagstyle{}\subpageref{NWtG4Lr-17LFpz-1}}}\moddef{access~{\nwtagstyle{}\subpageref{NWtG4Lr-17LFpz-1}}}\endmoddef\nwstartdeflinemarkup\nwusesondefline{\\{NWtG4Lr-4ffLeu-1}}\nwenddeflinemarkup
: N? ( a -- )   N@ . ;
: NU? ( a -- )   NU@ . ;
: L? ( a -- )   L@ . ;
[DEFINED] SFALIGN [IF]
: FL? ( a -- )   FL@ F. ;
[THEN]
: D? ( a -- )   D@ . ;
: 1? ( a -- )   1@ . ;
: B? ( n a -- )   2DUP B@ S.  SPACE ;
\nwused{\\{NWtG4Lr-4ffLeu-1}}\nwendcode{}\nwbegindocs{152}\nwdocspar

Two other words are included for storing data into \texttt{\textbf{BYTES}}
fields:
\begin{description}
\item [{Word}] \textbf{Stack Action}
\item [{\texttt{PUT}}] (\ n\ a) Copies the remainder of the input stream
into a \texttt{\textbf{BYTES}} field.
\end{description}
\nwenddocs{}\nwbegincode{153}\sublabel{NWtG4Lr-105aWa-1}\nwmargintag{{\nwtagstyle{}\subpageref{NWtG4Lr-105aWa-1}}}\moddef{put~{\nwtagstyle{}\subpageref{NWtG4Lr-105aWa-1}}}\endmoddef\nwstartdeflinemarkup\nwusesondefline{\\{NWtG4Lr-4ffLeu-1}}\nwenddeflinemarkup
: PUT ( n a -- )   1 TEXT B! ;
\nwused{\\{NWtG4Lr-4ffLeu-1}}\nwendcode{}\nwbegindocs{154}\nwdocspar

For example:\textbf{ }\texttt{\textbf{NAME PUT Fred Ferguson}}\texttt{
ok}

A string that is too long will be truncated when it is stored. If
it is shorter than the field size, it will be blank-filled. A copy
of the entire string is left in \texttt{\textbf{PAD}}.
\begin{description}
\item [{\texttt{ASK}}] (\ n\ a) Awaits (via \texttt{\textbf{ACCEPT}})
input from the keyboard, and copies it into a \texttt{\textbf{BYTES}}
field using \texttt{\textbf{PUT}}.
\end{description}
\nwenddocs{}\nwbegincode{155}\sublabel{NWtG4Lr-2k8QJQ-1}\nwmargintag{{\nwtagstyle{}\subpageref{NWtG4Lr-2k8QJQ-1}}}\moddef{ask~{\nwtagstyle{}\subpageref{NWtG4Lr-2k8QJQ-1}}}\endmoddef\nwstartdeflinemarkup\nwusesondefline{\\{NWtG4Lr-4ffLeu-1}}\nwenddeflinemarkup
: ASK ( n a -- )   FILE-PAD #TB 2DUP BLANK ACCEPT DROP B! ;
\nwused{\\{NWtG4Lr-4ffLeu-1}}\nwendcode{}\nwbegindocs{156}\nwdocspar

The word \texttt{\textbf{ENTIRE}} may be used in place of a field
name:
\begin{description}
\item [{Word}] \textbf{Stack Action}
\item [{\texttt{ENTIRE}}] (\ --\ n\ a) Returns parameters for the ``pseudo-field''
that occupies the entire record in \texttt{\textbf{BYTES}} format.
\end{description}
\nwenddocs{}\nwbegincode{157}\sublabel{NWtG4Lr-uRyxG-1}\nwmargintag{{\nwtagstyle{}\subpageref{NWtG4Lr-uRyxG-1}}}\moddef{entire~{\nwtagstyle{}\subpageref{NWtG4Lr-uRyxG-1}}}\endmoddef\nwstartdeflinemarkup\nwusesondefline{\\{NWtG4Lr-4ffLeu-1}}\nwenddeflinemarkup
: ENTIRE ( -- n a )   B/R @ WORKING ;
\nwused{\\{NWtG4Lr-4ffLeu-1}}\nwendcode{}\nwbegindocs{158}\nwdocspar

For example:\textbf{ }\texttt{\textbf{ENTIRE B?}}

types the contents of the current record as though it were a single
\texttt{\textbf{BYTES}} field.

\textbf{REFERENCES}

Fetching Input to \textbf{PAD}, \prettyref{sub:How-Data-is-Stored}\\
Report Generator, \prettyref{sub:Report-Generator} 


\subsubsection{Direct Access to Fields \label{sub:Direct-Access-to-Fields}}

The Data Base Support package is set up so that field names may be
used with field access operators in a transparent way, although in
fact more is going on with these words than meets the eye. In the
event that you need to directly access fields in a file (for instance,
if you wish to use \texttt{\textbf{MOVE}}, \texttt{\textbf{ERASE}},
etc. instead of \texttt{\textbf{N!}}, etc.), you should understand
the details explained in this section.

The addresses returned by user-defined field names are intended to
be consumed by the field reference operators (\prettyref{sub:Field-Reference-Operators}).
These addresses, however, are not the addresses of the actual data
in a block buffer, but rather addresses within working storage (\prettyref{sub:Working-Storage}).
The field refer\-ence operators perform the necessary offset correction,
call the appropriate block and access the data. In the case of ``fetch''
operators, the operators move the data elsewhere (numbers are pushed
onto the stack; strings are moved to \texttt{\textbf{PAD}}). This
allows the field-name words, which return the address, to be used
transparently with either working storage or the file data itself;
the difference depends solely upon the operator that fetches or stores
the data.

Each field reference operation can be an implied disk access, since
it may call file I/O operators. It is important not to carry the address
of a field in an I/O buffer on the stack across any I/O operation
(such as displaying the content of a field or accessing another field),
since in a multitasking environment another task may perform disk
activity that changes the content of the I/O buffer.

Occasionally it may be useful to bypass the protection of the field
reference operators, and determine the actual address of a field in
a disk buffer. This can be done by the following phrase:

\texttt{\textbf{field-name ADDRESS}}

This phrase places the actual memory address of the field on top of
the stack.

\nwenddocs{}\nwbegincode{159}\sublabel{NWtG4Lr-1pHJON-1}\nwmargintag{{\nwtagstyle{}\subpageref{NWtG4Lr-1pHJON-1}}}\moddef{address~{\nwtagstyle{}\subpageref{NWtG4Lr-1pHJON-1}}}\endmoddef\nwstartdeflinemarkup\nwusesondefline{\\{NWtG4Lr-4ffLeu-1}}\nwenddeflinemarkup
: RECORD ( n -- a )   'RECORD @ ?DUP IF EXECUTE THEN ;

: TOUCH ( -- )   'TOUCH @ ?DUP IF EXECUTE THEN ;

: ADDRESS ( a -- a' )   R# @ RECORD  WORKING -  + ;
\nwused{\\{NWtG4Lr-4ffLeu-1}}\nwendcode{}\nwbegindocs{160}\nwdocspar

For example, the following phrase will move an array of 100 2-byte
data elements from working storage to disk much faster than it would
take to calculate addresses repeatedly using \texttt{\textbf{N!}}:

\texttt{\textbf{DATA DATA ADDRESS 200 MOVE TOUCH}}

The first use of \texttt{\textbf{DATA}} returns the address of the
image of the field in working storage. The phrase \texttt{\textbf{DATA\ ADDRESS}}
returns the location of the field in virtual memory. \texttt{\textbf{200\ MOVE}}
moves the image in working storage to the disk buffer. \texttt{\textbf{TOUCH}}
is necessary after writing to a disk buffer.

For \texttt{\textbf{BYTES}} fields (since invoking the name of a \texttt{\textbf{BYTES}}
field pushes both the location and length onto the stack), the phrase:

\texttt{\textbf{field-name ADDRESS}}\textbf{ }

returns the length and virtual memory address (note that the order
is reversed from the standard ``address, count'' order).

If direct addressing is used, you must remember that the content of
the buffer can change at any time the task either requests I/O from
any source or causes execution of \texttt{\textbf{PAUSE}} or \texttt{\textbf{WAIT}}.
Furthermore, if you modify the contents of any field directly (without
using \texttt{\textbf{N!}}, \texttt{\textbf{B!}}, etc.), you must
invoke \texttt{\textbf{TOUCH}} after the modification.


\subsubsection{Access to the Record Image in Working Storage \label{sub:Access-to-Working-Storage}}

Because field names return addresses within local working storage,
you can directly access the working storage image of a record. This
lets you map data items as though they were contained in records,
although they are kept in resident memory instead of on the disk.
There is only one ``record'' in the working storage area.

Using ordinary memory-access operators in conjunction with field names
provides access to working storage locations:
\begin{description}
\item [{Word}] \textbf{Action}
\item [{\texttt{C@}}] Fetches an 8-bit number.\\
 Example: \texttt{\textbf{AGE\ C@}}
\item [{\texttt{C!}}] Stores an 8-bit number.\\
 Example: \texttt{\textbf{39\ AGE\ C!}}
\item [{\texttt{@}}] Fetches a single-length number. \\
 Example: \texttt{\textbf{LINK\ @}}
\item [{\texttt{!}}] Stores a single-length number. \\
 Example: \texttt{\textbf{16\ LINK\ !}}
\item [{\texttt{2@}}] Fetches a double-length number.\\
 Example: \texttt{\textbf{PRICE\ 2@}}
\item [{\texttt{2!}}] Stores a double-length number.\\
 Example: \texttt{\textbf{196.75\ PRICE\ 2!}}
\item [{\texttt{S@}}] Fetches a string from working storage to \texttt{\textbf{PAD}}.\\
 Example: \texttt{\textbf{NAME\ S@}}
\item [{\texttt{S!}}] Stores a string from \texttt{\textbf{PAD}} into working
storage.\\
 Example: \texttt{\textbf{NAME\ S!}}
\end{description}
\nwenddocs{}\nwbegincode{161}\sublabel{NWtG4Lr-jUK2q-1}\nwmargintag{{\nwtagstyle{}\subpageref{NWtG4Lr-jUK2q-1}}}\moddef{strings~{\nwtagstyle{}\subpageref{NWtG4Lr-jUK2q-1}}}\endmoddef\nwstartdeflinemarkup\nwusesondefline{\\{NWtG4Lr-4ffLeu-1}}\nwenddeflinemarkup
: S@ ( n a -- )   FILE-PAD ROT >MOVE< ;
: S! ( n a -- )   FILE-PAD SWAP ROT >MOVE< ;
: (S.) ( n a -- a' n' )   DROP FILE-PAD SWAP -TRAILING ;
: S. ( n a -- )   (S.) xTYPE ;
\nwused{\\{NWtG4Lr-4ffLeu-1}}\nwendcode{}\nwbegindocs{162}\nwdocspar

The word \texttt{\textbf{WORKING}} returns the address of the beginning
of the task's working storage. Note that there is no common standard
for manipulating half-cell quantities (i.e. 16-bit values on a 32-bit
system), so it is best to avoid using them within this context.

\textbf{REFERENCES}

Working Storage, \prettyref{sub:Working-Storage} 


\subsection{Ordered Index Files \label{sub:Ordered-Index-Files}}

An ordered index file is one in which the records are kept in ascending
order depending upon the ASCII values of a \textit{key}. A key is
an item of data that is used in a match or comparison.

There are two purposes for an ordered index file. First, it greatly
speeds up searches based on the key data. Second, it allows you to
display the main file alphabetically without having to sort it.

Each record in the index file contains a key together with a link
address to an associated main file. This link resides in a 32-bit
field called \texttt{\textbf{LINK}}. In Fig. \prettyref{fig:8}, the
index file \textbf{(}\texttt{\textbf{NAMES}}\textbf{)} contains the
names of people, ordered alphabetically, along with links to the main
file.

\begin{figure}
\caption{\label{fig:8}\protect\includegraphics{7_home_druffer_Dropbox_ColorForth_File_Docs_People.eps}}
\end{figure}


An ordered index file (on the left) showing links to the corresponding
records in the main data file (right).

You may have several index files addressing the same main file. For
example, both sample name and observation number, using two separate
index files, could index a file of scientific data. In Fig. \prettyref{fig:9},
a second index file (\texttt{\textbf{COMPANIES}}) points to the same
main file \texttt{\textbf{PEOPLE}}, but uses the company field as
a key, and keeps the records ordered alphabetically by company.

Searches on an ordered index are performed using a ``binary search,''
which locates a record (or the place that it should go if it is not
in the file) with only log2 \textit{n} steps rather than n/2 (which
is the average for a ``brute force'' or sequential search).

A binary search works by taking the occupied part of the file and
dividing it by two, then comparing the desired key with the field
in the middle record. If the key is larger, then the high half of
the file is halved again. This process is repeated until the size
of the remaining set of records is one. This remaining record must
match the key, if the key is in the file; other\-wise, it is the
record \textit{before} which the key would be inserted. For a file
of 128 records, a binary search requires only seven comparisons, as
compared with an average of 64 for a sequential search.

\begin{figure}
\caption{\label{fig:9}\protect\includegraphics{8_home_druffer_Dropbox_ColorForth_File_Docs_Companies.eps}}
\end{figure}


Another index to the same main file shown in Fig. \prettyref{fig:8},
this time using the company name as key.

An ordered index is a ``dense file.'' That is, there are no gaps
between active records. Therefore, \texttt{\textbf{AVAILABLE}} always
reflects the number of records in the index file, and all records
in the index file can be accessed with a \texttt{\textbf{DO\ LOOP}}
with the knowledge that all records are active. With files maintained
using \texttt{\textbf{SLOT}} and \texttt{\textbf{SCRATCH}}, you must
check the \texttt{\textbf{LINK}} field (first four bytes of every
record) to see whether each record is active. 

\nwenddocs{}\nwbegincode{163}\sublabel{NWtG4Lr-2dg69x-1}\nwmargintag{{\nwtagstyle{}\subpageref{NWtG4Lr-2dg69x-1}}}\moddef{index~{\nwtagstyle{}\subpageref{NWtG4Lr-2dg69x-1}}}\endmoddef\nwstartdeflinemarkup\nwusesondefline{\\{NWtG4Lr-2V5MiK-1}}\nwenddeflinemarkup
\LA{}head~{\nwtagstyle{}\subpageref{NWtG4Lr-QPhyQ-1}}\RA{}
\LA{}+ordered~{\nwtagstyle{}\subpageref{NWtG4Lr-xEqwf-1}}\RA{}
\LA{}-ordered~{\nwtagstyle{}\subpageref{NWtG4Lr-2HqFpk-1}}\RA{}
\LA{}-binary~{\nwtagstyle{}\subpageref{NWtG4Lr-1Wzvls-1}}\RA{}
\LA{}binary~{\nwtagstyle{}\subpageref{NWtG4Lr-1rYrP5-1}}\RA{}
\LA{}snatch~{\nwtagstyle{}\subpageref{NWtG4Lr-4GTjjD-1}}\RA{}
\LA{}-next~{\nwtagstyle{}\subpageref{NWtG4Lr-49QaAb-1}}\RA{}
\LA{}first~{\nwtagstyle{}\subpageref{NWtG4Lr-2i9BK7-1}}\RA{}
\LA{}-locate~{\nwtagstyle{}\subpageref{NWtG4Lr-3ikf9S-1}}\RA{}
\LA{}chain~{\nwtagstyle{}\subpageref{NWtG4Lr-4AREak-1}}\RA{}
\LA{}unchain~{\nwtagstyle{}\subpageref{NWtG4Lr-3m1wNz-1}}\RA{}
\nwused{\\{NWtG4Lr-2V5MiK-1}}\nwendcode{}\nwbegindocs{164}\nwdocspar


\subsubsection{Index File Records \label{sub:Index-File-Records}}

At minimum, an ordered index file must contain the key and the link
that associates the key with its main data record. The link is a 32-bit
record number residing in the first two bytes of the record, and the
key field immediately follows.

You can keep data other than keys in an index file and process this
data in the same manner as data in other types of files. Such a technique
should be avoided, however, if more than one user will have simultaneous
access to the file, because record numbers may change due to insertion
or deletion by other users.

The time required to search an index depends upon the length of each
record as well as the number of records, because longer records will
require more blocks to store the file, and hence more disk accesses
to search it. Therefore, you should keep these records as small as
possible.

The first four bytes of each record in an index file contain the link
to the associated record in the main file. polyFORTH ISD-4 predefines
this field as \texttt{\textbf{LINK}}. The phrase:

\texttt{\textbf{LINK L@}}

reads the link field of the currently selected record and returns
it on the stack.

When creating the record description (\prettyref{sub:Record-Description})
for an index file, you must skip over the \texttt{\textbf{LINK}} field
by using the phrase \texttt{\textbf{4\ FILLER <name>}} at the beginning
of the layout, or by starting with a displacement of four rather than
zero. For example:

\texttt{\textbf{0 4 FILLER LINK ( Link to PEOPLE file)}}~\\
\texttt{\textbf{ 10 BYTES NICK ( Last name key)}}~\\
\texttt{\textbf{ DROP}}

The key may be ASCII or binary. In order to make it possible to use
binary integers as keys, as well as to speed up the search, the comparison
made in the search routine compares \textit{word-by-word}, rather
than byte-by-byte. To accommodate this, you must make your key fields
an even number of bytes in length. To allow the exchange of data with
machines which use a byte order that would render the most significant
byte the second one in a string, the operators \texttt{\textbf{B@}}
and \texttt{\textbf{B!}} reverse bytes when fetching and storing from
disk such that the data on disk is in a compatible order. The can
be defeated by setting \texttt{\textbf{REVERSE}} to \texttt{\textbf{FALSE}}.

Be aware that the order of the records in the index file is subject
to frequent change as a result of file insertion or deletion. Because
the record number of an index record may change, it should not be
used directly for any purpose.

You must also take special care when sharing ordered files. We suggest
you limit the index file to keys, and keep all other data in an associated
main file record. Otherwise, a task may be pointing at a current record
in an index, but before it accesses the data in the record the index
record changes position. 


\subsubsection{Ordered File Maintenance \label{sub:Ordered-File-Maintenance}}

An ``ordered index'' file in polyFORTH is one in which the keys
are main\-tained in ascending ASCII sequence. For instance, an index
to a file of records of people might be ordered by last names.

An ordered file allows quick searching on key fields. For instance,
given a name, we can search the index file looking for a match. From
the index record where the match was found, we can obtain the link
to the main file. 


\paragraph{Searching an Ordered Index \label{par:Searching-an-Ordered-Index}}

In polyFORTH, this routine is called \texttt{\textbf{BINARY}} (named
because it performs a binary search). Here's how it works:

As we've seen (\prettyref{sub:Access-to-Working-Storage}), field
names return the address of the field in the ``image'' of the record
in working storage. \texttt{\textbf{BINARY}} expects to find the match
criteria for the desired field in this image (Fig. \prettyref{fig:10}).

\begin{figure}
\caption{\label{fig:10}\protect\includegraphics{9_home_druffer_Dropbox_ColorForth_File_Docs_Binary.eps}}
\end{figure}

\begin{description}
\item [{\texttt{BINARY}}] searches the ordered index for a match to the
key in working storage. It returns the content of the \texttt{\textbf{LINK}}
field of the matching record, and aborts if there is no match.
\end{description}
Make sure that you have allocated enough room in working storage for
all tasks (including the printer task) to hold the image of any record
on which you use this technique (\prettyref{sub:Installing-The-Package}).

For instance, suppose we want to take a name from the input stream,
then search for it in the \texttt{\textbf{NICK}} (short for ``nickname'')
field of our \textbf{(}\texttt{\textbf{NAMES}}\textbf{)} file. The
phrase:

\texttt{\textbf{1 TEXT nickname NICK S!}}

captures the name from the input stream and stores it into the image
of the \texttt{\textbf{NICK}} field in working storage.

\nwenddocs{}\nwbegincode{165}\sublabel{NWtG4Lr-KJlJC-1}\nwmargintag{{\nwtagstyle{}\subpageref{NWtG4Lr-KJlJC-1}}}\moddef{text~{\nwtagstyle{}\subpageref{NWtG4Lr-KJlJC-1}}}\endmoddef\nwstartdeflinemarkup\nwusesondefline{\\{NWtG4Lr-2V5MiK-1}}\nwenddeflinemarkup
256 VALUE #TB       \\ #TB is the size of the text input buffer.

CREATE FILE-PAD   #TB ALLOT     \\ Make a PAD for our exclusive use

: >TEXT ( a n -- )   FILE-PAD #TB BLANK  FILE-PAD SWAP MOVE ;
: TEXT ( c -- )   WORD COUNT  >TEXT ;
\nwused{\\{NWtG4Lr-2V5MiK-1}}\nwendcode{}\nwbegindocs{166}\nwdocspar

Now we use \texttt{\textbf{BINARY}} to search the index file for this
name, first ensuring that the index file is current. \texttt{\textbf{BINARY}}
expects on the stack the arguments returned by a \texttt{\textbf{BYTES}}
field:

\texttt{\textbf{(NAMES) NICK BINARY}}

Here's what \texttt{\textbf{BINARY}} does:
\begin{description}
\item [{Word}] \textbf{Stack Action}
\item [{\texttt{BINARY}}] (\ n\ a\ -\ n) Searches the current file
looking for a match between the criteria in working storage and the
given field in the data. Issues a system abort if it cannot find the
record requested. On the stack is the record number of associated
record in the main field (that is, the contents of the link field
of the matching index record). The match\-ing index record number
is in \texttt{\textbf{R\#}}.
\end{description}
\nwenddocs{}\nwbegincode{167}\sublabel{NWtG4Lr-1rYrP5-1}\nwmargintag{{\nwtagstyle{}\subpageref{NWtG4Lr-1rYrP5-1}}}\moddef{binary~{\nwtagstyle{}\subpageref{NWtG4Lr-1rYrP5-1}}}\endmoddef\nwstartdeflinemarkup\nwusesondefline{\\{NWtG4Lr-2dg69x-1}}\nwenddeflinemarkup
: BINARY ( n a -- n )   -BINARY  ORDERED RELEASE  ABORT" Unknown"  LINK L@ ;
\nwused{\\{NWtG4Lr-2dg69x-1}}\nwendcode{}\nwbegindocs{168}\nwdocspar

(A related word, \texttt{\textbf{-BINARY}}, is discussed in \prettyref{par:Inserting-a-Record-in-an-Ordered-Index}.)

\textbf{REFERENCES}

Access to the record image in working storage, \prettyref{sub:Access-to-Working-Storage}\\
Binary search principles, \prettyref{sub:Ordered-Index-Files}\\
\texttt{\textbf{R\#}}, \prettyref{sub:Record-Selection}\\
\texttt{\textbf{TEXT}}, \prettyref{par:Searching-an-Ordered-Index}\\
Working Storage, \prettyref{sub:Working-Storage} 


\paragraph{Inserting a Record in an Ordered Index \label{par:Inserting-a-Record-in-an-Ordered-Index}}

Inserting a new record in an ordered file involves two steps. First,
we must determine the location in the index file for a new key to
be inserted. This ensures that the index file will always be properly
``sorted.''

Second, we must be able to insert the new key into the file at the
appro\-priate place, moving all subsequent records one notch down
in the file.

Using the example in Fig. \prettyref{fig:8}, let's consider what
must happen when we add a new person to our database. First, must
insert a new index record into the \texttt{\textbf{(NAMES)}} file
in the appropriate place, and then allocate a new record in the \texttt{\textbf{PEOPLE}}
file for the data itself. Finally, we must point the \texttt{\textbf{LINK}}
field in the index record to the data record in the main file.

We've already seen in \prettyref{sub:Record-Allocation/Deallocation-Operators}
that the word \texttt{\textbf{SLOT}} is used to allocate new records
in data files. Adding a record to the index file is more compli\-cated,
because we must insert the new record at the appropriate place to
keep the keys ordered. For this purpose, we use the words \texttt{\textbf{-BINARY}}
and \texttt{\textbf{+ORDERED}}.
\begin{description}
\item [{Word}] \textbf{Stack Action}
\item [{\texttt{-BINARY}}] (\ n\ a\ -\ t) Searches the current file
looking for a match between the criteria in working storage and the
given field in the data. A zero result (`false') means that a match
was found; a non-zero flag means that no record in the file contains
the indicated key. On exit, if a match is found \texttt{\textbf{R\#}}
contains the number of the first matching index file record; otherwise
\texttt{\textbf{R\#}} contains the number of the index record before
which an insertion will be made. Pronounced ``not-binary,'' because
it returns `true' if a match is not found.
\end{description}
\nwenddocs{}\nwbegincode{169}\sublabel{NWtG4Lr-1Wzvls-1}\nwmargintag{{\nwtagstyle{}\subpageref{NWtG4Lr-1Wzvls-1}}}\moddef{-binary~{\nwtagstyle{}\subpageref{NWtG4Lr-1Wzvls-1}}}\endmoddef\nwstartdeflinemarkup\nwusesondefline{\\{NWtG4Lr-2dg69x-1}}\nwenddeflinemarkup
: -BINARY ( n a -- f )    FILE-HANDLE @ 0= DUP >R IF  BIND-FILE  THEN
    SWAP  AVAILABLE 4 nC@ 2/ 1+ DUP READ  ORDERED GET  BEGIN
        DUP 1+ 2/  2OVER OVER ADDRESS  -TEXT 1- IF
            NEGATE
        THEN  R# +!  2/ DUP 0=
    UNTIL  DROP  2DUP OVER ADDRESS  -TEXT 0> ABS  R# +!
    OVER ADDRESS -TEXT  R> IF  -FILE  THEN ;
\nwused{\\{NWtG4Lr-2dg69x-1}}\nwendcode{}\nwbegindocs{170}\nwdocspar

\begin{description}
\item [{\texttt{+ORDERED}}] (\ -\ ) Inserts the record whose image is
in working storage into the current record in an ordered index. Subsequent
records in the index file are advanced one position relative to the
start of the file.
\end{description}
\nwenddocs{}\nwbegincode{171}\sublabel{NWtG4Lr-xEqwf-1}\nwmargintag{{\nwtagstyle{}\subpageref{NWtG4Lr-xEqwf-1}}}\moddef{+ordered~{\nwtagstyle{}\subpageref{NWtG4Lr-xEqwf-1}}}\endmoddef\nwstartdeflinemarkup\nwusesondefline{\\{NWtG4Lr-2dg69x-1}}\nwenddeflinemarkup
: RSWAP ( n a ra -- n a )    >R 2DUP R>  ROT 2/ 0 DO
        OVER 2 nC@  OVER 2 nC@  2OVER  >R 2 nC!  R> 2 nC!
        SWAP 2 +  SWAP 2 +
    LOOP  2DROP ;

: DIRECTION ( n -- n a rh rl )
    AVAILABLE 4 nC@ +
    AVAILABLE 4 nC! TOUCH  ENTIRE
    AVAILABLE 4 nC@ 2 +
    SAFE R# @ ;

: +ORDERED ( -- )
    FILE-HANDLE @ 0= DUP >R IF  BIND-FILE  THEN
    1 DIRECTION DO
        I RECORD  RSWAP TOUCH
    LOOP  2DROP  ORDERED RELEASE
    R> IF  -FILE  THEN ;
\nwused{\\{NWtG4Lr-2dg69x-1}}\nwendcode{}\nwbegindocs{172}\nwdocspar

\texttt{\textbf{-BINARY}} expects the same conditions as \texttt{\textbf{BINARY}}
(\prettyref{par:Searching-an-Ordered-Index}):
\begin{enumerate}
\item The current file is the ordered index to be searched.
\item The match criterion is in the key field in working storage.
\item The arguments produced by a \texttt{\textbf{BYTES}} field name are
on the stack.
\end{enumerate}
\texttt{\textbf{+ORDERED}} expects the following conditions:
\begin{enumerate}
\item The current file is the index to be modified.
\item The record before which the insertion is to take place has been previously
selected by\\
 \texttt{\textbf{-BINARY}}.
\item The key and \texttt{\textbf{LINK}} fields to be inserted are in their
respective fields in working storage.
\end{enumerate}
Using our example, then, the standard procedure is:

\texttt{\textbf{1 TEXT}} (scan the input stream for the name)

\texttt{\textbf{NICK S!}} (store it into the image of \texttt{\textbf{NICK}})

\texttt{\textbf{NICK -BINARY}} (search the index file, using the \texttt{\textbf{NICK}}
field as the key)

\texttt{\textbf{IF}} (no match:)

\texttt{\textbf{SAVE PEOPLE SLOT RESTORE}} ( obtain available record
number in main file)

\texttt{\textbf{LINK !}} (store the record number into working storage)

\texttt{\textbf{+ORDERED}} (insert the new index record)

\texttt{\textbf{ELSE}} (duplicate entry)

\texttt{\textbf{ORDERED RELEASE}}

\texttt{\textbf{1 ABORT\textquotedbl{} Already in file \textquotedbl{}}}

\texttt{\textbf{THEN ...}}

Because your code must provide the location into which the insertion
will take place (using \texttt{\textbf{-BINARY}}), you have the option
of determining how to handle duplicate keys if \texttt{\textbf{-BINARY}}
returns a false (zero) indication. This is normally handled as an
abort condition, as shown above.

During execution of the \texttt{\textbf{-BINARY\ ...\ +ORDERED}}
sequence, the index file should not be accessed by any other task,
since the record numbers of all records following the insertion point
are changing.

To prevent conflicts, the Data Base Support package includes a facility
management variable called \texttt{\textbf{ORDERED}}. \texttt{\textbf{-BINARY}}
issues an \texttt{\textbf{ORDERED\ GET}}. This phrase protects the
file from being accessed by other tasks on the system until the current
task releases it. In this way, file integrity is maintained. \texttt{\textbf{+ORDERED}}
issues an \texttt{\textbf{ORDERED\ RELEASE}}. If you exit from the
operation in any other way, you must do this yourself. The intent
is for the task that performed the search to retain control of the
file from the moment when the insertion point has been found until
the expected insertion has taken place, or until it has decided not
to do one.

The word \texttt{\textbf{BINARY}} also performs an \texttt{\textbf{ORDERED\ GET}},
so that searches cannot be performed while another task is using this
facility. \texttt{\textbf{BINARY}} performs an \texttt{\textbf{ORDERED\ RELEASE}}
immediately after the search, how\-ever, so it ``holds'' the facility
only during the period of the search itself.

\nwenddocs{}\nwbegincode{173}\sublabel{NWtG4Lr-4Y6sNo-1}\nwmargintag{{\nwtagstyle{}\subpageref{NWtG4Lr-4Y6sNo-1}}}\moddef{ordered~{\nwtagstyle{}\subpageref{NWtG4Lr-4Y6sNo-1}}}\endmoddef\nwstartdeflinemarkup\nwusesondefline{\\{NWtG4Lr-4ffLeu-1}}\nwenddeflinemarkup
VARIABLE ORDERED   0 ORDERED !
\nwused{\\{NWtG4Lr-4ffLeu-1}}\nwendcode{}\nwbegindocs{174}\nwdocspar

\textbf{REFERENCES}

Binary Searches, \prettyref{par:Searching-an-Ordered-Index}\\
\texttt{\textbf{SAVE}} and \texttt{\textbf{RESTORE}}, \prettyref{sub:Available-Records}\\
\texttt{\textbf{TEXT}}, \prettyref{par:Searching-an-Ordered-Index} 


\paragraph{Deleting a Record From an Ordered Index \label{par:Deleting-a-Record-From-an-Ordered-Index}}

\texttt{\textbf{-ORDERED}} is used to delete a record from an index
file. It may only be issued immediately after the record has been
selected (normally by a prior use of \texttt{\textbf{BINARY}}).
\begin{description}
\item [{Word}] \textbf{Stack Action}
\item [{\texttt{-ORDERED}}] (\ ) Deletes the current record (\texttt{\textbf{R\#}})
from an order\-ed index which is the current file. Subsequent records
move back one position, relative to the start of the file.
\end{description}
\nwenddocs{}\nwbegincode{175}\sublabel{NWtG4Lr-2HqFpk-1}\nwmargintag{{\nwtagstyle{}\subpageref{NWtG4Lr-2HqFpk-1}}}\moddef{-ordered~{\nwtagstyle{}\subpageref{NWtG4Lr-2HqFpk-1}}}\endmoddef\nwstartdeflinemarkup\nwusesondefline{\\{NWtG4Lr-2dg69x-1}}\nwenddeflinemarkup
: -ORDERED ( -- )
    FILE-HANDLE @ 0= DUP >R IF  BIND-FILE  THEN
    ENTIRE SWAP ERASE  -1 DIRECTION SWAP DO
        I RECORD RSWAP TOUCH
    -1 +LOOP  2DROP  ORDERED RELEASE
    R> IF  -FILE  THEN ;
\nwused{\\{NWtG4Lr-2dg69x-1}}\nwendcode{}\nwbegindocs{176}\nwdocspar

Because the record that until now followed it will occupy the actual
space that was occupied by the deleted record, the record is completely
obliterated by this operation (unlike \texttt{\textbf{SCRATCH}}, which
only changes the first two bytes of the record).

Here is an example using \texttt{\textbf{-ORDERED}}.

\texttt{\textbf{1 TEXT}} (scan the input stream for the name)

\texttt{\textbf{NICK S!}} (store it into the image of \texttt{\textbf{NICK}})

\texttt{\textbf{NICK BINARY}} (search the index file, using the \texttt{\textbf{NICK}}
field as the key; return main file record number)

\texttt{\textbf{ORDERED GRAB}} (regain control of \texttt{\textbf{ORDERED}},
which \texttt{\textbf{BINARY}} released)

\texttt{\textbf{-ORDERED}} (delete the index record)

\texttt{\textbf{PEOPLE SCRATCH}} (de-allocate the record in the main
file whose number is on the stack from \textbf{BINARY}.)

In this example we had to \texttt{\textbf{GRAB}} the facility variable
\texttt{\textbf{ORDERED}} to prevent another task from accessing the
file during the moving of records that will occur during the \texttt{\textbf{-ORDERED}}
operation. \texttt{\textbf{GRAB}} is used instead of \texttt{\textbf{GET}}
because \texttt{\textbf{GET}} releases the CPU so other tasks can
run (and potentially alter the file). \texttt{\textbf{-ORDERED}} performs
an \texttt{\textbf{ORDERED\ RELEASE}} when it is finished.

\textbf{REFERENCES}

Binary Searches, \prettyref{sub:Ordered-File-Maintenance}\\
\textbf{SCRATCH}, \prettyref{sub:Record-Allocation/Deallocation-Operators} 


\subsubsection{An Example---A Simple Mailing List \label{sub:Simple-Mailing-List}}

The following pages show an example of a simple mailing list application.
It demonstrates the use of an ordered index to provide easy access
into a file based on a key, such as last name and first initial, and
a report, which is in alphabetic order based on that key.

This application is a good example of the layout of a Data Base appli\-cation,
with a ``help screen'' at the top, followed by the relevant file
definitions. The help screen may be displayed any time by the command
\texttt{\textbf{PERSON-HELP}}.

\nwenddocs{}\nwbegincode{177}\sublabel{NWtG4Lr-1aUp39-1}\nwmargintag{{\nwtagstyle{}\subpageref{NWtG4Lr-1aUp39-1}}}\moddef{personnel~{\nwtagstyle{}\subpageref{NWtG4Lr-1aUp39-1}}}\endmoddef\nwstartdeflinemarkup\nwusesondefline{\\{NWtG4Lr-2V5MiK-1}}\nwenddeflinemarkup
: PERSON-HELP ( -- )
    CR ." PERSON-HELP   Display these PERSONNEL instructions."
    CR ." enter name    Enter a new person into the file with"
    CR ."                   access key of 'name'."
    CR
    CR ." remove name   Delete 'name' from the data base."
    CR
    CR ." fix name      Enter new information replacing all"
    CR ."                   current data for 'name'."
    CR
    CR ." view name     Display a person whose key is 'name'."
    CR
    CR ." v             Display current person."
    CR
    CR ." all           Display all records in the file."
    CR ;

FILE Personnel.dbf

: /Personnel ( -- ) \\ Runtime file setup
    Personnel.dbf  S" Personnel.dbf" >FILE
    FALSE REVERSE !  BIG-ENDIAN ;

( Bytes  records   origin           name   )
     16      300        0 BLOCK-DATA (PERSONNEL)
    128      300  +ORIGIN BLOCK-DATA  PERSONNEL 

\LA{}personnel-layout~{\nwtagstyle{}\subpageref{NWtG4Lr-3Rjkip-1}}\RA{}
\LA{}person~{\nwtagstyle{}\subpageref{NWtG4Lr-1dYgi5-1}}\RA{}
\LA{}digits~{\nwtagstyle{}\subpageref{NWtG4Lr-8Jj7a-1}}\RA{}
\LA{}!label~{\nwtagstyle{}\subpageref{NWtG4Lr-3Pw6lL-1}}\RA{}
\LA{}enter~{\nwtagstyle{}\subpageref{NWtG4Lr-2GxEXJ-1}}\RA{}
\LA{}fix~{\nwtagstyle{}\subpageref{NWtG4Lr-OHhPA-1}}\RA{}
\LA{}remove~{\nwtagstyle{}\subpageref{NWtG4Lr-3W96CK-1}}\RA{}
\LA{}.phone~{\nwtagstyle{}\subpageref{NWtG4Lr-27PjQm-1}}\RA{}
\LA{}.zip~{\nwtagstyle{}\subpageref{NWtG4Lr-QWrSM-1}}\RA{}
\LA{}.person~{\nwtagstyle{}\subpageref{NWtG4Lr-GdxRl-1}}\RA{}
\LA{}view~{\nwtagstyle{}\subpageref{NWtG4Lr-2uul1M-1}}\RA{}
\LA{}all~{\nwtagstyle{}\subpageref{NWtG4Lr-j6xjA-1}}\RA{}
\nwused{\\{NWtG4Lr-2V5MiK-1}}\nwendcode{}\nwbegindocs{178}\nwdocspar

The record layout for both the \texttt{\textbf{PERSONNEL}} and \texttt{\textbf{(PERSONNEL)}}
files. The \texttt{\textbf{LINK}} is predefined, and subsequent fields
are offset from the previous 4 fields. For example, the \texttt{\textbf{NICKNAME}}
name is 12 bytes long starting in the 5th byte. \texttt{\textbf{ZIP-CODE}}
is a 32-bit number, as is \texttt{\textbf{PHONE-NUMBER}}. \texttt{\textbf{AREA-CODE}}
is single precision.

The offset for the field types is carried on the stack so that it
may be either displayed or dropped at the end of the load. We use
it in this case to display the record size.

\nwenddocs{}\nwbegincode{179}\sublabel{NWtG4Lr-3Rjkip-1}\nwmargintag{{\nwtagstyle{}\subpageref{NWtG4Lr-3Rjkip-1}}}\moddef{personnel-layout~{\nwtagstyle{}\subpageref{NWtG4Lr-3Rjkip-1}}}\endmoddef\nwstartdeflinemarkup\nwusesondefline{\\{NWtG4Lr-1aUp39-1}}\nwenddeflinemarkup
  4 ( LINK)
  12 BYTES NICKNAME      ( Nickname, used as the key.)
  32 BYTES FULL-NAME     ( Full name, first name first.)
  32 BYTES STREET-ADDR   ( Street addr. or PO Box, etc.)
  32 BYTES CITY-STATE
     LONG ZIP-CODE       ( Note: can only handle US zips)
     NUMERIC AREA-CODE
     LONG PHONE-NUMBER
.( Personnel.dbf records = ) .  .( Bytes ) CR
\nwused{\\{NWtG4Lr-1aUp39-1}}\nwendcode{}\nwbegindocs{180}\nwdocspar

\begin{description}
\item [{\texttt{PERSON}}] parses the input stream following it for the
\texttt{\textbf{NICKNAME}} field. It leaves us pointing at the \texttt{\textbf{NICKNAME}}
field in the \texttt{\textbf{(PERSONNEL)}} file.
\end{description}
\nwenddocs{}\nwbegincode{181}\sublabel{NWtG4Lr-1dYgi5-1}\nwmargintag{{\nwtagstyle{}\subpageref{NWtG4Lr-1dYgi5-1}}}\moddef{person~{\nwtagstyle{}\subpageref{NWtG4Lr-1dYgi5-1}}}\endmoddef\nwstartdeflinemarkup\nwusesondefline{\\{NWtG4Lr-1aUp39-1}}\nwenddeflinemarkup
: PERSON ( - n a)   1 TEXT  NICKNAME S!  (PERSONNEL) NICKNAME ;
\nwused{\\{NWtG4Lr-1aUp39-1}}\nwendcode{}\nwbegindocs{182}\nwdocspar

\begin{description}
\item [{\texttt{DIGITS}}] Prompts the terminal for input and converts it
to binary on the stack.
\end{description}
\nwenddocs{}\nwbegincode{183}\sublabel{NWtG4Lr-8Jj7a-1}\nwmargintag{{\nwtagstyle{}\subpageref{NWtG4Lr-8Jj7a-1}}}\moddef{digits~{\nwtagstyle{}\subpageref{NWtG4Lr-8Jj7a-1}}}\endmoddef\nwstartdeflinemarkup\nwusesondefline{\\{NWtG4Lr-1aUp39-1}}\nwenddeflinemarkup
: DIGITS ( - n)   QUERY  32 WORD COUNT  0 0 2SWAP
    BEGIN  >NUMBER  DUP
    WHILE  1 /STRING
    REPEAT  2DROP DROP ;
\nwused{\\{NWtG4Lr-1aUp39-1}}\nwendcode{}\nwbegindocs{184}\nwdocspar

\begin{description}
\item [{\texttt{!LABEL}}] Prompts for each field in order.
\end{description}
\nwenddocs{}\nwbegincode{185}\sublabel{NWtG4Lr-3Pw6lL-1}\nwmargintag{{\nwtagstyle{}\subpageref{NWtG4Lr-3Pw6lL-1}}}\moddef{!label~{\nwtagstyle{}\subpageref{NWtG4Lr-3Pw6lL-1}}}\endmoddef\nwstartdeflinemarkup\nwusesondefline{\\{NWtG4Lr-1aUp39-1}}\nwenddeflinemarkup
: !LABEL   CR ." Name: "   FULL-NAME ASK
     CR   ." Street: "   STREET-ADDR ASK
     CR   ." City, State: "   CITY-STATE ASK
     CR   ." Zip: "   DIGITS ZIP-CODE L!
     CR   ." Area: "   DIGITS AREA-CODE N!
          ." Phone: "   DIGITS PHONE-NUMBER L! ;
\nwused{\\{NWtG4Lr-1aUp39-1}}\nwendcode{}\nwbegindocs{186}\nwdocspar

\begin{description}
\item [{\texttt{enter}}] creates a new entry for the person whose nickname
follows in the input stream, prompting for entry of additional data.
If there is already an entry for that nickname, an error message is
issued. In either case, the record remains the current one for future
editing.
\end{description}
\nwenddocs{}\nwbegincode{187}\sublabel{NWtG4Lr-2GxEXJ-1}\nwmargintag{{\nwtagstyle{}\subpageref{NWtG4Lr-2GxEXJ-1}}}\moddef{enter~{\nwtagstyle{}\subpageref{NWtG4Lr-2GxEXJ-1}}}\endmoddef\nwstartdeflinemarkup\nwusesondefline{\\{NWtG4Lr-1aUp39-1}}\nwenddeflinemarkup
: enter   PERSON -BINARY IF  SAVE  PERSONNEL SLOT DUP
        READ  NICKNAME S@ NICKNAME B!  RESTORE  DUP LINK !
        +ORDERED  PERSONNEL READ !LABEL
    ELSE  ORDERED RELEASE  1 ABORT" Already known "
    THEN ;
\nwused{\\{NWtG4Lr-1aUp39-1}}\nwendcode{}\nwbegindocs{188}\nwdocspar

\begin{description}
\item [{\texttt{fix}}] accepts new data for the pre-existing entry whose
nickname follows in the input stream.
\end{description}
\nwenddocs{}\nwbegincode{189}\sublabel{NWtG4Lr-OHhPA-1}\nwmargintag{{\nwtagstyle{}\subpageref{NWtG4Lr-OHhPA-1}}}\moddef{fix~{\nwtagstyle{}\subpageref{NWtG4Lr-OHhPA-1}}}\endmoddef\nwstartdeflinemarkup\nwusesondefline{\\{NWtG4Lr-1aUp39-1}}\nwenddeflinemarkup
: fix   PERSON BINARY  PERSONNEL READ !LABEL ;
\nwused{\\{NWtG4Lr-1aUp39-1}}\nwendcode{}\nwbegindocs{190}\nwdocspar

\begin{description}
\item [{\texttt{remove}}] deletes the person whose nickname follows from
the data base.
\end{description}
\nwenddocs{}\nwbegincode{191}\sublabel{NWtG4Lr-3W96CK-1}\nwmargintag{{\nwtagstyle{}\subpageref{NWtG4Lr-3W96CK-1}}}\moddef{remove~{\nwtagstyle{}\subpageref{NWtG4Lr-3W96CK-1}}}\endmoddef\nwstartdeflinemarkup\nwusesondefline{\\{NWtG4Lr-1aUp39-1}}\nwenddeflinemarkup
: remove   PERSON  BINARY -ORDERED  PERSONNEL SCRATCH ;
\nwused{\\{NWtG4Lr-1aUp39-1}}\nwendcode{}\nwbegindocs{192}\nwdocspar

\begin{description}
\item [{\texttt{.PHONE}}] displays the \texttt{\textbf{AREA-CODE}} and
\texttt{\textbf{PHONE-NUMBER}} as one would expect to see them.
\end{description}
\nwenddocs{}\nwbegincode{193}\sublabel{NWtG4Lr-27PjQm-1}\nwmargintag{{\nwtagstyle{}\subpageref{NWtG4Lr-27PjQm-1}}}\moddef{.phone~{\nwtagstyle{}\subpageref{NWtG4Lr-27PjQm-1}}}\endmoddef\nwstartdeflinemarkup\nwusesondefline{\\{NWtG4Lr-1aUp39-1}}\nwenddeflinemarkup
: .PHONE   AREA-CODE N@ 0  <# 41 HOLD # # #  40 HOLD #>
    TYPE SPACE  PHONE-NUMBER L@ 0 <# # # # #  45 HOLD ( -)
    # # # #> TYPE ;
\nwused{\\{NWtG4Lr-1aUp39-1}}\nwendcode{}\nwbegindocs{194}\nwdocspar

\begin{description}
\item [{\texttt{.ZIP}}] forces the \texttt{\textbf{ZIP-CODE}} to be displayed
in \texttt{\textbf{nnnnn}} format.
\end{description}
\nwenddocs{}\nwbegincode{195}\sublabel{NWtG4Lr-QWrSM-1}\nwmargintag{{\nwtagstyle{}\subpageref{NWtG4Lr-QWrSM-1}}}\moddef{.zip~{\nwtagstyle{}\subpageref{NWtG4Lr-QWrSM-1}}}\endmoddef\nwstartdeflinemarkup\nwusesondefline{\\{NWtG4Lr-1aUp39-1}}\nwenddeflinemarkup
: .ZIP   ZIP-CODE L@ 0 <# # # # # # #>  TYPE ;
\nwused{\\{NWtG4Lr-1aUp39-1}}\nwendcode{}\nwbegindocs{196}\nwdocspar

\begin{description}
\item [{\texttt{n~.PERSON}}] displays the data from the nth record in
the \texttt{\textbf{PERSONNEL}} data file.
\end{description}
\nwenddocs{}\nwbegincode{197}\sublabel{NWtG4Lr-GdxRl-1}\nwmargintag{{\nwtagstyle{}\subpageref{NWtG4Lr-GdxRl-1}}}\moddef{.person~{\nwtagstyle{}\subpageref{NWtG4Lr-GdxRl-1}}}\endmoddef\nwstartdeflinemarkup\nwusesondefline{\\{NWtG4Lr-1aUp39-1}}\nwenddeflinemarkup
: .PERSON ( n)   PERSONNEL READ  CR  FULL-NAME B?  5 SPACES
    ." (" SPACE   NICKNAME B?  ." )"  CR   STREET-ADDR B?
    CR  CITY-STATE B? CR   .ZIP  10 SPACES  .PHONE  SPACE ;
\nwused{\\{NWtG4Lr-1aUp39-1}}\nwendcode{}\nwbegindocs{198}\nwdocspar

\begin{description}
\item [{\texttt{view}}] Parses the input stream and displays the proper
record. \texttt{\textbf{s}} does the same thing using \texttt{\textbf{R\#}}
(the current record).
\end{description}
\nwenddocs{}\nwbegincode{199}\sublabel{NWtG4Lr-2uul1M-1}\nwmargintag{{\nwtagstyle{}\subpageref{NWtG4Lr-2uul1M-1}}}\moddef{view~{\nwtagstyle{}\subpageref{NWtG4Lr-2uul1M-1}}}\endmoddef\nwstartdeflinemarkup\nwusesondefline{\\{NWtG4Lr-1aUp39-1}}\nwenddeflinemarkup
: view   PERSON BINARY  .PERSON ;
: v   R# @ .PERSON ;
\nwused{\\{NWtG4Lr-1aUp39-1}}\nwendcode{}\nwbegindocs{200}\nwdocspar

\begin{description}
\item [{\texttt{all}}] uses the \texttt{\textbf{RECORDS}} word which returns
the initial value and number of records+1 in the data file. The loop
counter is used to access each record in the ordered index \texttt{\textbf{(PERSONNEL)}},
where the \texttt{\textbf{LINK}} field points to the data in the \texttt{\textbf{PERSONNEL}}
file.
\end{description}
\nwenddocs{}\nwbegincode{201}\sublabel{NWtG4Lr-j6xjA-1}\nwmargintag{{\nwtagstyle{}\subpageref{NWtG4Lr-j6xjA-1}}}\moddef{all~{\nwtagstyle{}\subpageref{NWtG4Lr-j6xjA-1}}}\endmoddef\nwstartdeflinemarkup\nwusesondefline{\\{NWtG4Lr-1aUp39-1}}\nwenddeflinemarkup
: all   (PERSONNEL) RECORDS DO
        I (PERSONNEL) READ  LINK L@ .PERSON  CR
    LOOP  SPACE ;
\nwused{\\{NWtG4Lr-1aUp39-1}}\nwendcode{}\nwbegindocs{202}\nwdocspar

Here is a sample of the output of \texttt{\textbf{all}}:

\begin{lstlisting}
Andrews, Carl      ( Carl )
1432 Morriston Ave. 
Parkerville, PA 
17214          (717) 555-9853 

Cook, Dottie      ( Dot )
154 Sweet Rd. 
Grand Prairie, TX 
75050          (214) 642-0011 

Chapel, Doug      ( Doug )
75 Fleetwood Dr. 
Rockville, MD 
20852          (301) 777-1259 

Boehning, Greg      ( Greg )
POB 41256 
Santa Cruz, CA 
95061          (408) 666-7891  
\end{lstlisting}



\subsubsection{Hierarchical Ordered Files \label{sub:Hierarchical-Ordered-Files}}

polyFORTH's ordered indexes have the property than whenever a record
is inserted or deleted all records following the point at which the
action occurs are physically moved to accommodate the change. Although
this form of maintenance is somewhat slower than maintaining order
by updating chains or pointers (as some data bases do) it is substantially
more reliable.

The assumption is that in most applications an index is searched frequent\-ly,
and insertions and deletions occur relatively infrequently. As a result,
we have optimized search time and reliability above maintenance time.

The actual time an insertion or deletion will take depends upon the
position in the file at which the action occurs (if it is near the
beginning of the file more records must be moved), the number of records
in the file, and the size of each index record. In practice, indexes
of several thousand records may be maintained on a hard disk without
unacceptable delays.

Some applications, however, involve tens of thousands of records that
must be searched and maintained in order. In order to deal with such
appli\-cations, the recommended approach is to divide the total index
into several sub-indexes, each of which will be a manageable size.
For example, a company with 40,000 employees might separate them into
departments. The department code can index a table in memory giving
the appropriate origin block number for the index of employees in
each department. This block number may be put into the \texttt{\textbf{ORG}}
field of the FDA of a private copy of a generic file definition for
the index. Or, the first letter of the employee's last name may be
used to select one of 26 indexes.

Such a multi-layered approach is called a \textit{hierarchy}. If you
are designing a hierarchical file structure, the important considerations
include keeping the decision-making process simple and independent
of any frequently changing conditions. If possible, try to base the
initial choice on something that can be evaluated without need for
a special file search. Above all, you should avoid keeping record
numbers of records in an ordered index in a higher-level index, as
ordered index record numbers are subject to change.

\textbf{REFERENCES}

File Definition Areas, \prettyref{sub:File-Definition-and-Access} 


\subsection{Chaining \label{sub:Chaining}}

Chaining is the linkage of one record to another, whether in the same
or a different file. Generally, chaining is appropriate when an unknown
amount of data must be associated with a piece of information.

There are as many ways to chain records as there are varieties of
applica\-tions. In this section, we'll cover most of the situations
that require chaining, and present general solutions to each case. 


\subsubsection{Chaining Techniques \label{sub:Chaining-Techniques}}

Before you begin coding, make sure that you study the exact requirements
carefully. Reviewing this section for considerations will be helpful.

Here are some design considerations to take into account:
\begin{enumerate}
\item Will the chaining occur within the same file, or to an auxiliary file?
\item Must there always be at least one auxiliary record chained to a main
record, or may a main record have no auxiliary records?
\item When you traverse the chain, should it be in the order in which its
elements were added (first-in, first-out), or in reverse (last-in,
first-out), or should the chain be maintained in order by a key (such
as date and time)?
\end{enumerate}
Let's explore these issues one by one.

In some applications, it is possible to chain records within a single
file. Naturally, this is easier than chaining to another file.

For example, suppose that we have a file of customer names and addresses.
Some of our customers have several addresses: one for invoicing, one
for shipping, and so on. Because multiple addresses are the exception,
not the rule, and because address fields are large, we'd prefer not
to allow room for multiple address fields within each customer record.

So, we use chaining instead. At this point, we must examine how much
information each auxiliary record must contain. It turns out that
each auxiliary record must contain almost as much information as the
main record. If we create a separate file for the auxiliary records,
each record would need to be nearly as large as a record in the main
file.

If there is relatively little in the main record (the one all customers
have) beyond the primary address, you may as well use additional records
in the same file to contain additional addresses. As Fig. \prettyref{fig:11}
shows, this approach lets us re-use the field layout structure that
we created for the main file records, even though there are some fields
in the primary record that we don't use in the auxiliary records.

For another example of chaining within a single file, we turn to the
\texttt{\textbf{Glossary}} application included with polyFORTH ISD-4
(see \prettyref{sub:Glossary-Utility}). This application lets you
enter descriptions of the commands in your applications and produces
alphabetized glossaries.

For each word that you enter into the system, the \texttt{\textbf{Glossary}}
saves its name, vocabulary, stack effects (before and after) as text
strings, the source block, the date this entry was created or updated,
plus as many lines of descriptive text as you care to include.

\begin{figure}
\caption{\label{fig:11}\protect\includegraphics{10_home_druffer_Dropbox_ColorForth_File_Docs_Chains.eps}}
\end{figure}


\begin{figure}
\caption{\label{fig:12}\protect\includegraphics{11_home_druffer_Dropbox_ColorForth_File_Docs_Glossary.eps}}
\end{figure}


Fig. \prettyref{fig:12} shows the record structure for the \texttt{\textbf{Glossary}}.
All data except the text is stored in the main record for each command.
This record points to an auxiliary record that contains the text description.
This record may in turn point to a second text record, and so on.
A separate index file contains the alphabetized keys that point to
main records in this file.

Although the main records and auxiliary records share \textit{no}
fields in common (except \texttt{\textbf{LINK}}), they are the same
size. Thus it is most efficient to keep both types of records in the
same file.

\begin{figure}
\caption{\label{fig:13}\protect\includegraphics{12_home_druffer_Dropbox_ColorForth_File_Docs_Serials.eps}}
\end{figure}


A third example illustrates the opposite situation. Suppose we have
a list of customers who have purchased our products. For each customer,
we also have a list of the serial numbers of the units they received.
For some customers, there are no serial numbers; for others, as many
as twenty.

You can see in Fig. \prettyref{fig:13} that a serial number record
takes much less space than a customer record. Because of this size
variance, it's better to create two separate files, one called \texttt{\textbf{CUSTOMERS}}
and the other \texttt{\textbf{SERIALS}}. Each main record in the \texttt{\textbf{CUSTOMERS}}
file may chain to one or a series of records in the \texttt{\textbf{SERIALS}}
file. A record in \texttt{\textbf{CUSTOMERS}} can also contain an
empty link, which would be represented by a value of -1 in the \texttt{\textbf{LINK}}
field. A -1 \texttt{\textbf{LINK}} also identifies the last serial
number for a particular customer.

This last example raises the second consideration: whether the application
must be able to handle the case of no auxiliary records, or whether
the minimum number of auxiliary records attached to a main record
must be one.

In the first case, when a main record is created, its link can be
left alone (-1) and no auxiliary record need be \texttt{\textbf{SLOT}}ted.
However, the routine that appends a new auxiliary record to the chain
must check whether it is linking from the main record or an auxiliary
record.

In the second case, when a main record is created, an auxiliary record
must also be slotted, and its number saved in the main record's pointer.
Further\-more, the routines for advancing through the chain will
differ, as we'll see in the next section.

A third consideration is whether chaining must be last-in, last-out;
last-in, first-out; or both. In the case of the \texttt{\textbf{Glossary}}
described earlier, obviously chaining must be first-in, first-out.
In such cases, the process of adding a new record to the chain involves:
\begin{enumerate}
\item Finding the end of the existing chain;
\item Allocating a new record;
\item Setting the link in the last record of the existing chain to point
to the new record.
\end{enumerate}
An example of the opposite situation is a bookkeeping database in
which each customer record chains to a series of auxiliary records
containing transactions. Because we are almost always more interested
in recent transactions than ancient ones, we chain in a last-in, first-out
manner. In this case, the process of adding a new record to the chain
involves:
\begin{enumerate}
\item Allocating a new auxiliary record;
\item Setting the main record to point to it;
\item Placing the main record's previous link number into the link field
of the new record.
\end{enumerate}
If the application demands that both directions of chain-following
be allowed, then each auxiliary record must contain two link fields:
one to the next record in the chain, and one to the previous.

Each chained record should contain a pointer back to the record that
is the head of the chain (which may or may not be in the same file
as the chain). Some applications use this directly. For instance,
suppose in our serial number example we keep an ordered index file
using the serial number itself as the key. If records in \texttt{\textbf{SERIALS}}
contain a pointer to the owner of the chain as shown in Fig. \prettyref{fig:13},
then by entering a serial number the user can see which customer has
received that instrument.

The most important reason for including a pointer to the owner, even
if the application doesn't otherwise demand it, is for ensuring integrity
of the data. If through some mischance of hardware failure a link
in the main file becomes lost, the chains can be reconstructed and
attached to the main records. 


\subsubsection{Chaining Commands \label{sub:Chaining-Commands}}

As we have seen in the previous section, the choice of chaining techniques
depends on application needs and on performance tradeoffs. Rather
than attempt to decide for you, the developers of polyFORTH ISD-4
provide a collection of commonly used chaining tools. You may leave
them as is, or you may modify them. The table below gives the general
set of commands in the chaining toolbox. Some words appear more than
once; this is because several implementations may be useful, depending
on how you've answered the design questions in \prettyref{sub:Chaining-Techniques}.
The version shipped with the system is marked with an ({*}). The others
are minor variants; code for some of the alternate versions is given
elsewhere in this chapter.
\begin{description}
\item [{Word}] \textbf{Stack Action}
\item [{\texttt{HEAD}}] (\ --\ a) A user variable that points to the
first record (head) of the current chain.
\end{description}
\nwenddocs{}\nwbegincode{203}\sublabel{NWtG4Lr-QPhyQ-1}\nwmargintag{{\nwtagstyle{}\subpageref{NWtG4Lr-QPhyQ-1}}}\moddef{head~{\nwtagstyle{}\subpageref{NWtG4Lr-QPhyQ-1}}}\endmoddef\nwstartdeflinemarkup\nwusesondefline{\\{NWtG4Lr-2dg69x-1}}\nwenddeflinemarkup
#USER CELL +USER HEAD TO #USER
\nwused{\\{NWtG4Lr-2dg69x-1}}\nwendcode{}\nwbegindocs{204}\nwdocspar

\begin{description}
\item [{\texttt{LINK}}] (\ --\ a) A pre-defined field (the first four
bytes of any record) which may be used for chaining. This same field
is used in ordered index records to link to the main file records.
\end{description}
\nwenddocs{}\nwbegincode{205}\sublabel{NWtG4Lr-3NK3Qp-1}\nwmargintag{{\nwtagstyle{}\subpageref{NWtG4Lr-3NK3Qp-1}}}\moddef{link~{\nwtagstyle{}\subpageref{NWtG4Lr-3NK3Qp-1}}}\endmoddef\nwstartdeflinemarkup\nwusesondefline{\\{NWtG4Lr-4ffLeu-1}}\nwenddeflinemarkup
0 LONG LINK DROP
\nwused{\\{NWtG4Lr-4ffLeu-1}}\nwendcode{}\nwbegindocs{206}\nwdocspar

\begin{description}
\item [{\texttt{FIRST}}] (\ ) ({*}) \texttt{\textbf{READ}}s the \texttt{\textbf{HEAD}}
record in the chain. This version is used in applications in which
there is always at least one auxiliary record and all are within the
same file.
\end{description}
\nwenddocs{}\nwbegincode{207}\sublabel{NWtG4Lr-2i9BK7-1}\nwmargintag{{\nwtagstyle{}\subpageref{NWtG4Lr-2i9BK7-1}}}\moddef{first~{\nwtagstyle{}\subpageref{NWtG4Lr-2i9BK7-1}}}\endmoddef\nwstartdeflinemarkup\nwusesondefline{\\{NWtG4Lr-2dg69x-1}}\nwenddeflinemarkup
: FIRST ( -- )   HEAD @ READ ;
\nwused{\\{NWtG4Lr-2dg69x-1}}\nwendcode{}\nwbegindocs{208}\nwdocspar

\begin{description}
\item [{\texttt{FIRST}}] (\ --\ t) Returns a flag indicating whether
the main record is chained to any auxiliary records, and if it is,
\texttt{\textbf{READ}}s the record. This version is used in applications
in which the \texttt{\textbf{HEAD}} record may have no auxiliary records,
and when auxiliary records are in a different file.
\item [{\texttt{-NEXT}}] (\ --\ t) ({*}) Reads the next record, assuming
that the chain is linked through the field called \texttt{\textbf{LINK}}.
Returns `true' if there is \textit{not} a next record in the chain.
Pronounced ``not-next.''
\end{description}
\nwenddocs{}\nwbegincode{209}\sublabel{NWtG4Lr-49QaAb-1}\nwmargintag{{\nwtagstyle{}\subpageref{NWtG4Lr-49QaAb-1}}}\moddef{-next~{\nwtagstyle{}\subpageref{NWtG4Lr-49QaAb-1}}}\endmoddef\nwstartdeflinemarkup\nwusesondefline{\\{NWtG4Lr-2dg69x-1}}\nwenddeflinemarkup
: -NEXT ( -- t )   LINK L@  DUP 0> IF  READ 1  THEN  1- ;
\nwused{\\{NWtG4Lr-2dg69x-1}}\nwendcode{}\nwbegindocs{210}\nwdocspar

\begin{description}
\item [{\texttt{-NEXT}}] (\ --\ r/0) Alternate version of \texttt{\textbf{-NEXT}};
returns the record number of the next record in the chain, if any,
0 (`false') otherwise. Does not read the record.
\item [{\texttt{-LOCATE}}] (\ n\ -\ t) Searches the chain, starting
from \texttt{\textbf{HEAD}}, for the nth record, returning true if
the chain isn't that long. Otherwise, it returns false, having left
\texttt{\textbf{R\#}} pointing to the specified record.
\end{description}
\nwenddocs{}\nwbegincode{211}\sublabel{NWtG4Lr-3ikf9S-1}\nwmargintag{{\nwtagstyle{}\subpageref{NWtG4Lr-3ikf9S-1}}}\moddef{-locate~{\nwtagstyle{}\subpageref{NWtG4Lr-3ikf9S-1}}}\endmoddef\nwstartdeflinemarkup\nwusesondefline{\\{NWtG4Lr-2dg69x-1}}\nwenddeflinemarkup
: -LOCATE ( n -- r t )
    1+ FIRST  BEGIN
        1- DUP WHILE
        -NEXT IF  EXIT
    THEN  REPEAT ;
\nwused{\\{NWtG4Lr-2dg69x-1}}\nwendcode{}\nwbegindocs{212}\nwdocspar

\begin{description}
\item [{\texttt{CHAIN}}] (\ n) Inserts a new record at the nth position.
If \textit{n} is larger than the length of the chain, inserts the
new record at the end. Alternate versions might take no argument and
chain at the beginning (last-in, first-out), end (first-in, first-out)
or according to a key.
\end{description}
\nwenddocs{}\nwbegincode{213}\sublabel{NWtG4Lr-4AREak-1}\nwmargintag{{\nwtagstyle{}\subpageref{NWtG4Lr-4AREak-1}}}\moddef{chain~{\nwtagstyle{}\subpageref{NWtG4Lr-4AREak-1}}}\endmoddef\nwstartdeflinemarkup\nwusesondefline{\\{NWtG4Lr-2dg69x-1}}\nwenddeflinemarkup
: CHAIN ( n -- )    -LOCATE DROP  ( nth record or end )
    SLOT LINK OVER  SNATCH  SWAP READ  LINK L! ;
\nwused{\\{NWtG4Lr-2dg69x-1}}\nwendcode{}\nwbegindocs{214}\nwdocspar

\begin{description}
\item [{\texttt{UNCHAIN}}] (\ n) Removes the nth record from the chain.
\end{description}
\nwenddocs{}\nwbegincode{215}\sublabel{NWtG4Lr-3m1wNz-1}\nwmargintag{{\nwtagstyle{}\subpageref{NWtG4Lr-3m1wNz-1}}}\moddef{unchain~{\nwtagstyle{}\subpageref{NWtG4Lr-3m1wNz-1}}}\endmoddef\nwstartdeflinemarkup\nwusesondefline{\\{NWtG4Lr-2dg69x-1}}\nwenddeflinemarkup
: UNCHAIN ( n -- )    DUP 0= ABORT" Won't"  -LOCATE ABORT" Not found"
    SAVE  LINK L@ READ  LINK 0 SNATCH  RESTORE  LINK L! ;
\nwused{\\{NWtG4Lr-2dg69x-1}}\nwendcode{}\nwbegindocs{216}\nwdocspar

\begin{description}
\item [{\texttt{SNATCH}}] (\ a\ r\ -\ r) Given a field address and
record number, fetches the record number from that field and replaces
it with the record number given. It is used to update chains.
\end{description}
\nwenddocs{}\nwbegincode{217}\sublabel{NWtG4Lr-4GTjjD-1}\nwmargintag{{\nwtagstyle{}\subpageref{NWtG4Lr-4GTjjD-1}}}\moddef{snatch~{\nwtagstyle{}\subpageref{NWtG4Lr-4GTjjD-1}}}\endmoddef\nwstartdeflinemarkup\nwusesondefline{\\{NWtG4Lr-2dg69x-1}}\nwenddeflinemarkup
: SNATCH ( a r -- r )   OVER L@  SWAP ROT L! ;
\nwused{\\{NWtG4Lr-2dg69x-1}}\nwendcode{}\nwbegindocs{218}\nwdocspar

The arguments for \texttt{\textbf{-LOCATE}}, \texttt{\textbf{CHAIN}},
and \texttt{\textbf{UNCHAIN}} count from zero, where zero is the first
record in the chain, and count sequentially down the chain. An argument
of -1 is conventionally used to specify the end of the chain (since
you don't necessarily know how long the chain is).

The standard versions of \texttt{\textbf{FIRST}} and \texttt{\textbf{-NEXT}}
assume there is always at least one record in the chain, and it's
also the \texttt{\textbf{HEAD}} record (i.e., it will be subject to
the same processing as the others). The chain may be processed in
a \texttt{\textbf{BEGIN\ ...\ UNTIL}} loop:

\texttt{\textbf{FIRST BEGIN ... -NEXT UNTIL ...}}

The alternate versions allow for the possibility that there are no
auxiliary chains, and are optimized for a \textbf{BEGIN}\ \textbf{...}\ \textbf{WHILE}\ \textbf{...}\ \textbf{REPEAT}
loop:

\texttt{\textbf{FIRST BEGIN ?DUP WHILE READ ...}}~\\
\texttt{\textbf{ NEXT REPEAT ...}}

If you have only one set of chained records and the top of the chain
is in a different file from the members, you may incorporate the selection
of the file in the words \texttt{\textbf{FIRST}}, etc. If you have
several sets, you will need to select the file externally. Still another
set of variations might allow for the fact that you have more than
one chain attached to your main file, and therefore not all chains
start with the \texttt{\textbf{LINK}} in the main file record. Moreover,
there may even be multiple chains through the auxiliary records. In
these cases, you would remove the references to \texttt{\textbf{LINK}}
in these words and specify the field externally.

The intent here is to present a design concept that has worked in
many applications, but which presumes that you will tailor a basic
vocabulary to your specific application needs---a practice that is
consistent with the overall design of Forth in general. Assuming you
are adding custom versions of the chaining words for your application,
don't forget to remove from the \texttt{\textbf{FILES}} load block
the reference to the standard ones.

\textbf{REFERENCES}

\textbf{FILES} Load Block, \prettyref{sub:Installing-The-Package} 


\subsubsection{Application Examples \label{sub:Application-Examples}}

This section offers coded solutions to two application problems.

We introduced the \texttt{\textbf{Glossary}} program, which is included
with your polyFORTH system, in \prettyref{sub:Chaining-Techniques}.
The use of this utility is more thoroughly documented in \prettyref{sub:Glossary-Utility}.

The word \texttt{\textbf{(SHOW)}} includes this sequence:

\texttt{\textbf{... ( display data from the main record)}}~\\
\texttt{\textbf{ BEGIN +L -NEXT 0= WHILE}}~\\
\texttt{\textbf{ 10 SPACES PHRASE B?}}~\\
\texttt{\textbf{ REPEAT ;}}

The word \texttt{\textbf{+L}} is similar to \texttt{\textbf{CR}};
see \prettyref{sub:The-Page-Banner}.

The word \texttt{\textbf{(SHOW)}} displays all information about a
command. The code fragment shown above displays the list of description
lines for the com\-mand. When it begins, the main record is still
current.

As we saw in \prettyref{sub:Chaining-Techniques}, the main record's
link field points to the first descriptive record, if there is one,
which resides in the same file. When the loop begins, \texttt{\textbf{-NEXT}}
determines whether the main record is linked to an auxiliary record.
If not, the loop ends and nothing is displayed. If so, the \texttt{\textbf{WHILE}}
portion is executed, which displays the first line of text and repeats
the loop. Now \texttt{\textbf{-NEXT}} indicates whether there is another
auxili\-ary record.

When the last record is reached, \texttt{\textbf{-NEXT}} indicates
this and the loop ends.

The word \texttt{\textbf{?LINES}} is defined as:

\texttt{\textbf{: ?LINES 1 BEGIN 1+ -NEXT UNTIL ?PAGE ;}}

The purpose of \texttt{\textbf{?LINES}} is to determine whether the
current command's description will fit entirely on the page, or whether
it is necessary to advance the page first to keep all of its lines
together. The loop counts the number of lines (the head plus an unknown
number of auxiliary records, at one line each). The word \texttt{\textbf{?PAGE}},
introduced in \prettyref{sub:The-Page-Banner}, takes an argument
from the stack, starting a new page if that many lines will not fit
on the current page.

Here is a definition using \texttt{\textbf{SNATCH}}:

\texttt{\textbf{: DELETE ( r\#) ... BEGIN READ}}~\\
\texttt{\textbf{ LINK 0 SNATCH DUP 0< UNTIL DROP ;}}

The part of \texttt{\textbf{DELETE}} shown here removes both the main
record and all auxiliary records chained to it. The code begins on
the main record. The phrase \texttt{\textbf{LINK\ 0\ SNATCH}} fetches
the record's link field, and then replaces it with zero. This has
the effect of ``scratching'' the record, but also provides a pointer
to the next record to scratch.

The phrase \texttt{\textbf{DUP\ 0<}} tests whether the pointer indicates
that the record just scratched was the last in the chain. If so, the
loop ends; otherwise, it reads the next record, and so on.

You may also wish to study the definitions of \texttt{\textbf{T}},
\texttt{\textbf{P}}, and \texttt{\textbf{U}}, which use \texttt{\textbf{-LOCATE}},
\texttt{\textbf{CHAIN}}, and \texttt{\textbf{UNCHAIN}} in straightforward
ways.

Our second coding example is another that we introduced earlier in
this section: the customer file and associated serial numbers. Here
we will present two versions of the application. The first, in the
\texttt{\textbf{Customers}} example, uses the versions of \texttt{\textbf{FIRST}},
\texttt{\textbf{-NEXT}}, and \texttt{\textbf{-LOCATE}} that are provided
with your polyFORTH system.

In the first block we've defined the record structures for the two
files. In the second block, we have words for entering new customers
and serial num\-bers. The word \texttt{\textbf{add}} makes use of
chaining.

As we saw in our earlier discussion of this application, it is legitimate
for a \texttt{\textbf{CUSTOMERS}} record to have no serial number
attached to it. In this case, the \texttt{\textbf{CUSTOMERS}} record
will contain --1 in its \texttt{\textbf{LINK}} field. If auxiliary
records are chained, they will reside in a separate file called \texttt{\textbf{SERIALS}}.

The process of adding a new serial-number record is not as simple
as it would be if all records were contained in the same file. Here,
add must make a decision. If there is no chaining yet, it must go
to the \texttt{\textbf{SERIALS}} file and use \texttt{\textbf{SLOT}}
to allocate a record.

Since this is the first record in the chain, it must also store this
in the main record's \texttt{\textbf{LINK}} field. But if a chain
has already been started, it will go to \texttt{\textbf{SERIALS}}
and use \texttt{\textbf{CHAIN}} to add a new record.

The problem is that we cannot use \texttt{\textbf{CHAIN}} unless a
chain exists already. If all records existed in the same file, then
the main record would be the first record in the chain; and we could
simply use \texttt{\textbf{CHAIN}} in all cases. We would not need
a conditional. Or, even if records existed in separate files, but
a minimum of one auxiliary record was always present, we could use
\texttt{\textbf{CHAIN}} and avoid the conditional.

The phrase \texttt{\textbf{-1\ CHAIN}} is a clich� that means ``attach
a new record onto the end of the chain.'' The -1 serves as a number
that never gets reached, and \texttt{\textbf{CHAIN}} is defined so
that if it never reaches n it adds the new record to the end of the
chain.

An application example using the standard polyFORTH chaining operators.

\nwenddocs{}\nwbegincode{219}\sublabel{NWtG4Lr-i0vIO-1}\nwmargintag{{\nwtagstyle{}\subpageref{NWtG4Lr-i0vIO-1}}}\moddef{customers~{\nwtagstyle{}\subpageref{NWtG4Lr-i0vIO-1}}}\endmoddef\nwstartdeflinemarkup\nwusesondefline{\\{NWtG4Lr-2V5MiK-1}}\nwenddeflinemarkup
FILE Customers.dbf

: /Customers ( -- ) \\ Runtime file setup
    Customers.dbf  S" Customers.dbf" >FILE
    TRUE REVERSE !  LITTLE-ENDIAN ;

70 500       0 BLOCK-DATA CUSTOMERS
46 500 +ORIGIN BLOCK-DATA SERIALS

( CUSTOMERS records:)
0  4 FILLER LINK  ( LINK to 1st serial#)
   20 BYTES COMPANY
   16 BYTES CONTACT
   30 BYTES ADDR
DROP

( SERIALS records:)
0  4 FILLER LINK  ( LINK to next serial#)
   10 BYTES SERIAL#
    NUMERIC PRODUCT  ( product code)
    NUMERIC OWNER  ( link to owner CUSTOMERS record)
DROP

: edit-company   CR  ." Company name? "  COMPANY ASK
    CR  ." Contact? "  CONTACT ASK
    CR  ." Address? "  ADDR ASK ;
: new-company   CUSTOMERS  SLOT DUP . READ  edit-company ;

VARIABLE ALT  TRUE ALT ! \\ Another approach

\LA{}alt-locate~{\nwtagstyle{}\subpageref{NWtG4Lr-4dMbP0-1}}\RA{}

: (add)   CR  ." Serial# ? "   SERIAL# ASK ;

: add   ALT @ IF  \LA{}alt-add~{\nwtagstyle{}\subpageref{NWtG4Lr-1M2tOz-1}}\RA{}
    ELSE  SAVE  LINK L@ DUP 0< IF  ( empty chain) DROP
            SAVE  SERIALS SLOT  RESTORE
            DUP LINK L!  SERIALS READ
        ELSE  HEAD !  SERIALS -1 CHAIN ( add at end)
        THEN (add)  RESTORE
    THEN ;

: edit-serial ( n)   SAVE SERIALS  1- -LOCATE ABORT" Can't"
    CR  Serial# B?  (add)  RESTORE ;

: .company   COMPANY B?  CONTACT B?  ADDR B? ;

: .companies   CUSTOMERS  RECORDS ?DO
    CR  I .  I READ  .company  LOOP ;

: .serials   0  SERIALS  FIRST BEGIN
    CR 1+ DUP .  Serial# B?  -NEXT UNTIL  DROP ;
: all-serials   ALT @ IF  \LA{}alt-serials~{\nwtagstyle{}\subpageref{NWtG4Lr-2TgUOO-1}}\RA{}
    ELSE  LINK L@  0>  IF
            LINK L@  HEAD !  SAVE  .serials  RESTORE
    THEN  THEN ;

: show ( n)   CUSTOMERS READ  CR .company  all-serials ;
\nwused{\\{NWtG4Lr-2V5MiK-1}}\nwendcode{}\nwbegindocs{220}\nwdocspar

The word \texttt{\textbf{edit-serial}}\textbf{ }may be \texttt{\textbf{used}}
to change an existing serial number. From its purpose we can assume
that a chain exists, and therefore it doesn't have to check the main
record's \texttt{\textbf{LINK}} to make sure it points to a valid
auxiliary record. It simply goes to the \texttt{\textbf{SERIALS}}
file and uses \texttt{\textbf{-LOCATE}} to make the desired record
current (aborting if the argument is not valid and \texttt{\textbf{-LOCATE}}
terminates before reaching it). Then it displays the current contents
of the field and lets the user re-enter it. Finally it restores the
file pointers to the main file.

In the next block, the word \texttt{\textbf{show}} displays the given
company data, followed by a list of all associated serial numbers.
Again, since there may be no chain at all, \texttt{\textbf{all-serials}}
must make a decision. The test \texttt{\textbf{LINK\ L@\ 0>}} returns
`true' if the link is positive (that is, not -1 or 0), indicating
the first record in the chain. In this event, \texttt{\textbf{all-serials}}
saves this link in the variable \texttt{\textbf{HEAD}}, selects the
\texttt{\textbf{SERIALS}} file, and invokes \texttt{\textbf{.serials}}
which uses \texttt{\textbf{FIRST}} and \texttt{\textbf{-NEXT}}\textbf{
}to loop through all records in the chain.

To give you an idea of some of the many possibilities, we've coded
the same application using different versions of the words \texttt{\textbf{FIRST}},
\texttt{\textbf{NEXT}}, and \textbf{-LOCATE}. While these definitions
themselves are more complicated, they reduce the complexity of the
application words that use them. These versions are sensitive to the
possibility that a main record may not have any auxiliary records
attached to it.

Here are the re-definitions, followed by the new versions of the affected
application commands:

\nwenddocs{}\nwbegincode{221}\sublabel{NWtG4Lr-4dMbP0-1}\nwmargintag{{\nwtagstyle{}\subpageref{NWtG4Lr-4dMbP0-1}}}\moddef{alt-locate~{\nwtagstyle{}\subpageref{NWtG4Lr-4dMbP0-1}}}\endmoddef\nwstartdeflinemarkup\nwusesondefline{\\{NWtG4Lr-i0vIO-1}}\nwenddeflinemarkup
: VALID ( n - t)   0 OVER < DUP  IF
    SWAP READ  ELSE  SWAP DROP  THEN ;
: ALT-FIRST ( - t)   HEAD @ VALID ;
: ALT-NEXT ( - t)   LINK L@ VALID ;
: ALT-LOCATE ( n - t)   ALT-FIRST IF BEGIN  DUP WHILE
        1-  ALT-NEXT 0= IF  DROP -1 EXIT  THEN
    REPEAT  ELSE DROP -1 THEN ;
\nwused{\\{NWtG4Lr-i0vIO-1}}\nwendcode{}\nwbegindocs{222}\nwdocspar
\nwenddocs{}\nwbegincode{223}\sublabel{NWtG4Lr-1M2tOz-1}\nwmargintag{{\nwtagstyle{}\subpageref{NWtG4Lr-1M2tOz-1}}}\moddef{alt-add~{\nwtagstyle{}\subpageref{NWtG4Lr-1M2tOz-1}}}\endmoddef\nwstartdeflinemarkup\nwusesondefline{\\{NWtG4Lr-i0vIO-1}}\nwenddeflinemarkup
LINK L@ HEAD !  SAVE  SERIALS
    ALT-FIRST IF  -1 CHAIN  ELSE ( no chain)
        SLOT DUP RESTORE  LINK L!  SAVE
        SERIALS READ  THEN
    (add)  RESTORE
\nwused{\\{NWtG4Lr-i0vIO-1}}\nwendcode{}\nwbegindocs{224}\nwdocspar
\nwenddocs{}\nwbegincode{225}\sublabel{NWtG4Lr-2TgUOO-1}\nwmargintag{{\nwtagstyle{}\subpageref{NWtG4Lr-2TgUOO-1}}}\moddef{alt-serials~{\nwtagstyle{}\subpageref{NWtG4Lr-2TgUOO-1}}}\endmoddef\nwstartdeflinemarkup\nwusesondefline{\\{NWtG4Lr-i0vIO-1}}\nwenddeflinemarkup
LINK L@ HEAD !  SAVE  0
    SERIALS  ALT-FIRST BEGIN  WHILE  CR  1+ DUP .
        SERIAL# B?  ALT-NEXT REPEAT  RESTORE  DROP
\nwused{\\{NWtG4Lr-i0vIO-1}}\nwendcode{}\nwbegindocs{226}\nwdocspar

In the first block, \texttt{\textbf{ALT-FIRST}} returns a flag that
is true if a chain exists at all. If so, the first record in the chain
is made current. The word \texttt{\textbf{ALT-NEXT}} returns a flag
that is true if another record exists in the chain. If so, that record
is made current.

As you can see, both words make use of the same code, which we have
factored into the definition called \texttt{\textbf{VALID}}.

We have also re-coded \texttt{\textbf{ALT-LOCATE}} in this block.
As usual, \texttt{\textbf{ALT-LOCATE}} returns a ``true'' flag if
the requested element of the chain cannot be found. In this version,
it also returns a ``true'' flag if no chain exists.

These changes simplify our application definitions. \texttt{\textbf{add}}
still has to make a decision, but it uses \texttt{\textbf{ALT-FIRST}}
for the test.

Because of the way we have rewritten \texttt{\textbf{ALT-FIRST}},
\texttt{\textbf{all-serials}} no longer needs an \texttt{\textbf{IF}}
statement at all. The only conditional is \texttt{\textbf{WHILE}},
which gets its argument the first time around from \texttt{\textbf{ALT-FIRST}},
and henceforth from \texttt{\textbf{ALT-NEXT}}. Thus, if a first record
is absent, the \texttt{\textbf{WHILE}} phrase never gets executed.
We eliminated the need for a subordinate word \texttt{\textbf{.serials}}
completely.

\textbf{REFERENCES}

Data Base Design, \prettyref{sub:Data-Base-Design} 


\subsection{Report Generator \label{sub:Report-Generator}}

The polyFORTH Report Generator is a set of words that assist you in
the preparation of formatted output reports. Once you have specified
the page format and column headings, and indicated the layout of a
single record as a row of data, the Report Generator performs all
required output formatting and also controls paging, the heading of
each page and related operations.

An optional feature of the Report Generator allows subtotals and grand
totals to be accumulated in a simple manner; these totals can then
be printed on a separate line with a minimum of effort.

\nwenddocs{}\nwbegincode{227}\sublabel{NWtG4Lr-1HFlBt-1}\nwmargintag{{\nwtagstyle{}\subpageref{NWtG4Lr-1HFlBt-1}}}\moddef{reports~{\nwtagstyle{}\subpageref{NWtG4Lr-1HFlBt-1}}}\endmoddef\nwstartdeflinemarkup\nwusesondefline{\\{NWtG4Lr-2V5MiK-1}}\nwenddeflinemarkup
\LA{}cols~{\nwtagstyle{}\subpageref{NWtG4Lr-3BC2yN-1}}\RA{}
\LA{}page~{\nwtagstyle{}\subpageref{NWtG4Lr-4GHBLz-1}}\RA{}
\LA{}register~{\nwtagstyle{}\subpageref{NWtG4Lr-8Tw6T-1}}\RA{}
\LA{}totals~{\nwtagstyle{}\subpageref{NWtG4Lr-3QzPba-1}}\RA{}
\LA{}sum~{\nwtagstyle{}\subpageref{NWtG4Lr-3qrQOy-1}}\RA{}
\LA{}foot~{\nwtagstyle{}\subpageref{NWtG4Lr-W27Dj-1}}\RA{}
\LA{}sub~{\nwtagstyle{}\subpageref{NWtG4Lr-2JLP8G-1}}\RA{}
\LA{}total~{\nwtagstyle{}\subpageref{NWtG4Lr-1G4TAa-1}}\RA{}
\LA{}grand~{\nwtagstyle{}\subpageref{NWtG4Lr-1VMgcH-1}}\RA{}
\LA{}heading~{\nwtagstyle{}\subpageref{NWtG4Lr-14go8R-1}}\RA{}
\LA{}title~{\nwtagstyle{}\subpageref{NWtG4Lr-3B2QOB-1}}\RA{}
\LA{}title-heading~{\nwtagstyle{}\subpageref{NWtG4Lr-3WJ4qF-1}}\RA{}
\LA{}rlc~{\nwtagstyle{}\subpageref{NWtG4Lr-2YnRrI-1}}\RA{}
\LA{}layout~{\nwtagstyle{}\subpageref{NWtG4Lr-4JwrrS-1}}\RA{}
\LA{}output~{\nwtagstyle{}\subpageref{NWtG4Lr-2ruhRa-1}}\RA{}
\nwused{\\{NWtG4Lr-2V5MiK-1}}\nwendcode{}\nwbegindocs{228}\nwdocspar

The following example will serve as a quick introduction to the Report
Generator. It assumes the fields defined in the example in \prettyref{sub:Overview}.
Here is the code:

\nwenddocs{}\nwbegincode{229}\sublabel{NWtG4Lr-36NJG5-1}\nwmargintag{{\nwtagstyle{}\subpageref{NWtG4Lr-36NJG5-1}}}\moddef{all-people~{\nwtagstyle{}\subpageref{NWtG4Lr-36NJG5-1}}}\endmoddef\nwstartdeflinemarkup\nwusesondefline{\\{NWtG4Lr-3Y3Oqi-1}}\nwenddeflinemarkup
: display.person   NAME ?B  STREET ?B  CITY ?B  STATE ?B  ZIP ?B ;

[R                 People\\Name          \\Address            \\City         \\St.\\Zip ]
    CONSTANT PEOPLE-TITLE

: all-people   PEOPLE-TITLE LAYOUT  +L
    PEOPLE RECORDS ?DO  I READ  display.person  +L  LOOP ;
\nwused{\\{NWtG4Lr-3Y3Oqi-1}}\nwendcode{}\nwbegindocs{230}\nwdocspar

This produces:

\begin{lstlisting}
Page 1  07/30/2012
                People 
Name           Address             City          St. Zip 

Andrews, Carl  1432 Morriston Ave. Parkerville   PA  17214 
Boehning, Greg POB 41256           Santa Cruz    CA  95061 
Chapel, Doug   75 Fleetwood Dr.    Rockville     MD  20852 
Cook, Dottie   154 Sweet Rd.       Grand Prairie TX  75050 
\end{lstlisting}


In the example above, the word \texttt{\textbf{display.person}} is
defined similarly to the version given in \prettyref{sub:Overview},
except that the field reference operator \texttt{\textbf{?B}} is used
instead of \texttt{\textbf{B?}}. \texttt{\textbf{?B}} is the Report
Generator version of \texttt{\textbf{B?}}, and takes the same stack
arguments. The difference is that it performs ``tabbing'' based
on a table of columns created by the word \texttt{\textbf{{[}R}} (third
line of example). \prettyref{sub:Controlling-Paging} lists all the
output operators that use this table.

The word \texttt{\textbf{{[}R}} specifies both a title (the word ``People,''
centered) and the column headings (the row of labels above each column).
It also creates the column table mentioned above, leaving this address
on the stack. Note that the line is shown wrapping to the next line
here, but that it must be on the same line in the source code. See
\prettyref{sub:Formatting-Lines} for more on \texttt{\textbf{{[}R}}.

The final word, \texttt{\textbf{all-people}} prints the tabulated
report. It begins by invoking \texttt{\textbf{PEOPLE-TITLE}} so that
the address passed from \texttt{\textbf{{[}R}} will become part of
this definition, then calls \texttt{\textbf{LAYOUT}}, which consumes
this address.

\texttt{\textbf{+L}} (short for ``plus-line'') forces an extra carriage
return into the report above the first row of data. Next, \texttt{\textbf{PEOPLE}}
guarantees that the \texttt{\textbf{PEOPLE}} file is current whenever
we display this report. \texttt{\textbf{RECORDS}} supplies the appropriate
arguments for \texttt{\textbf{DO}}. Each time through the loop, the
next record is made current with \texttt{\textbf{READ}}, and the row
is displayed with \texttt{\textbf{all-people}}. Then, \texttt{\textbf{+L}}
forces a new-line.

The report also contains a page banner, which includes some text at
the upper-left hand corner of the page and the page number and date
in the upper right. These are formatted automatically by \texttt{\textbf{LAYOUT}},
but are user-configurable.

\textbf{REFERENCES}

Controlling Paging, \prettyref{sub:Controlling-Paging}\\
Page Banner, \prettyref{sub:The-Page-Banner} 


\subsubsection{Specifying a Title/Column-Heading Pair \label{sub:Specifying-a-Title/Column-Heading-Pair}}

A single word, \texttt{\textbf{{[}R}}, lets you specify both the title
and the column headings. The set-up phrase usually appears just preceding
the definition of the report for which they are designed.

The format for a title/column heading pair is:

\texttt{\textbf{{[}R title-text \textbackslash{}column-heading-text
{]}}}

The entire title/heading pair statement must not extend over multiple
lines.

All characters up to the backslash, except the first blank that follows
\texttt{\textbf{{[}R}}, are used for the title text.

All characters that follow the first \texttt{\textbf{\textbackslash{}}},
ending with the delimiter \texttt{\textbf{{]}}}, are used for the
heading. The first character (blank or non-blank) that follows the
first \texttt{\textbf{\textbackslash{}}} corre\-sponds to the first
column of the report page. The following backslashes determine where
the actual column positions are located.

In addition to being saved in the dictionary, the heading text is
parsed at the time the sourcce that contains the heading is loaded,
to produce a table of column widths and locations of the text to be
displayed. This table is used by the set of words that output the
contents of fields for the Report Generator; this word set includes:
\texttt{\textbf{?B}}, \texttt{\textbf{?N}}, and \texttt{\textbf{?1}}.
Thus, each column ``knows'' where it should appear on the page and
how wide it should be.

When displaying \texttt{\textbf{BYTES}} fields, it is necessary to
ensure that the width of the heading text for that field matches the
width of the storage field, plus a few extra spaces as desired for
column separation. Any fewer spaces, or significantly more spaces,
will result in a skewed output.

With numeric fields, caution should be exercised that the length of
the field to be printed does not exceed the width of the column to
be used. Should the actual size of a string exceed the column width,
it will none\-theless be printed in full and the remaining columns
will be shifted right to accommodate it.

The address of the columns table is left on the stack at load time
by \texttt{\textbf{{[}R}}; this is the address that must be passed
to the word \texttt{\textbf{LAYOUT}}. \texttt{\textbf{LAYOUT}} initiates
the printing of a report and specifies the type of page heading routine
to be invoked. It also saves the address of the title/column heading
table (in user variable \texttt{\textbf{RPT}}) so that each page of
the report will display the same header information.

\nwenddocs{}\nwbegincode{231}\sublabel{NWtG4Lr-4JwrrS-1}\nwmargintag{{\nwtagstyle{}\subpageref{NWtG4Lr-4JwrrS-1}}}\moddef{layout~{\nwtagstyle{}\subpageref{NWtG4Lr-4JwrrS-1}}}\endmoddef\nwstartdeflinemarkup\nwusesondefline{\\{NWtG4Lr-1HFlBt-1}}\nwenddeflinemarkup
: LAYOUT ( a -- )   RPT !  0 P# !  +PAGE ;
\nwused{\\{NWtG4Lr-1HFlBt-1}}\nwendcode{}\nwbegindocs{232}\nwdocspar

If the title/column-heading pair is to be used in several reports,
the address of the table for the title/heading pair may be used as
the value for a \texttt{\textbf{CONSTANT}}, thus giving a name to
the title/heading pair:

\texttt{\textbf{{[}R A Report $\backslash$Col1$\backslash$Col2$\backslash$Col3{]}}}~\\
\texttt{\textbf{ CONSTANT 'SHOW'}}~\\
\texttt{\textbf{ : SHOW 'SHOW' LAYOUT ... ;}}

Otherwise, it is more efficient to just keep this address for a \texttt{\textbf{LITERAL}}
to compile as a literal in the definition that uses this report:

\texttt{\textbf{{[}R A Report $\backslash$Col1$\backslash$Col2$\backslash$Col3{]}}}~\\
\texttt{\textbf{ : SHOW LITERAL LAYOUT ; }}

This address may, of course, be \texttt{\textbf{DUP}}ed if more than
one reference is required, provided the \texttt{\textbf{DUP}} appears
outside any definition (and thus is executed):

\texttt{\textbf{{[}R A Report $\backslash$Col1$\backslash$Col2$\backslash$Col3{]}
DUP}}~\\
\texttt{\textbf{ : SUMMARY LITERAL LAYOUT ... ;}}~\\
\texttt{\textbf{ : SHOW LITERAL LAYOUT ; }}

An additional word, \texttt{\textbf{{[}R+}} provides additional functionality
by parsing the word that immediately follows it and executing this
word on the first line of each page. Otherwise, its behavior is identical
to \texttt{\textbf{{[}R}}.

\nwenddocs{}\nwbegincode{233}\sublabel{NWtG4Lr-3WJ4qF-1}\nwmargintag{{\nwtagstyle{}\subpageref{NWtG4Lr-3WJ4qF-1}}}\moddef{title-heading~{\nwtagstyle{}\subpageref{NWtG4Lr-3WJ4qF-1}}}\endmoddef\nwstartdeflinemarkup\nwusesondefline{\\{NWtG4Lr-1HFlBt-1}}\nwenddeflinemarkup
: (R) ( -- a )   ['] TITLE ,  HERE  HERE 0 ,  HERE 0 ,
    [char] \\ WORD COUNT ,string DROP  HERE SWAP !  [char] ] WORD COUNT ,string
    HERE DUP ALIGNED  SWAP ?DO  0 C,  LOOP
    HERE ROT !  COUNT  DUP ROT 0  ROT 0 DO
        OVER I + C@  [char] \\ = IF
            I SWAP -  ,  BL OVER I + C!  I 1+
    THEN  LOOP  NIP - ,  -1 , ;

: [R+ ( -- a ) \\ Usage: [R+ <name> <title> \\ <col> \\ ... \\ <col> ]
    ' , (R) ;
: [R ( -- a ) \\ Usage: [R <title> \\ <col> \\ ... \\ <col> ]
    0 , (R) ;
\nwused{\\{NWtG4Lr-1HFlBt-1}}\nwendcode{}\nwbegindocs{234}\nwdocspar

\textbf{REFERENCES}

\texttt{\textbf{+PAGE}}, \prettyref{sub:The-Page-Banner} 


\subsubsection{Formatting Lines \label{sub:Formatting-Lines}}

To the report generator, a line consists of a series of columns, each
of which has as fixed width. These columns are used to align the data
to be printed, with all data right justified in the current column.

The following words are provided by the Report Generator to display
fields within the columns determined by the title/column heading pair:
\begin{description}
\item [{Word}] \textbf{Stack Action}
\item [{\texttt{.N}}] (\ n) Displays the single-length integer \textit{n}
right justified in the next column, in the format used by \texttt{\textbf{.}}
(dot).
\item [{\texttt{.L}}] (\ n) Displays the single-length integer \textit{n}
right justified in the next column, in the format used by \texttt{\textbf{L.}}.
\item [{\texttt{.D}}] (\ d) Displays the double-length integer \textit{d}
right justified in the next column, in the format used by \texttt{\textbf{D.}}
\item [{\texttt{?N}}] (\ a) Displays the contents \textit{a} address as
a single-length integer \textit{n} right justified in the next column,
in the format used by \texttt{\textbf{.}} (dot).
\item [{\texttt{?L}}] (\ a) Displays the contents \textit{a} address as
a single-length integer \textit{n} right justified in the next column,
in the format used by \texttt{\textbf{L.}}.
\item [{\texttt{?D}}] (\ a) Displays the contents \textit{a} address as
a single-length integer \textit{n} right justified in the next column,
in the format used by \texttt{\textbf{D.}}.
\item [{\texttt{?1}}] (\ a) Displays the contents of the specified \texttt{\textbf{1BYTE}}
field, right justified in the next column.
\item [{\texttt{?B}}] (\ n\ a) Reads and displays a \texttt{\textbf{BYTES}}
field, according to the declared length, left-justified in the next
column. \texttt{\textbf{PAD}} is used as intermediate storage of the
field.
\item [{\texttt{.M/D/Y}}] (\ n) Given a Julian date, displays it in the
next report column. Since this routine invokes \texttt{\textbf{(DATE)}},
it will work with either calendar. Most data base applications prefer
to use the mm/dd/yyyy format.
\item [{\texttt{.D/M/Y}}] (\ n) Given a Julian date, displays it in the
next report column. Since this routine invokes \texttt{\textbf{(DATE)}},
it will work with either calendar. Some data base applications prefer
to use the dd-mmm-yyyy format.
\item [{\texttt{.WHEN}}] ( n) Given the time in seconds, displays the hh:mm:ss
in the next report column.
\end{description}
Each of these operators advances the columns table to the next column,
determines the width of the new field, and then right-justifies the
output string in this column.

\nwenddocs{}\nwbegincode{235}\sublabel{NWtG4Lr-2ruhRa-1}\nwmargintag{{\nwtagstyle{}\subpageref{NWtG4Lr-2ruhRa-1}}}\moddef{output~{\nwtagstyle{}\subpageref{NWtG4Lr-2ruhRa-1}}}\endmoddef\nwstartdeflinemarkup\nwusesondefline{\\{NWtG4Lr-1HFlBt-1}}\nwenddeflinemarkup
: (D.) ( d -- a n )
    SWAP OVER DUP 0< IF DNEGATE THEN
    <#  #S ROT SIGN  #> ;

: .N ( n -- )   DUP 0< (D.) RIGHT ;
: .L ( n -- )   DUP 0< (D.) RIGHT ;
[DEFINED] SFALIGN [IF]
: .FL ( F: r -- )   PAD 4 REPRESENT IF  <#
        >R 1- DUP ABS 0 #S 2DROP SIGN  [CHAR] e HOLD        \\ exponent
        3 0 DO  PAD 3 + I - C@ HOLD  LOOP  [CHAR] . HOLD
        PAD C@ HOLD  R> IF  [CHAR] - HOLD  THEN  0 0 #>
    ELSE  S" ?.?e?"  THEN  RIGHT ;
[THEN]
: .D ( d -- )   (D.) RIGHT ;

: ?N ( a -- )   N@ .N ;
: ?L ( a -- )   L@ .L ;
[DEFINED] SFALIGN [IF]
: ?FL ( a -- )   FL@ .FL ;
[THEN]
: ?D ( a -- )   D@ .D ;
: ?1 ( a -- )   1@ .N ;
: ?S ( n a -- )   (S.) LEFT ;
: ?B ( n a -- )    2DUP B@  OVER FILE-PAD SWAP 4 MIN nC@ IF
        ?S  ELSE  2DROP SKIP-COL
    THEN ;

: .M/D/Y ( n -- )    ?DUP IF
        MM/DD/YYYY RIGHT  ELSE  SKIP-COL
    THEN ;
: .D-M-Y ( n -- )    ?DUP IF
        DD-MMM-YYYY RIGHT  ELSE  SKIP-COL
    THEN ;
: .WHEN ( n -- )    ?DUP IF
        BASE @ >R  0  <#
        DECIMAL #  6 BASE ! #  [char] : HOLD
        DECIMAL #  6 BASE ! #  [char] : HOLD
        DECIMAL # #  #>  R> BASE !
        RIGHT  ELSE  SKIP-COL
    THEN ;
\nwused{\\{NWtG4Lr-1HFlBt-1}}\nwendcode{}\nwbegindocs{236}\nwdocspar

You may also build your own formatting words to display columns, using
the words \texttt{\textbf{RIGHT}}, \texttt{\textbf{LEFT}} and \texttt{\textbf{CENTER}}.
\begin{description}
\item [{\texttt{RIGHT}}] (\ a\ n) Displays an alphanumeric string of
length \textit{n}, beginning at address \textit{a}, right-justified
in the next column.
\item [{\texttt{LEFT}}] (\ a\ n) Displays an alphanumeric string of length
\textit{n}, beginning at address \textit{a}, left-justified in the
next column.
\item [{\texttt{CENTER}}] (\ a\ n) Displays an alphanumeric string of
length \textit{n}, beginning at address \textit{a}, centered in the
next column.
\end{description}
\nwenddocs{}\nwbegincode{237}\sublabel{NWtG4Lr-2YnRrI-1}\nwmargintag{{\nwtagstyle{}\subpageref{NWtG4Lr-2YnRrI-1}}}\moddef{rlc~{\nwtagstyle{}\subpageref{NWtG4Lr-2YnRrI-1}}}\endmoddef\nwstartdeflinemarkup\nwusesondefline{\\{NWtG4Lr-1HFlBt-1}}\nwenddeflinemarkup
: RIGHT ( a n -- )   COLS OVER -  SPACES TYPE  SPACE ;
: LEFT ( a n -- )   COLS OVER - >R  TYPE  R> 1+ SPACES ;
: CENTER ( a n -- )    COLS OVER - DUP 2/  DUP >R - SPACES
    TYPE  R> 1+ SPACES ;
\nwused{\\{NWtG4Lr-1HFlBt-1}}\nwendcode{}\nwbegindocs{238}\nwdocspar

The stack arguments are identical to those of \texttt{\textbf{TYPE}}.

In fact, \texttt{\textbf{.N}}, \texttt{\textbf{.L}}, \texttt{\textbf{.D}},
\texttt{\textbf{?N}}, \texttt{\textbf{?L}}, \texttt{\textbf{?D}},
\texttt{\textbf{?1}}, and \texttt{\textbf{?B}} are defined using \texttt{\textbf{RIGHT}}
and behave according to its rules:
\begin{enumerate}
\item If the length of the output string exceeds the width of the column,
the results are unpredictable but will include loss of format control.
\item If the length of the output string equals the width of the column,
the string is displayed and the column pointer is advanced.
\item If the length of the output string is less than the width of the column,
the difference is output as blank spaces, so that the string will
be right justified.
\item Text strings are also right justified; however the string's trailing
blanks are included, making them appear left justified.
\end{enumerate}
Here is an example:

\nwenddocs{}\nwbegincode{239}\sublabel{NWtG4Lr-34dcAv-1}\nwmargintag{{\nwtagstyle{}\subpageref{NWtG4Lr-34dcAv-1}}}\moddef{accounts~{\nwtagstyle{}\subpageref{NWtG4Lr-34dcAv-1}}}\endmoddef\nwstartdeflinemarkup\nwusesondefline{\\{NWtG4Lr-2V5MiK-1}}\nwenddeflinemarkup
FILE Accounts.dbf

: /Accounts ( -- ) \\ Runtime file setup
    Accounts.dbf  S" Accounts.dbf" >FILE
    TRUE REVERSE !  LITTLE-ENDIAN ;

76 500 0 BLOCK-DATA ACCOUNTS

0  10 BYTES NAMES   NUMERIC ACCT#   DOUBLE BALANCE   DROP

: (.$) ( d - a n)   SWAP OVER DABS
    <#  # #  46 HOLD  #S  SIGN  #> ;
: .ACCOUNT   ACCT# ?N  NAMES ?B
    BALANCE D@ (.$) RIGHT ;

[R Account Balances\\   Account#\\Name          \\Balance]
    CONSTANT ACCOUNTS-TITLE

: balances   ACCOUNTS-TITLE LAYOUT
    ACCOUNTS RECORDS ?DO  I READ  .ACCOUNT  LOOP ;

: enter-bal ( n d)   ACCOUNTS  SLOT READ  BALANCE D!
    ACCT# N!  NAMES PUT ;
\nwused{\\{NWtG4Lr-2V5MiK-1}}\nwendcode{}\nwbegindocs{240}\nwdocspar

The word \textbf{BALANCES} produces:

\begin{lstlisting}
Page 1  05/12/2005     
Account Balances 
   Account# Name           Balance 
        456 John Doe        100.00 
        489 Mary Smith     2970.00 
        620 Ed Poore          2.59 
\end{lstlisting}


Notice that the first column heading, ``Account\#'' appears in the
title/ column-heading pair three spaces after the backslash. This
causes the heading on the output report to be indented three spaces
(the first space after the backslash counts). On the corresponding
formatted lines, the first field is formatted with \texttt{\textbf{?N}},
which right-justifies the string against the end of the ``Account\#''
heading.

The middle column is formatted with \texttt{\textbf{?B}}; as a text
string this field is effectively left-justified. To make the output
more pleasing, we have forced the ``Name'' column heading to be
flush left to match.

In the third column, the data is once again right justified under
the last character of the ``Balance'' column heading. In this case,
we wished to display the double-length field in dollars-and-cents
format, requiring the use of a pictured numeric output routine (Lines
4 and 5 of the listing). On Line 6, this pictured numeric output string
is displayed, but with \texttt{\textbf{RIGHT}} rather than \texttt{\textbf{TYPE}}.

If the previous column displayed was the final column on a line, \texttt{\textbf{RIGHT}}
automatically advances to the next line and resets the column table
to begin with the first column on the line.

The following words are available for special formatting requirements:
\begin{description}
\item [{Word}] \textbf{Stack Action}
\item [{\texttt{0COL}}] (\ \ ) Resets the column table pointer to point
to the first column width. Exercise care with this word, since it
can cause the output to be misaligned if it is not issued when the
actual output print position is at the beginning of a line.
\item [{\texttt{COLS}}] (\ --\ n) Advances the column pointer and returns
the width of the new column.
\item [{\texttt{SKIP-COL}}] (\ \ ) Skips one column.
\item [{\texttt{SKIP-COLS}}] (\ n) Skips \textit{n} columns. 
\end{description}
\nwenddocs{}\nwbegincode{241}\sublabel{NWtG4Lr-3BC2yN-1}\nwmargintag{{\nwtagstyle{}\subpageref{NWtG4Lr-3BC2yN-1}}}\moddef{cols~{\nwtagstyle{}\subpageref{NWtG4Lr-3BC2yN-1}}}\endmoddef\nwstartdeflinemarkup\nwusesondefline{\\{NWtG4Lr-1HFlBt-1}}\nwenddeflinemarkup
#USER CELL +USER RPT  \\ holds the address of the current report title block.
      CELL +USER #COL \\ holds the address of the current column width.
TO #USER

: 0COL ( -- )   RPT @ @ #COL ! ;
\LA{}line~{\nwtagstyle{}\subpageref{NWtG4Lr-47jTCN-1}}\RA{}
: COL ( -- n )   #COL @  DUP CELL+  #COL ! ;
: COLS ( -- n )    COL @  DUP 0< 0= IF  EXIT
    THEN  DROP +L  COL @ ;

: SKIP-COL ( -- )   COLS 1+ SPACES ;
: SKIP-COLS ( n -- )   0 DO  SKIP-COL  LOOP ;
\nwused{\\{NWtG4Lr-1HFlBt-1}}\nwendcode{}\nwbegindocs{242}\nwdocspar


\subsubsection{Controlling Paging \label{sub:Controlling-Paging}}

The report generator does not count each output line, since this capability
tends to be too environmentally dependent. Instead, it assumes that
the output can be captured and paginated appropriately. This does
not, however, stop an application from doing its own page control.

When using the Report Generator, it is not necessary to explicitly
invoke a ``new-line'' function at the beginning of each row of data.
As the field-display operators cycle through the columns table, after
the last column has been displayed, the next operator resets the column
pointer to the beginning of the column table again and issues a ``new-line.''

The following words control pagination:
\begin{description}
\item [{Word}] \textbf{Stack Action}
\item [{\texttt{+PAGE}}] Starts a new page, incrementing the page count
in \texttt{\textbf{P\#}} and displaying the headings for the new page.
\end{description}
\nwenddocs{}\nwbegincode{243}\sublabel{NWtG4Lr-4GHBLz-1}\nwmargintag{{\nwtagstyle{}\subpageref{NWtG4Lr-4GHBLz-1}}}\moddef{page~{\nwtagstyle{}\subpageref{NWtG4Lr-4GHBLz-1}}}\endmoddef\nwstartdeflinemarkup\nwusesondefline{\\{NWtG4Lr-1HFlBt-1}}\nwenddeflinemarkup
#USER CELL +USER P# TO #USER

VARIABLE 'PAGE   ' PAGE 'PAGE !

: +PAGE ( -- )
    'PAGE @ EXECUTE  1 P# +!  +L  ." Page "  P# ?
    SPACE  DATE  SPACE  'APP @ COUNT TYPE
    +L  RPT @  CELL - @ ?DUP IF EXECUTE THEN ;
\nwused{\\{NWtG4Lr-1HFlBt-1}}\nwendcode{}\nwbegindocs{244}\nwdocspar

\begin{description}
\item [{\texttt{+L}}] Issues a \texttt{\textbf{CR}} and used to increment
the line count in \texttt{\textbf{L\#}}, but that was removed in favor
of shorter reports that do not have extra page headers. Also resets
the column pointers using \texttt{\textbf{0COL}}.
\end{description}
\nwenddocs{}\nwbegincode{245}\sublabel{NWtG4Lr-47jTCN-1}\nwmargintag{{\nwtagstyle{}\subpageref{NWtG4Lr-47jTCN-1}}}\moddef{line~{\nwtagstyle{}\subpageref{NWtG4Lr-47jTCN-1}}}\endmoddef\nwstartdeflinemarkup\nwusesondefline{\\{NWtG4Lr-3BC2yN-1}}\nwenddeflinemarkup
: +L ( -- )   -spin  CR  0COL ;
\nwused{\\{NWtG4Lr-3BC2yN-1}}\nwendcode{}\nwbegindocs{246}\nwdocspar

\textbf{REFERENCES}

\textbf{0COL}, \prettyref{sub:Controlling-Paging} 


\subsubsection{The Page Banner \label{sub:The-Page-Banner}}

At the top of each page of the report appears the ``page banner''
which includes:

\texttt{\textbf{Page nn <date> <optional text>}}

where \texttt{\textbf{\textit{nn}}} is the current page number, and
\texttt{\textbf{<date>}} is the current system date.

If you wish to modify or eliminate the optional text, simply change
the string in the definition of \texttt{\textbf{'APP}} variable. The
definition \texttt{\textbf{APP''~<optional~text>''}} is available
to make this easier.

It is possible to eliminate the page banner entirely by replacing
the word \texttt{\textbf{LAYOUT}} with \texttt{\textbf{HEADING}}.
Like \texttt{\textbf{LAYOUT}}, \texttt{\textbf{HEADING}} takes as
an argument the address of a title/column heading table as provided
by the word \texttt{\textbf{{[}R}}, and establishes this table as
current. It then displays the ``title'' line, without attempting
to center it, and on the next line displays the column headings.
\begin{description}
\item [{\texttt{HEADING}}] (\ a) Saves the address of a title/heading
table, and outputs the title and column headings. 
\end{description}
\nwenddocs{}\nwbegincode{247}\sublabel{NWtG4Lr-14go8R-1}\nwmargintag{{\nwtagstyle{}\subpageref{NWtG4Lr-14go8R-1}}}\moddef{heading~{\nwtagstyle{}\subpageref{NWtG4Lr-14go8R-1}}}\endmoddef\nwstartdeflinemarkup\nwusesondefline{\\{NWtG4Lr-1HFlBt-1}}\nwenddeflinemarkup
: HEADING ( a -- )    DUP RPT !
    CELL+  DUP >R  CELL+ COUNT TYPE
    R@ 3 CELLS - @ ?DUP IF EXECUTE THEN  +L
    R> @  COUNT TYPE +L ;
\nwused{\\{NWtG4Lr-1HFlBt-1}}\nwendcode{}\nwbegindocs{248}\nwdocspar


\subsubsection{How the Columns Table Works \label{sub:How-the-Columns-Table-Works}}

The format of the columns table is:
\begin{description}
\item [{Byte}] \textbf{Contents}
\item [{address\ -\ 8}] Address of the optional routine executed at the
end of the header.
\item [{-\ 4}] Address of the page heading vectored routine.
\item [{+\ 0}] Address of column widths table.
\item [{+~4}] Address of column headings.
\item [{+~8}] Counted string of header.
\item [{+~8~+~(h)}] Counted string of columns.
\item [{+~8~+~(h)~+~(c)}] Column widths table.
\item [{+~8~+~(h)~+~(c)~+~n}] -1 marks end of column table.
\end{description}
A -1 entry in the table indicates the end.

A heading line can contain up to 128 characters. These lines are used
to establish a table of column widths at load time in the following
manner.

Starting from the backslash in the title/column-heading pair, \texttt{\textbf{{[}R}}
scans forward looking for additional backslashes in a loop. Each time
it encounters a backslash; it replaces it with a space and computes
the difference from the starting point or previous heading (the width
of the field), and compiles this into the table. This loop repeats
until the \texttt{\textbf{{]}}} delimiter is encountered. At this
point, the indicator for the end of the line (a column width of -1)
is inserted and the scan is complete.

For example, suppose the following is the set-up string for a set
of column headings (the numbers across the top are your guide to indicate
column positions):

\texttt{01234567890123456789012345678901234567890123456789}

\texttt{\textbf{\textbackslash{} Account\#\textbackslash{}Name \textbackslash{}Balance{]}}}

The backslash after the ``Account\#'' heading occurs at relative
position 12; thus the number 12 is compiled into the table as the
width of the first column. The backslash after the ``Name'' heading
occurs at relative position 22; the difference, 10, is compiled as
the width of the second column. The delimiting \textbf{{]}} occurs
at 30, and the difference of 8 is compiled as the width of the third
column. Finally, a -1 is compiled to indicate the end of the table.

The finished column table, as constructed by \texttt{\textbf{{[}R}},
contains:

\texttt{\textbf{12 10 8 -1}}\textbf{ }

The total width of all columns equals the position number of the last
non-blank character.

A line may contain as many columns as required for the output format.
Due to the method of establishing columns, the minimum width of a
column is one character. 


\subsubsection{Non-standard Report Headings \label{sub:Non-standard-Report-Headings}}

By default, the ``new-page function'' performs the following steps
at the beginning of each page, including the first page:
\begin{enumerate}
\item Displays the page banner as described in \prettyref{sub:The-Page-Banner};
\item Performs a \texttt{\textbf{+L}};
\item Executes a word called \texttt{\textbf{TITLE}}. \texttt{\textbf{TITLE}}
is defined as:
\end{enumerate}
\nwenddocs{}\nwbegincode{249}\sublabel{NWtG4Lr-3B2QOB-1}\nwmargintag{{\nwtagstyle{}\subpageref{NWtG4Lr-3B2QOB-1}}}\moddef{title~{\nwtagstyle{}\subpageref{NWtG4Lr-3B2QOB-1}}}\endmoddef\nwstartdeflinemarkup\nwusesondefline{\\{NWtG4Lr-1HFlBt-1}}\nwenddeflinemarkup
: TITLE ( -- )   RPT @ HEADING ;
\nwused{\\{NWtG4Lr-1HFlBt-1}}\nwendcode{}\nwbegindocs{250}\nwdocspar

\texttt{\textbf{RPT}} is the user variable that points to the current
title/column-heading table. \texttt{\textbf{HEADING}} displays the
title and column-heading lines from the given table (\prettyref{sub:How-the-Columns-Table-Works}).

However, the Report Generator lets you vector the third function above.
This feature lets you execute your own definition instead of, or in
addition to, \texttt{\textbf{TITLE}}. For instance, you might add
other lines of information below the page banner.

This vectoring is possible without recompiling the \texttt{\textbf{FILES}}
utility because the second cell of the title/column-heading table
contains the address of the routine to be executed at the top of each
page. When \texttt{\textbf{{[}R}} generates this table, it copies
in the address of the routine \texttt{\textbf{TITLE}} by default.
By re-setting this address to point to your own definition, you can
change the output of the new-page function.

Here are some examples:

\texttt{\textbf{( Non-standard Report Headings )}}

\texttt{\textbf{APP\textquotedbl{} Acme Manufacturing Co.\textquotedbl{}}}

\texttt{\textbf{VARIABLE WHICH}}

\texttt{\textbf{{[}R \textbackslash{}Col1 \textbackslash{}Col2 \textbackslash{}Col3{]}
CONSTANT SHOW-TITLE1}}

\texttt{\textbf{: .ITEM WHICH ? ;}}~\\
\texttt{\textbf{ : .\textquotedbl{}ITEM\textquotedbl{} .\textquotedbl{}
Report on Item No. \textquotedbl{} .ITEM ;}}~\\
\texttt{\textbf{ : 'ITEM' .\textquotedbl{}ITEM\textquotedbl{} TITLE
;}}

\texttt{\textbf{: SHOW1 ( n) WHICH !}}~\\
\texttt{\textbf{ {[}'{]} 'ITEM' SHOW-TITLE1 CELL- !}}~\\
\texttt{\textbf{ SHOW-TITLE1 LAYOUT ;}}

\texttt{\textbf{{[}R+ .\textquotedbl{}ITEM\textquotedbl{} \textbackslash{}Col1
\textbackslash{}Col2 \textbackslash{}Col3{]} CONSTANT SHOW-TITLE2}}

\texttt{\textbf{: SHOW2 ( n) WHICH !}}~\\
\texttt{\textbf{ SHOW-TITLE2 LAYOUT ;}}

\texttt{\textbf{{[}R+ .ITEM Report on Item No. \textbackslash{}Col1
\textbackslash{}Col2 \textbackslash{}Col3{]} CONSTANT SHOW-TITLE3}}

\texttt{\textbf{: SHOW3 ( n) WHICH !}}~\\
\texttt{\textbf{ SHOW-TITLE3 LAYOUT ;}}

This example shows a report for some particular item that is selected
numerically, like this:

\texttt{\textbf{2500\ SHOW1}} Stores \textit{2500} into \texttt{\textbf{WHICH}}
so that you can see a report for Item 2500.

The report generator will print the item number at the top of each
page, with headings:

\begin{lstlisting}
Page 1 05/18/2005 Acme Manufacturing Co.
Report on Item No. 2500
Col1 Col2 Col3
\end{lstlisting}


where the top line is the standard page banner, and the text ``Report
on Item No.\_\_\_'' is formatted by user-defined code.

Here are the steps used in the above examples to vector the user-defined
code into the new-page routine:
\begin{enumerate}
\item Create a title/column heading pair as usual (cases 1 and 2, leave
the ``title'' blank).
\item Define words, that will be executed as the third step of the new-page
routine. It includes the message ``Report on Item No.,'' followed
by the display of the chosen item number. Finally it invokes \texttt{\textbf{TITLE}},
which displays the title/column-heading pair.
\item Define the report-generating word (the word \texttt{\textbf{SHOW1}})
in the usual way, using the address of the title to set the vectored
location (1 cell back).
\item Alternatively, use \texttt{\textbf{{[}R+}} to create titles that have
custom execution (the words \texttt{\textbf{SHOW2}} and \texttt{\textbf{SHOW3}}).
\end{enumerate}
All of these reports have the same output, and other reports may be
co-resident; since each has its own title/ column-heading table, each
has its own new-page execution behavior. 


\subsubsection{Totals and Subtotals \label{sub:Totals-and-Subtotals}}

The Data Base Support package includes a simple utility for computing
subtotals and totals of numeric fields as the report is being displayed.
In general, the following steps must be followed:
\begin{enumerate}
\item Allot enough ``working storage'' for the registers. Working storage
is created by invoking \texttt{\textbf{n\ ALLOT}} immediately after
the word \texttt{\textbf{EMPTY}} at the beginning of the \texttt{\textbf{FILES}}
load block (\prettyref{sub:Installing-The-Package}). The value of
\textit{n} is calculated by this Forth phrase:\textbf{}\\
\texttt{\textbf{( \# of registers needed) 3 {*} 2 {*} CELLS 2 CELLS
+ 16 +}}\\
See the source comment associated with the word \texttt{\textbf{REGISTER}}
on your system.
\end{enumerate}
\nwenddocs{}\nwbegincode{251}\sublabel{NWtG4Lr-8Tw6T-1}\nwmargintag{{\nwtagstyle{}\subpageref{NWtG4Lr-8Tw6T-1}}}\moddef{register~{\nwtagstyle{}\subpageref{NWtG4Lr-8Tw6T-1}}}\endmoddef\nwstartdeflinemarkup\nwusesondefline{\\{NWtG4Lr-1HFlBt-1}}\nwenddeflinemarkup
: REGISTER ( -- a )   WORKING 16 + ;
\nwused{\\{NWtG4Lr-1HFlBt-1}}\nwendcode{}\nwbegindocs{252}\nwdocspar

\begin{enumerate}
\item At the beginning of your report word, simultaneously define and clear
as many accumulator-pairs as there are fields you wish to total, using
the word \texttt{\textbf{TOTALS}} (see below).
\item As the fields are being displayed, accumulate the values in the subtotal
registers by using either \texttt{\textbf{SUM}} or \texttt{\textbf{FOOT}}.
\item When you wish to display the subtotals (if at all), use the word \texttt{\textbf{SUB}},
followed by an appropriate numeric output command.
\item When you wish to display the totals, invoke \texttt{\textbf{TOTAL}}.
This copies the totals to the subtotals registers and adds them to
the grand totals. Then use \texttt{\textbf{SUB}} as in Step 3.
\item When you wish to display the grand totals, invoke \texttt{\textbf{GRAND}}.
This copies the grand totals to the subtotals registers. Then use
\texttt{\textbf{SUB}} as in Step 3.
\end{enumerate}
Here are the relevant words in detail:
\begin{description}
\item [{Word}] \textbf{Stack Description}
\item [{\texttt{TOTALS}}] (\ n) Defines \textit{n} subtotal accumulators,
and \textit{n} grand-total accumulators, and sets all to zero. Each
accumulator is double-length.\\
For example, if you are totaling three fields, the phrase:\\
\texttt{\textbf{3 TOTALS}}\textbf{}\\
creates three subtotal accumulators and three grand-total accumulators,
and sets all to zero. \textbf{TOTALS} must be used at the beginning
of a report if any of the following words are used.
\end{description}
\nwenddocs{}\nwbegincode{253}\sublabel{NWtG4Lr-3QzPba-1}\nwmargintag{{\nwtagstyle{}\subpageref{NWtG4Lr-3QzPba-1}}}\moddef{totals~{\nwtagstyle{}\subpageref{NWtG4Lr-3QzPba-1}}}\endmoddef\nwstartdeflinemarkup\nwusesondefline{\\{NWtG4Lr-1HFlBt-1}}\nwenddeflinemarkup
: TOTALS ( n -- )   2* CELLS REGISTER  OVER 3 * 2 CELLS + ERASE  REGISTER ! ;
\nwused{\\{NWtG4Lr-1HFlBt-1}}\nwendcode{}\nwbegindocs{254}\nwdocspar

\begin{description}
\item [{\texttt{SUM}}] (\ d\ n) Adds \textit{d} to the subtotal accumulator
for the nth relative field.
\end{description}
\nwenddocs{}\nwbegincode{255}\sublabel{NWtG4Lr-3qrQOy-1}\nwmargintag{{\nwtagstyle{}\subpageref{NWtG4Lr-3qrQOy-1}}}\moddef{sum~{\nwtagstyle{}\subpageref{NWtG4Lr-3qrQOy-1}}}\endmoddef\nwstartdeflinemarkup\nwusesondefline{\\{NWtG4Lr-1HFlBt-1}}\nwenddeflinemarkup
: SUM ( d n -- )   2* CELLS REGISTER + 2+! ;
\nwused{\\{NWtG4Lr-1HFlBt-1}}\nwendcode{}\nwbegindocs{256}\nwdocspar

\begin{description}
\item [{\texttt{FOOT}}] (\ d\ -\ d) Advances to the next subtotal register
and adds \textit{d} to it. If at the last register, wraps around to
the first.\\
For instance, suppose you have a \texttt{\textbf{DOUBLE}} field called
\texttt{\textbf{SALARY}} that you want to both display and add to
the running total. The phrase:\\
\texttt{\textbf{SALARY D@ FOOT .D}}\textbf{}\\
fetches the contents, adds it to the corresponding sub\-total register,
then displays it.
\end{description}
\nwenddocs{}\nwbegincode{257}\sublabel{NWtG4Lr-W27Dj-1}\nwmargintag{{\nwtagstyle{}\subpageref{NWtG4Lr-W27Dj-1}}}\moddef{foot~{\nwtagstyle{}\subpageref{NWtG4Lr-W27Dj-1}}}\endmoddef\nwstartdeflinemarkup\nwusesondefline{\\{NWtG4Lr-1HFlBt-1}}\nwenddeflinemarkup
: REG ( -- a )   REGISTER 2@ MOD  2 CELLS + DUP
    REGISTER CELL+ !  REGISTER + ;

: FOOT ( d -- d )   2DUP REG 2+! ;
\nwused{\\{NWtG4Lr-1HFlBt-1}}\nwendcode{}\nwbegindocs{258}\nwdocspar

\begin{description}
\item [{\texttt{SUB}}] (\ --\ d) Advances to the next subtotal register
and fetches its contents. Also adds the contents into the corresponding
grand-total accumulator and clears the subtotal register. If at the
last register, wraps around to the first.
\end{description}
\nwenddocs{}\nwbegincode{259}\sublabel{NWtG4Lr-2JLP8G-1}\nwmargintag{{\nwtagstyle{}\subpageref{NWtG4Lr-2JLP8G-1}}}\moddef{sub~{\nwtagstyle{}\subpageref{NWtG4Lr-2JLP8G-1}}}\endmoddef\nwstartdeflinemarkup\nwusesondefline{\\{NWtG4Lr-1HFlBt-1}}\nwenddeflinemarkup
: SUB ( -- d )   REG  DUP >R  2@  2DUP REGISTER @ R@ + 2+!  0 0 R> 2! ; 
\nwused{\\{NWtG4Lr-1HFlBt-1}}\nwendcode{}\nwbegindocs{260}\nwdocspar

\begin{description}
\item [{\texttt{TOTAL}}] ( ) Adds the totals to the grand totals and copies
the totals to the subtotal accumulators. It leaves the totals in a
state such that the display of the subtotals will set the totals to
0.
\end{description}
\nwenddocs{}\nwbegincode{261}\sublabel{NWtG4Lr-1G4TAa-1}\nwmargintag{{\nwtagstyle{}\subpageref{NWtG4Lr-1G4TAa-1}}}\moddef{total~{\nwtagstyle{}\subpageref{NWtG4Lr-1G4TAa-1}}}\endmoddef\nwstartdeflinemarkup\nwusesondefline{\\{NWtG4Lr-1HFlBt-1}}\nwenddeflinemarkup
: TOTAL ( -- )    REGISTER DUP @ DUP 0 DO  >R
        2 CELLS + DUP DUP R@ + 2@  DUP 2OVER ROT ROT 2!
        DUP 2OVER ROT ROT R@ 2* + 2+!  DNEGATE ROT R@ + 2!  R>
    2 CELLS +LOOP  2DROP ;
\nwused{\\{NWtG4Lr-1HFlBt-1}}\nwendcode{}\nwbegindocs{262}\nwdocspar

\begin{description}
\item [{\texttt{GRAND}}] (\ \ ) Copies the grand totals to the subtotal
accumulators.\\
For example, the phrase:\\
\texttt{\textbf{SUB .D}}\textbf{}\\
will display the subtotal of the next field.\\
The phrase:\\
\texttt{\textbf{GRAND SUB .D}}\textbf{}\\
will display the grand total of the next field.
\end{description}
\nwenddocs{}\nwbegincode{263}\sublabel{NWtG4Lr-1VMgcH-1}\nwmargintag{{\nwtagstyle{}\subpageref{NWtG4Lr-1VMgcH-1}}}\moddef{grand~{\nwtagstyle{}\subpageref{NWtG4Lr-1VMgcH-1}}}\endmoddef\nwstartdeflinemarkup\nwusesondefline{\\{NWtG4Lr-1HFlBt-1}}\nwenddeflinemarkup
: GRAND ( -- )   REGISTER DUP @ >R  2 CELLS + DUP R@ 2* +  SWAP R> MOVE ;
\nwused{\\{NWtG4Lr-1HFlBt-1}}\nwendcode{}\nwbegindocs{264}\nwdocspar

The following example shows how subtotals and grand totals can be
easily computed and displayed:

\begin{lstlisting}
Page 1  05/17/2005    
Wine Inventory by Store 
Location         Chablis   Rose  Champagne 

Northern California 
Palo Alto            25     42         78 
San Jose             16     32         50 
Mill Valley          31     29         36 
San Francisco        70     59         82 
                    142    162        246 

Southern California 
Chatsworth           35     48         29 
Woodland Hills       32     40         60 
                     67     88         89 
Grand Total:        209    250        335   
\end{lstlisting}


Here is the code that produced this display:

\nwenddocs{}\nwbegincode{265}\sublabel{NWtG4Lr-1KI31P-1}\nwmargintag{{\nwtagstyle{}\subpageref{NWtG4Lr-1KI31P-1}}}\moddef{wines~{\nwtagstyle{}\subpageref{NWtG4Lr-1KI31P-1}}}\endmoddef\nwstartdeflinemarkup\nwusesondefline{\\{NWtG4Lr-2V5MiK-1}}\nwenddeflinemarkup
FILE Wines.dbf

: /Wines ( -- ) \\ Runtime file setup
    Wines.dbf  S" Wines.dbf" >FILE
    TRUE REVERSE !  LITTLE-ENDIAN ;

28 500 0 BLOCK-DATA WINES

0  16 BYTES Location   NUMERIC Chablis   NUMERIC Rose
     NUMERIC Champagne   DROP

: .amounts   Location ?B  Chablis N@ S>D FOOT .D
    Rose N@ S>D FOOT .D  Champagne N@ S>D FOOT .D ;

: .subs   SUB .D  SUB .D  SUB .D  +L ;

[R Wine Inventory by Store\\Location        \\Chablis\\  Rose\\ Champagne]
    CONSTANT WINES-TITLE

: INVENTORY   WINES-TITLE LAYOUT  3 TOTALS  +L
    ." Northern California"  +L
    WINES RECORDS DO  I READ  .amounts  I 4 = IF  +L
        SKIP-COL .subs +L  ." Southern California " +L
    THEN LOOP  +L
    SKIP-COL  .subs  ." Grand Total:     " COLS DROP
    TOTAL .subs ;

: enter-wine ( Cablis Rose Champagne -- )   WINES  SLOT READ
    Champagne N!  Rose N!  Chablis N!  Location PUT ;
\nwused{\\{NWtG4Lr-2V5MiK-1}}\nwendcode{}\nwbegindocs{266}\nwdocspar

The phrase \texttt{\textbf{3\ TOTALS}} appears in the definition
of \texttt{\textbf{INVENTORY}}. This creates and clears three sets
of accumulators, one set for each field we wish to total.

The word \texttt{\textbf{FOOT}} appears in the definition of \texttt{\textbf{.amounts}}:

\texttt{\textbf{Chablis N@ S>D FOOT .D}}

In this case, the field is \texttt{\textbf{NUMERIC}} (single-length),
so we fetch it with the operator \texttt{\textbf{N@. FOOT}}, however,
expects a double-length number; \texttt{\textbf{S>D}} supplies the
high-order part. \texttt{\textbf{FOOT}} will add the value to the
first subtotal accumulator. \texttt{\textbf{FOOT}} also returns a
copy of the value (as a double-length number). Finally \texttt{\textbf{.D}}
displays the value in Report Generator format.

The second invocation of \texttt{\textbf{FOOT}} in:

\texttt{\textbf{Rose N@ S>D FOOT .D}}

will cause the value of the \texttt{\textbf{Rose}} field to be added
to the second accumulator, and so on.

The word \texttt{\textbf{SUB}} appears in the definition of \texttt{\textbf{.subs}}.
This definition displays the contents of the three subtotal accumulators
in turn. Notice that the three uses of \texttt{\textbf{.D}} correspond
to the second, third, and fourth columns in the report generator;
thus we can only invoke \texttt{\textbf{.subs}} when we are about
to display the second column (after having output or \texttt{\textbf{SKIP}}ped
the first column).

In \texttt{\textbf{INVENTORY}}, we display the standard header with
\texttt{\textbf{LAYOUT}}, below which we display the category heading
``Northern California.''

Inside the loop we display the fields in the usual way, except that
we check to see if the index is 4. If so, then it is time to display
the subtotals for Northern California and the category heading for
Southern California. Here we \texttt{\textbf{SKIP}} the first Report
Generator column, then issue \texttt{\textbf{.subs}}.

After the loop has been completed and the second set of records displayed,
the phrase:

\texttt{\textbf{SKIP .subs}}

displays the subtotals for Southern California, and issues a \texttt{\textbf{+L}}.

Finally we display the text ``Grand Total.'' The trick here is that
we also want to display the grand totals on the same line. We cannot
use \texttt{\textbf{SKIP}}, because it outputs the necessary number
of spaces to get to the next report column; after printing the text,
we're half the way there already. Our solution is to pad the message
with trailing blanks so that the message is 17 characters long (the
width of the first field plus one); this leaves us in position to
display the second column.

However, the columns table must also be advanced to point to the second
column. The phrase:

\texttt{\textbf{COLS DROP}}

is the same as \texttt{\textbf{SKIP}} without issuing the spaces.

Finally, \texttt{\textbf{TOTAL}} copies the total accumulators to
the subtotal registers, and \texttt{\textbf{.subs}} displays these. 


\subsection{Data Base Design \label{sub:Data-Base-Design}}

Before building a house, it is best to have a blueprint. So too, before
defin\-ing files and records, it is best to map-out the overall database
needs.

In general, we can formulate two simple rules for planning your database:
\begin{enumerate}
\item Look at the kinds of information you have.
\item Arrange like kinds of information into files. 
\end{enumerate}

\subsubsection{A Hospital Patient Management Data Base \label{sub:A-Hospital-Patient-Management-Data-Base}}

Our goal in this example is to create a database for tracking patients
in a large hospital. For each patient there is a set of information:
items such as address, height, weight, date-of-birth, and so on. (Note
that we save date-of-birth and not age. We can always compute age
if that's what we need in a report, but a date-of-birth is never obsolete.)

Clearly, this information all belongs in a single record, one per
patient. However, there are also a variable number of information
items that may be associated with each patient. For instance, each
patient may have a different number of tests, and each type of test
may have a different amount of information that it produces. In short,
the amount of information that we need to keep for each patient is
variable in length.

At this juncture, many database designers would opt for variable-length
records and fields. But variable-length records are complex and slow,
as we saw in \prettyref{sub:Overview}. With nearly the same convenience
we can achieve the same results by using a fixed header plus a variable
number of subordinate records.

\begin{figure}
\caption{\label{fig:14}\protect\includegraphics{13_home_druffer_Dropbox_ColorForth_File_Docs_Patient_Results.eps}}
\end{figure}


\begin{figure}
\caption{\label{fig:15}\protect\includegraphics{14_home_druffer_Dropbox_ColorForth_File_Docs_Test_Results.eps}}
\end{figure}


This is where chains come in. Fig. \prettyref{fig:14} shows that
a field in each \texttt{\textbf{PATIENTS}} record can point to the
first in a series of \texttt{\textbf{RESULTS}} records, each of which
is chained to the next. We achieve the same effect, but at much less
expense.

Now suppose that we need to record particular test results for three
differ\-ent tests for each patient. We can accomplish this by providing
three fields in each \texttt{\textbf{PATIENTS}} record, each pointing
to a different result record or chain of records (Fig. \prettyref{fig:15}).

\begin{figure}
\caption{\label{fig:16}\protect\includegraphics{15_home_druffer_Dropbox_ColorForth_File_Docs_Patients.eps}}
\end{figure}


Here's another intriguing problem. The application demands that a
patient record can be found on the basis of a ``patient number.''
A patient number is a very large number issued in sequence; in other
words, the patient number last issued reflects the total number of
patients that have ever been admitted to the hospital in its history.
This number could reach 200,000 during the lifetime of our system.
However, the department for which we are designing this application
expects to see only 30,000 patients during the lifetime of this system.
Unfortunately, maintaining an ordered index even of 30,000 records,
indexed on ``patient numbers,'' is unmanageable.

Is there any way we can translate a patient number directly into a
record number for our \texttt{\textbf{PATIENTS}} file? Let's try this:
we'll create a file of 200,000 records, each record being only two
bytes long. This gives us one record per potential patient number.
The two-byte field will contain a record number, pointing to the record
in the \texttt{\textbf{PATIENTS}} file corresponding to the patient
number (Fig. \prettyref{fig:16}).

This elegant scheme requires 400 blocks for the look-up file, and
yet gives immediate access to a patient record, with only one intermediate
disk access. No searching is needed. Furthermore, when new patients
are added, \texttt{\textbf{SLOT}} is not needed in the look-up file.

In general, direct access is much faster than searching, and should
be used whenever appropriate. 


\subsubsection{An Integrated Business System \label{sub:An-Integrated-Business-System}}

Our goal in this example is to create a package that will track income
(sales and accounts receivable) and expenses (purchase orders, accounts
payable, and payroll), and from these inputs will produce general
ledgers, income statements and balance sheets.

Although many commercial business packages treat these functions as
sep\-arate programs, our goal is to integrate them into a single
system. By doing so, we will make the system simpler to use and reduce
the opportunity for error. For instance, when a sales order is entered,
the order should be forwarded to the accounts receivable component,
and the sale automatic\-ally posted to the general ledger without
further manual entry.

How shall we organize our database? Let's begin by identifying the
entities and operations that are part of our business, and the reports
that we wish to obtain:

\begin{tabular}{|c|c|c|}
\hline 
\textbf{Entities} &
\textbf{Operations} &
\textbf{Reports}\tabularnewline
\hline 
\hline 
customers &
sales &
general ledger\tabularnewline
\hline 
vendors &
order entry &
income statement\tabularnewline
\hline 
employees &
accounts receivable &
balance sheet\tabularnewline
\hline 
 &
 &
payments received\tabularnewline
\hline 
 &
 &
purchases\tabularnewline
\hline 
 &
 &
purchase orders\tabularnewline
\hline 
 &
 &
accounts payable\tabularnewline
\hline 
 &
 &
checks written\tabularnewline
\hline 
 &
 &
payroll\tabularnewline
\hline 
\end{tabular}

Looking first at the left column, clearly we will need to store information
about the entities in a file structure. The question to ask is, ``What
do we need to know about these entities?'' It turns out that for
each of our three types of entity, the answer is remarkably similar.
In each case we need to know:

\texttt{\textbf{name}}~\\
\texttt{\textbf{address (street, city, state, zip)}}~\\
\texttt{\textbf{phone number}}

This observation suggestions the possibility of using shared code,
an oppor\-tunity for program simplification. At the very least, this
means we can use the same field definition names (\texttt{\textbf{NAME}},
\texttt{\textbf{STREET}}, etc.) for three different files.

In fact, though, we never have more than several hundred people and
companies that we do business with in any year. As a result, we can
mix all people and companies in single file, called \texttt{\textbf{PEOPLE}},
and add an extra field called \texttt{\textbf{KIND}} to indicate whether
the entity is a customer, vendor, or employee.

This reduces the number of files for ``entities'' from three to
one, and simplifies the program accordingly.

Because we will need to search and order this file on an alphabetical
basis, we must also create an index file, called \texttt{\textbf{(PEOPLE)}}.
This index will con\-tain simply a link field---to point to the corresponding
record in \texttt{\textbf{PEOPLE}}---and a ``nickname'' field, which
contains the name in a form that we want it alphabetized by.

We can establish the following rules for entry of the ``nickname''
field:

\textbf{for human beings}: last name, first name, initial

\textbf{for companies} (\textbf{customers or vendors}): company name\\
 (sometimes somewhat abbreviated)

As for the additional fields that employees need, we find it simplest
to create an additional file called \texttt{\textbf{AUXILIARY}}. Each
employee record contains a pointer to a record in \texttt{\textbf{AUXILIARY}}.

Now let's turn to the operations. Each operation results in a transaction
that must be saved. These transactions will become records in a file
of \textit{events}. What do we need to know about these events? In
the case of a sale, we have:

\texttt{\textbf{customer}}~\\
\texttt{\textbf{date of sale}}~\\
\texttt{\textbf{amount }}~\\
\texttt{\textbf{check number}}

In the case of a purchase order, we have:

\texttt{\textbf{vendor}}~\\
\texttt{\textbf{date of order}}~\\
\texttt{\textbf{amount}}~\\
\texttt{\textbf{purchase order number}}

In the case of payroll, we have:

\texttt{\textbf{employee}}~\\
\texttt{\textbf{date of paycheck}}~\\
\texttt{\textbf{amount}}~\\
\texttt{\textbf{check number}}~\\
\texttt{\textbf{commissions (for commissioned salespeople)}}~\\
\texttt{\textbf{tax contributions, etc.}}

Once again, it appears that many fields exist in common. With the
exception of the extra information needed for payroll, we can summarize
the above requirements as:

\texttt{\textbf{WHO}}~\\
\texttt{\textbf{WHEN}}~\\
\texttt{\textbf{AMOUNT}}~\\
\texttt{\textbf{NO.}}

We decide to keep all events in a single file. We will call this file
\texttt{\textbf{DETAIL}}. Besides the fields described above, we will
add a field called \texttt{\textbf{KIND}} to indicate whether the
event is a sale, an order, etc.

When we organize our data needs in this way, we see that entities
and events can be organized together for simplicity. With this understanding,
it will be easier to integrate the entire system.

Let's look at the \texttt{\textbf{WHO}} field. What should it contain?
Perhaps the name of the person or company.

On the other hand, we know there will be many more \texttt{\textbf{DETAIL}}
records than anything else, so we want to make each record as small
as possible. Were we to keep a name field in the \texttt{\textbf{DETAIL}}
file, it would take up considerable space and require that we look
up the name in an index in order to get the address or other information
on the name.

Instead, we will keep the record number of the related person or company
in the \texttt{\textbf{WHO}} field. This occupies only two bytes,
and requires no searches.

Now let's study some of the operations we'll want to perform. Suppose
it is the end of the month and time to write checks. This is easy.
We simply look through the \texttt{\textbf{DETAIL}} file looking for
accounts payable entries that are due now. From the record in \texttt{\textbf{DETAIL}}
we can follow the pointer into \texttt{\textbf{PEOPLE}} to get the
name and address of the payee.

Let's take another example. We want to be able to determine the current
balance owed by a particular customer or to a vendor. But we have
not included a ``Balance'' field in the \texttt{\textbf{PEOPLE}}
records. All we have to do is let each \texttt{\textbf{PEOPLE}} record
point to the most recent transaction, then let each transaction record
point to the next-most-recent transaction, etc. Here we are using
chains.

\begin{lstlisting}
ACME Widgets, Inc.                 Page 1 31 OCT 1986
        So. Bay Office Supply Account Status
#    Job  Ref  Due     DR#  CR# Amount   Paid Balance
3344     1086  29 NOV 5220 2100 189.24   0.00  189.24
3343  47  626  29 NOV 1210 2100  10.74   0.00   10.74
3205     2270  10 OCT 2100 1030 779.74 779.74    0.00
2773  61  930  30 SEP 1210 2100  59.04  59.04    0.00
\end{lstlisting}


Portion of a report showing a vendor account. The first column shows
the number by which each detail item is referenced; it is actually
the number of the record in the \texttt{\textbf{DETAIL}} file. The
report title is a `custom' one, showing the subject account. Custom
report titles are described in \prettyref{sub:Totals-and-Subtotals}.

Chaining is appropriate in cases such as this, in which there is no
way to predict how many elements there will be, and it makes it easy
to generate reports of activity for a vendor such as this.

There are at least three ways that chaining can be done:
\begin{enumerate}
\item Chaining from most recently entered transaction to least recent.
\item Chaining from least recent transaction to most recent.
\item Chaining by something other than order of entry, such as date field,
etc.
\end{enumerate}
In this case, we prefer to list transactions starting with the most
recent events. This makes possible reports such as shown in Fig. 2.16.
As we saw in Section 2.7, the polyFORTH Data Base Support package
includes a block of chain manipulation words that you can customize
for your particular application.

So far we have a \texttt{\textbf{PEOPLE}} file and a \texttt{\textbf{DETAIL}}
file. Now let us look at our desired reports.

The general ledger is produced monthly, organized by account. Under
each account are itemized all transactions, both credits and debits
involving that account during the month. In this balance sheet, we
show year-to-date summaries for each account.

\begin{lstlisting}
ACME Widgets, Inc.    Page 1 31 OCT 1986
           Balance Sheet
CURRENT ASSETS
  CASH
  Continental Bank        24,165
  Amalgamated Bank       104,965
  Short Term Investments 248,000
  Petty Cash               5,000 382,130
\end{lstlisting}


The traditional data base approach to General Ledger might involve
running, once each day, some program that looks through the latest
events and posts them to another file containing the general ledger
data. To produce the general ledger at month's end, this approach
would require sorting the transactions file by accounts.

But daily posting not only requires creating another file, it also
involves an extra step for the bookkeeper. And sorting would take
longer because it requires handling each record several times. As
we've mentioned, the polyFORTH Data Base Support package offers many
techniques that reduce the need to sort.

Still, each transaction record must indicate the account it affects.
When we produce the general ledger and income statements, we will
need to display accounts by name, \textit{e.g.} ``Advertising.''
It doesn't make sense to place an account name in each transaction
record. Following our dictum, ``Arrange like kinds of information
into files,'' we create a new file called \texttt{\textbf{ACCOUNTS}},
containing the name of each account. Now our transaction records can
simply use a record number to point to an account record.

Actually, each transaction affects two accounts, one as a debit and
one as a credit. So, each transaction really needs two pointers. We
will call one \texttt{\textbf{DR\#}} (debit record number) and the
other \texttt{\textbf{CR\#}} (credit record number). As each transaction
is made, we know what accounts are involved. For example, when we
enter an invoice, the amount will be credited as a sale and debited
as an account receivable.

\begin{figure}
\caption{\label{fig:17}\protect\includegraphics{16_home_druffer_Dropbox_ColorForth_File_Docs_Accounts.eps}}
\end{figure}


Chains can be used to link transactions affecting each general ledger
account.

Since each transaction points to a pair of account records, when displaying
a transaction we can also display the account names.

What about a general ledger file? Upon further analysis, we realize
that numbers don't need to be transferred to another file, when they
are in one file already. It will be easier to run a program once a
month that computes the account balances.

To run the general ledger, we need to start with each account, and
then look at the transactions that affected that account to produce
an account summary. We could loop through the \texttt{\textbf{ACCOUNTS}}
file, and for each account record, step through each transaction that
affected the current account during the month. This approach could
also use chains. Each \texttt{\textbf{ACCOUNTS}} record would point
to the most recent transaction affecting it, and each transaction
would point to the previous transaction affecting the same account
(Fig. \prettyref{fig:17}).

However, chaining is somewhat complicated, and better avoided whenever
possible. Rather than chaining from \texttt{\textbf{ACCOUNTS}} to
\texttt{\textbf{DETAIL}}, we can simply loop through our \texttt{\textbf{DETAIL}}
file for this month. Each \texttt{\textbf{DETAIL}} record points to
a pair of \texttt{\textbf{ACCOUNTS}} records. For each transaction,
we can \textit{add} the amount to an accumulator for a credit account,
and \textit{subtract} the amount from an accumulator for a debit account.
In this way, we can tally \textit{all} our account totals by looping
through the \texttt{\textbf{DETAIL}} file only once.

But where do we keep these accumulators? Since we need one and only
one for every account, it makes sense to add a field called \texttt{\textbf{BALANCES}}
to our \texttt{\textbf{ACCOUNTS}} records.

Is this idea really better than following chains? By following chains
from \texttt{\textbf{ACCOUNTS}} to \texttt{\textbf{DETAIL}}, we would
have to handle each transaction record twice: once while following
a credit-account chain, and once for a debit-account chain. By keeping
balances, we can loop through our transactions only once.

By using a one-pass posting algorithm with no chaining, we improve
performance a great deal by avoiding sorting, and by about a factor
of two by not using chains.

Our \texttt{\textbf{ACCOUNTS}} file can use some embellishments. In
addition to the two fields it already has:

\texttt{\textbf{Account No.}}~\\
\texttt{\textbf{Balance}}

we can add \texttt{\textbf{HISTORY}}, which is an array of balances
for the past 12 months.

In addition, the \texttt{\textbf{ACCOUNTS}} file needs an index, which
we will call \texttt{\textbf{(ACCOUNTS)}}. At first it would appear
that we could use the account numbers themselves to sort the accounts
when preparing the balancing statement. In fact, however, accountants
prefer to sub classify accounts into groups for their own reasons.
For instance, taxes are an expense account, but they are usually listed
at the end of the list of expense accounts. For this reason, the \texttt{\textbf{(ACCOUNTS)}}
file is numbered according to the order in which we want accounts
to appear on the balance sheet.

Our next step is to write words that reflect the kinds of high-level
actions the bookkeepers want to record. Let's start with the operation
of placing an order. How must this order affect our database? What
do we need to know?

Clearly we are going to create a new \texttt{\textbf{DETAIL}} record.
This record will include a \texttt{\textbf{WHO}} field to indicate
the company from which we are ordering. Since we have our vendors
in the \texttt{\textbf{PEOPLE}} ordered index file, we need supply
only the name of the company. The program can then look up the company,
find the record number and place it in the \texttt{\textbf{WHO}} field
of the new transaction record. The program must also link this new
transaction into the chain for that vendor.

We also need to supply the amount of the purchase, and our purchase-order
number.

The program itself can place the current system date into the \texttt{\textbf{WHEN}}
field, and by default, place the date 30 days hence into the \texttt{\textbf{DUE}}
field. Since this is an order, the program must place the code for
a purchase in the \texttt{\textbf{KIND}} field.

So what should our ``program'' for entering an order look like to
the bookkeeper? We know the bookkeeper must supply:
\begin{enumerate}
\item The amount.
\item The purchase-order number.
\item The name of the vendor.
\end{enumerate}
The simplest, most Forth-like solution is to call the word \texttt{\textbf{BOUGHT}},
precede it with the two numeric data items and follow it with the
string data. This gives us the syntax:

\texttt{\textbf{200.00 5134 BOUGHT ACME}}

We can now take a similar approach with a program to record a sale:

\texttt{\textbf{3998.00 7409 SOLD CROFT}}

The word \texttt{\textbf{SOLD}} is preceded by amount and their purchase-order
number, and followed by the name of the customer.

We can record the receipt of a check with the word \texttt{\textbf{FROM}}:

\texttt{\textbf{amount check\# line\# FROM Conway}}

In the above, \texttt{\textbf{line\#}} is a number that identifies
the sale for which this is a payment received. The bookkeeper finds
this number on a report of outstanding balances. While this is simple
for the bookkeeper, it is also simple for the program because \texttt{\textbf{line\#}}
just happens to be the record number of the \texttt{\textbf{DETAIL}}
record showing the sale.

The same syntax can be used for writing a check:

\texttt{\textbf{amount check\# line\# TO ROSS}}

Thus, each ``program'' is simply a Forth word. This approach allows
our application to use the Forth interpreter. The problem of how the
bookkeeper selects a given operation is effectively eliminated.

To appreciate the significance of this, consider the typical alternative.
Most business applications are menu-based. From the main menu, the
bookkeeper might select Accounts Payable. Then, from the Accounts
Payable menu the bookkeeper might choose Purchase. From there, an
entry form might appear, wherein the bookkeeper can select or enter
the customer, and then fill in the data.

While popular, this menu-based approach can be more laborious for
the user. To avoid the switching application modes, the bookkeeper
may separate all the purchases from the sales, etc., and do each group
one at a time. This requires more paper shuffling.

Our approach, with no hierarchy, lets the user enter various transactions
in any order, leading to a more pleasant, efficient working environment.
A ``help screen'' can display the syntax of the commands on request
during the learning curve.

In retrospect, we seem to have designed the database very efficiently.
The file with the most records, \texttt{\textbf{DETAIL}}, also has
the smallest records. Each record in \texttt{\textbf{DETAIL}} is only
16 bytes long, and contains no text at all. (This means that 64 such
records will fit in a block.) 


\subsubsection{A Facility Management System \label{sub:A-Facility-Management-System}}

In this example we will see how to organize and simplify a massive
data problem by studying the data and looking for a natural hierarchy.

The example involves the problem of controlling digital and analog
input/ output with a distributed computer system, where there are
several thou\-sand I/O points in dozens of buildings and other locations
at a large industrial plant.

Digital ``points'' include switches, buttons, pressure-sensitive
floor plates, pulses to unlock doors, and so on. Analog points include
thermocouples, meters on control panels, heating levels, lighting
levels, and so on. Our task is to install a distributed computer system
to control all these points.

We begin by studying the points as the architects and engineers designated
them. The ID for an individual point has the form:

\texttt{\textbf{ABC-123-1234}}

Experience has taught us that numbers such as this are usually encoded,
and that usually the coding scheme presents a goldmine of information
on how to organize the system. Upon further investigation we discover
these relationships:

\texttt{\textbf{ABC - 123 - 1234}}

\texttt{\textbf{a facility, a control panel, a point number on a}}~\\
\texttt{\textbf{ control panel in a facility.}}

This information provides the key for our establishing a hierarchical
database, a necessary strategy when dealing with thousands of anything.
Another example in which coded numbers can reveal hierarchy is with
inventory or parts numbers.

\textbf{}
\begin{figure}
\textbf{\caption{\textbf{\label{fig:18}\protect\includegraphics{17_home_druffer_Dropbox_ColorForth_File_Docs_Facilities.eps}}}
}
\end{figure}


\textbf{Diagram of the database for the Facility Management example.}

One of the things we must do is allow the control of individual points
from the central computer. The operator can enter a point ID and indicate
some action. So one of our problems is to associate a point ID with
a physical hardware location. One approach would be to have a points
index containing the ID of each point in the system. The nature of
the coding scheme makes it a candidate for an ordered index, but with
over 20,000 records even a binary search is more cumbersome than we
can afford. The logical hierarchy will help deal with the volume,
and thus the performance. It will also improve the information content;
for example, in reports we can make use of the implicit meaning of
the data.

The first part of the code is a ``facility:'' A physical building
or location on the plant, like the parking garage, the fire station,
etc. A ``panel'' may be a manually attended console; a switchbox
in a closet, or it may be imagin\-ary, as in the case of points attacked
directly to the computer.

A single panel may control many points. And a computer may control
many points in many buildings (for instance, the fire station computer
needs fire alarms in all buildings). But what is the relationship
between computers, facilities, panels and points?

Further digging uncovers fact that a panel is attached to only one
computer. This means that each computer can be responsible for its
own private database of panels. Each panel can be chained to any number
of points. The master computer can have a file of facilities. Each
facility can be chained to any number of panels.

We can now identify four files:

\texttt{\textbf{FACILITIES}}~\\
\texttt{\textbf{COMPUTERS}}~\\
\texttt{\textbf{PANELS}}~\\
\texttt{\textbf{POINTS}}

We have solved the problem of chaining from computers down to points.
Now let's consider the reverse problem. Inputs generate ``events.''
An event has to be dealt with quickly; for instance, the event may
be an alarm. If a fire alarm is triggered, a computer will need to
display information about the point, such as which building it is
in. In other words, we must have linkage from point to panel to building.

The first step of this linkage is the association between an electrical
event within the computer and the corresponding point in the \texttt{\textbf{POINTS}}
file. For instance, the pressing of a button might cause execution
of an interrupt routine. This routine must be able to determine which
point caused the interrupt.

At first it may seem logical to keep a table that associates hardware
addresses with point names. But this would require an extra search.
It is more direct to create a table that associates points' hardware
addresses with record numbers within the \texttt{\textbf{POINTS}}
file. Another benefit is that the point record number only requires
two bytes, so the table is small.

With this scheme, an electrical event is associated with a record
in \texttt{\textbf{POINTS}}, which in turn contains the information
we need to know about the point, including its code name. Because
the code name contains meaningful information, we can now determine
which building the point is in.

We can now rest assured that we have found a good solution to the
problem, since we have achieved good performance while at the same
time reducing complexity. This sort of win/win situation provides
the positive feedback that tells us we're on the right track as we
iterate through our design. 


\subsubsection{A Filing Scheme for Image Processing Applications \label{sub:A-Filing-Scheme-for-Image-Processing-Applications}}

Our final example illustrates the flexibility the Data Base Support
package provides---including the freedom to \textit{not} use some
of its features when the application dictates otherwise!

Conceptually, a filed image has two elements, a header and the pixel
data. The header indicates what the image is, when it was recorded,
who made the image, the dimensions of the image (in pixels), and so
on.

There are a variety of ways to index into images. But the real problem
is managing the pixel data. Image processing is a prime example of
an application in which speed is critical, because there is simply
so much pixel data to handle. An array of 512x512 points contains
262,144 pixels, which at 8 bytes per pixel occupies 256 blocks. Just
reading this many blocks will take some time.

Now imagine trying to access these pixels one at a time using \texttt{\textbf{1@}}
(or\texttt{\textbf{ N@}}). This approach involves the invocation of
\texttt{\textbf{BLOCK}} plus the record and field accessing computations
for each and every pixel. This will be unacceptably slow.

An approach that has proven effective is an interesting hybrid of
the Data Base Support package tools, plus ordinary direct disk-access
techniques. In this approach, we use the Data Base Support commands
for header infor\-mation, but we keep the pixel data elsewhere on
the disk. In other words, we reserve three regions on the disk: a
file for headers, another for an index to our headers, and a region
of blocks that are \textit{not} files for pixel data. Within the header,
a field points to the \textit{block number} where the pixel data begins
for that image. Another field indicates how many blocks are used.

We also recommend keeping the data in the form used by the image-processing
device (usually binary integers). You may want to process an image
using floating point (although in the absence of a hardware floating
point processor the fixed-point routines supplied with polyFORTH will
be much faster). But a 64-bit floating point number is eight bytes
long, which means an image will require eight times as many blocks
and take eight times as long to read and write off the disk. It is
faster to float the numbers after fetching them.

Some users believe that saving pixel data in floating point form retains
better resolution. In fact, however, the typical A/D converter on
a Vidicon camera (for instance) does not possess many bits of resolution.
In industrial vision applications, these devices rarely provide more
than one byte of precision. The extra bits that floating point provides
simply represent noise.

On the other hand, some applications do utilize greater precision,
but store a much smaller number of pixels. In astronomy, for example,
an image size may be only 64x64. But the image might be recorded with
a highly sensitive detector over a four-hour period with atmospheric
correction. Thus, each pixel has already been integrated and may contain
as 16 bits or more of information. 


\subsection{Glossary Utility \label{sub:Glossary-Utility}}

The \texttt{\textbf{Glossary}} is a Forth utility that allows maintenance
of a file that contains descriptions of Forth words. This provides
a convenient way to document polyFORTH programs. It is also an excellent
example of the use of all data base management features.

Each word defined in the glossary has the following information associated
with it:
\begin{enumerate}
\item The block in which the word is defined.
\item The glossary vocabulary.
\item Stack usage.
\item One or more lines of text that describe the word and its use.
\end{enumerate}
The \texttt{\textbf{Glossary}} utility provides commands to maintain
this file and to print reports that include either selected glossary
vocabularies or the complete file of words. 

\nwenddocs{}\nwbegincode{267}\sublabel{NWtG4Lr-2RYM6N-1}\nwmargintag{{\nwtagstyle{}\subpageref{NWtG4Lr-2RYM6N-1}}}\moddef{glossary~{\nwtagstyle{}\subpageref{NWtG4Lr-2RYM6N-1}}}\endmoddef\nwstartdeflinemarkup\nwusesondefline{\\{NWtG4Lr-2V5MiK-1}}\nwenddeflinemarkup
: HELP ( S: -- ) ( G: This is the  GLOSSARY  help screen )
 CR PAGE ." HELP                 To display this  GLOSSARY  application"
      CR ."    This utility uses a disk file named Glossary.dbf"
      CR ." SUMMARY              Displays all words in the file."
      CR ." /VOCABULARY _____    Displays all words in vocabulary _____ ."
      CR ." VOCAB _____          Specifies vocabulary for entries & searches"
      CR ." n SOURCE 2 nC!       Specifies source block  n  for entries."
      CR ." NEW _____            Enters  ____  making it current."
      CR ."    Thereafter use  U  to enter descriptive text."
      CR ." FIND _____           Locates and shows _____ making it current."
      CR ." F                    Re-displays the current word."
      CR ." n AT                 Changes the current word's source to  n ."
      CR ." STACKS               Enter 'input' and 'output' stack arguments."
      CR ." l T  ( or  t)        Types line  l (0-n)  making it current."
      CR ." P  ( or  p ) _____   Puts  _____  on the current line."
      CR ." U  ( or  u ) _____   Inserts  _____  under the current line."
      CR ." The date used for today's entries will be "  DATE ;

\LA{}glossary-file~{\nwtagstyle{}\subpageref{NWtG4Lr-1yjrvH-1}}\RA{}
\LA{}vocab~{\nwtagstyle{}\subpageref{NWtG4Lr-26DLU4-1}}\RA{}
\LA{}f~{\nwtagstyle{}\subpageref{NWtG4Lr-1EuC3M-1}}\RA{}
\LA{}stacks~{\nwtagstyle{}\subpageref{NWtG4Lr-4Agpfb-1}}\RA{}
\LA{}new~{\nwtagstyle{}\subpageref{NWtG4Lr-3OC4Hk-1}}\RA{}
\LA{}delete~{\nwtagstyle{}\subpageref{NWtG4Lr-3WJOq3-1}}\RA{}
\LA{}at~{\nwtagstyle{}\subpageref{NWtG4Lr-100f20-1}}\RA{}
\LA{}t~{\nwtagstyle{}\subpageref{NWtG4Lr-zn3NQ-1}}\RA{}
\LA{}p~{\nwtagstyle{}\subpageref{NWtG4Lr-CgRsG-1}}\RA{}
\LA{}u~{\nwtagstyle{}\subpageref{NWtG4Lr-3Zsojo-1}}\RA{}
\LA{}x~{\nwtagstyle{}\subpageref{NWtG4Lr-OSnlw-1}}\RA{}
\LA{}find~{\nwtagstyle{}\subpageref{NWtG4Lr-MECIZ-1}}\RA{}
\LA{}moved~{\nwtagstyle{}\subpageref{NWtG4Lr-1nqpD2-1}}\RA{}
\LA{}summary~{\nwtagstyle{}\subpageref{NWtG4Lr-3d0IxR-1}}\RA{}
\LA{}vocabulary~{\nwtagstyle{}\subpageref{NWtG4Lr-424Fqv-1}}\RA{}
\nwused{\\{NWtG4Lr-2V5MiK-1}}\nwendcode{}\nwbegindocs{268}\nwdocspar


\subsubsection{File Structure }

The \texttt{\textbf{Glossary}} uses one or more glossary files that
are specified by the user. Each glossary file is physically composed
of two separate files; a data file and an index file. Index file support
in the polyFORTH system is required to implement the glossary.

You must pre-allocate the two polyFORTH files required for a glossary.
The data file is composed of 64-byte records, several of which may
be chained together to provide multiple lines of text, 64 characters
per line. It is named \texttt{\textbf{GLOSSARY}}.

The index file used for the glossary is compiled of 28-byte records,
with a 24-byte key length comprised of the word name\ (12 bytes)
and a vocabulary name (12 bytes). One index record is required for
each glossary entry. The name of the index file is \texttt{\textbf{(GLOSSARY)}}.
It is an ordered index, ordered by word name and vocabulary; this
has the effect of maintaining the glossary in alphabetic order.

Here is a sample definition for a glossary file that contains 450
entries.

\nwenddocs{}\nwbegincode{269}\sublabel{NWtG4Lr-1yjrvH-1}\nwmargintag{{\nwtagstyle{}\subpageref{NWtG4Lr-1yjrvH-1}}}\moddef{glossary-file~{\nwtagstyle{}\subpageref{NWtG4Lr-1yjrvH-1}}}\endmoddef\nwstartdeflinemarkup\nwusesondefline{\\{NWtG4Lr-2RYM6N-1}}\nwenddeflinemarkup
FILE Glossary.dbf

: /Glossary ( -- ) \\ Runtime file setup
    Glossary.dbf  S" Glossary.dbf" >FILE
    TRUE REVERSE !  LITTLE-ENDIAN ;

( Bytes  records  origin              name )
   28      432    0       BLOCK-DATA (GLOSSARY)
   64     2300   +ORIGIN  BLOCK-DATA  GLOSSARY

4 ( LINK )  24 BYTES NAME+VOC  DROP   ( Index )

4 ( LINK )  12 BYTES WORD-NAME
            12 BYTES VOC
            16 BYTES BEFORE
            16 BYTES AFTER
             NUMERIC SOURCE
             NUMERIC ENTERED  DROP

4 ( LINK )  60 BYTES PHRASE  DROP
\nwused{\\{NWtG4Lr-2RYM6N-1}}\nwendcode{}\nwbegindocs{270}\nwdocspar


\subsubsection{Loading Instructions \label{sub:Loading-Instructions}}

As with the other examples, the \texttt{\textbf{Glossary}} is loaded
automatically with the pFDatabase.fth

The procedure for entering a word into a \texttt{\textbf{Glossary}}
consists of making the block number and glossary vocabulary current,
entering the stack usage and the word-name, and then entering associated
text. You can change block number, stack usage, and text lines easily.
The following sections are interdependent; reading through them at
one sitting will pro\-vide a helpful overview. 


\subsubsection{Source Block Identification \label{sub:Source-Block-Identification}}

When you begin to document your application, you will usually specify
a source block to be documented and then enter all the words that
are defined in that block.

To specify a source block, use the following phrase:

\texttt{\textbf{blk\# SOURCE 2 nC!}}

Until changed by re-use of the phrase above, this current block number
will automatically be stored with each succeeding word entry.

\textbf{REFERENCES}

Entry Changes, \prettyref{sub:Changes} 


\subsubsection{Glossary Vocabulary Identification \label{sub:Glossary-Vocabulary-Identification}}

Along with each word, the system stores the name of the entry's applica\-tion
vocabulary. This usually means the name of the portion of the appli\-cation
in which the word is used, such as the name of its load block. These
vocabularies are not necessarily the same as program vocabularies.
Gloss\-ary vocabularies exist only for logical grouping of words
and to enable the same word to be variously defined several times
in different blocks.

Before you begin entering words for a new glossary vocabulary, make
it the current vocabulary by typing:

\texttt{\textbf{VOCAB vocabulary-name}}

\nwenddocs{}\nwbegincode{271}\sublabel{NWtG4Lr-26DLU4-1}\nwmargintag{{\nwtagstyle{}\subpageref{NWtG4Lr-26DLU4-1}}}\moddef{vocab~{\nwtagstyle{}\subpageref{NWtG4Lr-26DLU4-1}}}\endmoddef\nwstartdeflinemarkup\nwusesondefline{\\{NWtG4Lr-2RYM6N-1}}\nwenddeflinemarkup
: VOCAB ( -- )   BL TEXT  VOC S! ;
\nwused{\\{NWtG4Lr-2RYM6N-1}}\nwendcode{}\nwbegindocs{272}\nwdocspar

Note that the name cannot be longer than ten characters. Until changed,
this name is kept in memory and copied into each succeeding data record
entered.

In order to search for a previously entered word, you must make its
vocabulary the current one.

The glossary vocabulary name serves as a secondary key for searches.
This means that the same word may be entered in numerous vocabularies,
with each entry unique.

The vocabulary is also set by the report command \texttt{\textbf{/VOCABULARY}}.

\textbf{REFERENCES}

\texttt{\textbf{/VOCABULARY}}, \prettyref{sub:Definition-Display}\\
Finding Previously Entered Words, \prettyref{sub:Changes} 


\subsubsection{Glossary Entries \label{sub:Glossary-Entries}}

Words are entered into the glossary through the \texttt{\textbf{NEW}}
command. This command sets the basic entry into the file. It has the
following format:

\texttt{\textbf{NEW word-name (}}\texttt{\textbf{\textit{e.g.}}}\texttt{\textbf{,
NEW\ NAME)}}

The program will prompt you for brief (16-character) descriptions
of stack entries before and after execution. Any valid Forth word
name may be used; the maximum length recognized by the \texttt{\textbf{Glossary}}
is twelve characters. If a longer word name is entered, its length
will be truncated to twelve characters.

\nwenddocs{}\nwbegincode{273}\sublabel{NWtG4Lr-3OC4Hk-1}\nwmargintag{{\nwtagstyle{}\subpageref{NWtG4Lr-3OC4Hk-1}}}\moddef{new~{\nwtagstyle{}\subpageref{NWtG4Lr-3OC4Hk-1}}}\endmoddef\nwstartdeflinemarkup\nwusesondefline{\\{NWtG4Lr-2RYM6N-1}}\nwenddeflinemarkup
: (FIND) ( -- n a ) \\ Usage: (FIND) <name>
    BL TEXT WORD-NAME S!  (GLOSSARY) NAME+VOC ;

: (ENTER) ( -- ) \\ Usage: (ENTER) <name>
    (FIND) -BINARY IF
        SAVE  GLOSSARY SLOT  DUP HEAD !  LINK !  RESTORE
        NAME+VOC S@  +ORDERED  NAME+VOC S!
        GLOSSARY FIRST  NAME+VOC B!
        SOURCE 2 nC@  SOURCE N!
    ELSE  ORDERED RELEASE  LINK L@  GLOSSARY READ
    THEN  @DATE ENTERED N! ;

: NEW ( -- ) \\ Usage: NEW <name>
    (ENTER)  STACKS  CR  64 SPACES  0 #L ! ;
\nwused{\\{NWtG4Lr-2RYM6N-1}}\nwendcode{}\nwbegindocs{274}\nwdocspar

Following \texttt{\textbf{NEW}}, the new word is made the current
word, with which will be stored the current block number, current
glossary vocabulary name, and up to four lines of associated text.
Immediate subsequent use of \texttt{\textbf{AT}}, \texttt{\textbf{STACKS}},
\texttt{\textbf{T}}, \texttt{\textbf{U}}, or \texttt{\textbf{P}} will
affect this entry.

\textbf{REFERENCES}

Entry Changes, \prettyref{sub:Changes}\\
\texttt{\textbf{T}} and \texttt{\textbf{P}}, \prettyref{sub:Text-Specification} 


\subsubsection{Text Specification \label{sub:Text-Specification}}

The \texttt{\textbf{Glossary}} provides commands that allow up to
four lines of text to be associated with each entry and also allow
modification of previously entered text.

The following command is used to enter a line of text that is associated
with a definition:

\texttt{\textbf{U new text line}}

The command U inserts ``new text line'' under the current text line
(which begins at 0 after a new entry). The new text line may be composed
of one to 64 characters, including embedded blanks.

\nwenddocs{}\nwbegincode{275}\sublabel{NWtG4Lr-3Zsojo-1}\nwmargintag{{\nwtagstyle{}\subpageref{NWtG4Lr-3Zsojo-1}}}\moddef{u~{\nwtagstyle{}\subpageref{NWtG4Lr-3Zsojo-1}}}\endmoddef\nwstartdeflinemarkup\nwusesondefline{\\{NWtG4Lr-2RYM6N-1}}\nwenddeflinemarkup
: U ( -- ) \\ Usage: U text
    #L @  CHAIN P  1 #L +! ;
\nwused{\\{NWtG4Lr-2RYM6N-1}}\nwendcode{}\nwbegindocs{276}\nwdocspar

Following the use of \texttt{\textbf{NEW}} or \texttt{\textbf{FIND}},
the current text line is initialized to zero. Use of \texttt{\textbf{U}}
not only inserts a new text line, it also increments the current line
number. Thus subsequent usage of \texttt{\textbf{U}} adds additional
text lines.

The command \texttt{\textbf{P}} is used to modify existing text. You
do this by displaying the line to be changed and then using \texttt{\textbf{P}}
to replace the old text with new text. Remember that you can only
work on the current word in the current vocabulary. You display the
appropriate line of text (lines are numbered starting from zero) by
typing:

\texttt{\textbf{line\# T (}}\texttt{\textbf{\textit{e.g.}}}\texttt{\textbf{,
3\ T to display the fourth line)}}

\nwenddocs{}\nwbegincode{277}\sublabel{NWtG4Lr-zn3NQ-1}\nwmargintag{{\nwtagstyle{}\subpageref{NWtG4Lr-zn3NQ-1}}}\moddef{t~{\nwtagstyle{}\subpageref{NWtG4Lr-zn3NQ-1}}}\endmoddef\nwstartdeflinemarkup\nwusesondefline{\\{NWtG4Lr-2RYM6N-1}}\nwenddeflinemarkup
: T ( n -- )    1+ DUP -LOCATE ABORT" Not there"  #L !  CR
    2 SPACES  PHRASE B? ;
\nwused{\\{NWtG4Lr-2RYM6N-1}}\nwendcode{}\nwbegindocs{278}\nwdocspar

After a line of text has been displayed, you can modify it by using
the following command:

\texttt{\textbf{P replacement-text-line}}

\nwenddocs{}\nwbegincode{279}\sublabel{NWtG4Lr-CgRsG-1}\nwmargintag{{\nwtagstyle{}\subpageref{NWtG4Lr-CgRsG-1}}}\moddef{p~{\nwtagstyle{}\subpageref{NWtG4Lr-CgRsG-1}}}\endmoddef\nwstartdeflinemarkup\nwusesondefline{\\{NWtG4Lr-2RYM6N-1}}\nwenddeflinemarkup
: P ( -- ) \\ Usage: P text
    1 TEXT  PHRASE B! ;
\nwused{\\{NWtG4Lr-2RYM6N-1}}\nwendcode{}\nwbegindocs{280}\nwdocspar

The command \texttt{\textbf{X}} is used to delete a text line previously
selected by the \texttt{\textbf{T}} command. Thus, to delete Line
2 you would type:

\texttt{\textbf{2 T}}~\\
\texttt{\textbf{ X}}

\nwenddocs{}\nwbegincode{281}\sublabel{NWtG4Lr-OSnlw-1}\nwmargintag{{\nwtagstyle{}\subpageref{NWtG4Lr-OSnlw-1}}}\moddef{x~{\nwtagstyle{}\subpageref{NWtG4Lr-OSnlw-1}}}\endmoddef\nwstartdeflinemarkup\nwusesondefline{\\{NWtG4Lr-2RYM6N-1}}\nwenddeflinemarkup
: X ( -- )   #L @ 1- UNCHAIN ;
\nwused{\\{NWtG4Lr-2RYM6N-1}}\nwendcode{}\nwbegindocs{282}\nwdocspar

You cannot delete the last remaining descriptive line, but you can
change it. All others may be deleted.

\textbf{REFERENCES}

\texttt{\textbf{NEW}}, \prettyref{sub:Glossary-Entries}\\
\texttt{\textbf{FIND}}, \prettyref{sub:Changes} 


\subsubsection{Definition Display \label{sub:Definition-Display}}

To display the current entry, type:

\texttt{\textbf{F}}

\nwenddocs{}\nwbegincode{283}\sublabel{NWtG4Lr-1EuC3M-1}\nwmargintag{{\nwtagstyle{}\subpageref{NWtG4Lr-1EuC3M-1}}}\moddef{f~{\nwtagstyle{}\subpageref{NWtG4Lr-1EuC3M-1}}}\endmoddef\nwstartdeflinemarkup\nwusesondefline{\\{NWtG4Lr-2RYM6N-1}}\nwenddeflinemarkup
: (SHOW) ( -- )    FIRST  WORD-NAME ?B  VOC ?B  BEFORE ?B  AFTER ?B
    SOURCE ?N  ENTERED NU@  .M/D/Y  BEGIN
        +L  -NEXT 0= WHILE
            10 SPACES  PHRASE B?
    REPEAT ;

[R FORTH GLOSSARY\\ NAME        \\VOCABULARY  \\INPUT           \\OUTPUT          \\BLK\\      ENTERED]
    CONSTANT GLOSSARY-TITLE

VARIABLE #L

: F ( -- )    GLOSSARY-TITLE HEADING  FIRST (SHOW)  1 #L ! ;
\nwused{\\{NWtG4Lr-2RYM6N-1}}\nwendcode{}\nwbegindocs{284}\nwdocspar

To print all definitions in all the vocabularies in the glossary,
in ASCII alphabetical sequence, use the word \texttt{\textbf{SUMMARY}}.
The same information as for \texttt{\textbf{FIND}} is printed for
each word entered in the glossary. The printed report is paged and
numbered.

\nwenddocs{}\nwbegincode{285}\sublabel{NWtG4Lr-3d0IxR-1}\nwmargintag{{\nwtagstyle{}\subpageref{NWtG4Lr-3d0IxR-1}}}\moddef{summary~{\nwtagstyle{}\subpageref{NWtG4Lr-3d0IxR-1}}}\endmoddef\nwstartdeflinemarkup\nwusesondefline{\\{NWtG4Lr-2RYM6N-1}}\nwenddeflinemarkup
: SUMMARY ( -- )    GLOSSARY-TITLE LAYOUT
    (GLOSSARY) RECORDS DO
        I GLOSS IF
            +L (SHOW)
    THEN  LOOP  SPACE ;
\nwused{\\{NWtG4Lr-2RYM6N-1}}\nwendcode{}\nwbegindocs{286}\nwdocspar

The command:

\texttt{\textbf{/VOCABULARY}}

will print definitions as for \texttt{\textbf{SUMMARY}} but only in
the glossary vocabulary whose name is specified.

\nwenddocs{}\nwbegincode{287}\sublabel{NWtG4Lr-424Fqv-1}\nwmargintag{{\nwtagstyle{}\subpageref{NWtG4Lr-424Fqv-1}}}\moddef{vocabulary~{\nwtagstyle{}\subpageref{NWtG4Lr-424Fqv-1}}}\endmoddef\nwstartdeflinemarkup\nwusesondefline{\\{NWtG4Lr-2RYM6N-1}}\nwenddeflinemarkup
: /VOCABULARY ( -- ) \\ Usage: \\VOCABULARY <name>
    VOCAB  GLOSSARY-TITLE LAYOUT  (GLOSSARY) RECORDS DO
        I GLOSS IF
            VOC  SWAP OVER ADDRESS -TEXT 0= IF
                +L (SHOW)
    THEN  THEN  LOOP  SPACE ;
\nwused{\\{NWtG4Lr-2RYM6N-1}}\nwendcode{}\nwbegindocs{288}\nwdocspar

\textbf{REFERENCES}

Glossary Vocabularies, \prettyref{sub:Glossary-Vocabulary-Identification}
\\
Making an Entry Current, \prettyref{sub:Changes} 


\subsubsection{Changes \label{sub:Changes}}

Changes always affect the current word. Words are made current in
two ways.
\begin{enumerate}
\item A word just entered is the current word.
\item A previously entered word in the current vocabulary may be made current
by using the command:\textbf{}\\
\texttt{\textbf{FIND word-name}}\textbf{}\\
This displays the requested word, with its vocabulary name, block
number, stack usage, and text description.
\end{enumerate}
Following the use of \texttt{\textbf{NEW}} or \texttt{\textbf{FIND}},
the current line number is initialized to zero.

\nwenddocs{}\nwbegincode{289}\sublabel{NWtG4Lr-MECIZ-1}\nwmargintag{{\nwtagstyle{}\subpageref{NWtG4Lr-MECIZ-1}}}\moddef{find~{\nwtagstyle{}\subpageref{NWtG4Lr-MECIZ-1}}}\endmoddef\nwstartdeflinemarkup\nwusesondefline{\\{NWtG4Lr-2RYM6N-1}}\nwenddeflinemarkup
: FIND ( -- ) \\ Usage: FIND <name>
    (FIND) BINARY HEAD !  GLOSSARY F ;
\nwused{\\{NWtG4Lr-2RYM6N-1}}\nwendcode{}\nwbegindocs{290}\nwdocspar

The current word's stack entries may be changed by typing:

\texttt{\textbf{STACKS}}

\nwenddocs{}\nwbegincode{291}\sublabel{NWtG4Lr-4Agpfb-1}\nwmargintag{{\nwtagstyle{}\subpageref{NWtG4Lr-4Agpfb-1}}}\moddef{stacks~{\nwtagstyle{}\subpageref{NWtG4Lr-4Agpfb-1}}}\endmoddef\nwstartdeflinemarkup\nwusesondefline{\\{NWtG4Lr-2RYM6N-1}}\nwenddeflinemarkup
: STACKS ( -- )   FIRST
    CR  ." Before: "  BEFORE ASK
    CR ." After: "  AFTER ASK  F ;
\nwused{\\{NWtG4Lr-2RYM6N-1}}\nwendcode{}\nwbegindocs{292}\nwdocspar

The current word's source-block\# may be changed by typing:

\texttt{\textbf{new-blk\# AT}}

\nwenddocs{}\nwbegincode{293}\sublabel{NWtG4Lr-100f20-1}\nwmargintag{{\nwtagstyle{}\subpageref{NWtG4Lr-100f20-1}}}\moddef{at~{\nwtagstyle{}\subpageref{NWtG4Lr-100f20-1}}}\endmoddef\nwstartdeflinemarkup\nwusesondefline{\\{NWtG4Lr-2RYM6N-1}}\nwenddeflinemarkup
: AT ( n -- )   FIRST  SOURCE N! ;
\nwused{\\{NWtG4Lr-2RYM6N-1}}\nwendcode{}\nwbegindocs{294}\nwdocspar

All of the words defined on a source block\# can be moved by typing:

\texttt{\textbf{old-blk\# new-mlk\# MOVED}}

\nwenddocs{}\nwbegincode{295}\sublabel{NWtG4Lr-1nqpD2-1}\nwmargintag{{\nwtagstyle{}\subpageref{NWtG4Lr-1nqpD2-1}}}\moddef{moved~{\nwtagstyle{}\subpageref{NWtG4Lr-1nqpD2-1}}}\endmoddef\nwstartdeflinemarkup\nwusesondefline{\\{NWtG4Lr-2RYM6N-1}}\nwenddeflinemarkup
: GLOSS ( n -- t )    (GLOSSARY) READ  LINK L@ DUP 0> IF
        HEAD !  GLOSSARY FIRST TRUE
    ELSE  DROP 0  THEN ;

: MOVED ( s d -- )    SWAP  (GLOSSARY) RECORDS DO
        I GLOSS DROP  SOURCE N@ OVER = IF
            OVER SOURCE N!
    THEN  LOOP  2DROP ;
\nwused{\\{NWtG4Lr-2RYM6N-1}}\nwendcode{}\nwbegindocs{296}\nwdocspar

You may not change vocabulary and word names except by deleting and
re-entering the entry, since these two items form the index keys.

To redisplay the complete entry for the current word, type:

\texttt{\textbf{F}}

\textbf{REFERENCES}

Changing Description Lines, \prettyref{sub:Text-Specification}\\
Making a Vocabulary Current, \prettyref{sub:Glossary-Vocabulary-Identification} 


\subsubsection{Text and Definition Deletion \label{sub:Text-and-Definition-Deletion}}

The following command is used to remove all text lines associated
with the current definition and then to delete the current definition
from the glossary:

\texttt{\textbf{DELETE word-name}}

\nwenddocs{}\nwbegincode{297}\sublabel{NWtG4Lr-3WJOq3-1}\nwmargintag{{\nwtagstyle{}\subpageref{NWtG4Lr-3WJOq3-1}}}\moddef{delete~{\nwtagstyle{}\subpageref{NWtG4Lr-3WJOq3-1}}}\endmoddef\nwstartdeflinemarkup\nwusesondefline{\\{NWtG4Lr-2RYM6N-1}}\nwenddeflinemarkup
: DELETE ( -- )    (FIND)  BINARY
    ORDERED GRAB  -ORDERED
    GLOSSARY  BEGIN
        READ  LINK 0 SNATCH  DUP 0<
    UNTIL  DROP ;
\nwused{\\{NWtG4Lr-2RYM6N-1}}\nwendcode{}\nwbegindocs{298}\nwdocspar


\section{Testing}

Part of the motivation for converting this tool into a Literate Programming
document was the discovery that there was some problem using it under
a 64-bit Forth. That led to the discovery of problems with the existing
samples even in 32-bit Forths. I have loaded all of the sames in the
\texttt{\textbf{pFDatabase.fth}} source file (see \prettyref{sub:Installing-The-Package}),
so it is a simple matter to run their display routines. I am only
using \texttt{\textbf{gforth}} right now, and it has a built-in \texttt{\textbf{assert(
... )}} function. So I will use that right now, until I need to worry
about doing something that is more portable.

\nwenddocs{}\nwbegincode{299}\sublabel{NWtG4Lr-Joiug-1}\nwmargintag{{\nwtagstyle{}\subpageref{NWtG4Lr-Joiug-1}}}\moddef{test~{\nwtagstyle{}\subpageref{NWtG4Lr-Joiug-1}}}\endmoddef\nwstartdeflinemarkup\nwusesondefline{\\{NWtG4Lr-2V5MiK-1}}\nwenddeflinemarkup
\LA{}fits~{\nwtagstyle{}\subpageref{NWtG4Lr-3C1YNj-1}}\RA{}
\LA{}text=~{\nwtagstyle{}\subpageref{NWtG4Lr-3PN52u-1}}\RA{}
\LA{}-linked~{\nwtagstyle{}\subpageref{NWtG4Lr-2WycFE-1}}\RA{}
\LA{}double-stacked~{\nwtagstyle{}\subpageref{NWtG4Lr-3SXF7P-1}}\RA{}

: test ( -- )
    /People assert( PEOPLE PHONE fits ) display-everyone
    /Personnel assert( PERSONNEL RECORDS - 4 = ) all
        2 READ assert( S" Greg" NICKNAME text= )
    /Customers FALSE ALT ! assert( SERIALS 2 -linked ) 1 show 2 show
        TRUE ALT ! assert( SERIALS 3 -linked ) 1 show 2 show
    /Accounts assert( ACCOUNTS BALANCE 1 double-stacked ) balances
    /Wines INVENTORY
    /Glossary SUMMARY
;
\nwused{\\{NWtG4Lr-2V5MiK-1}}\nwendcode{}\nwbegindocs{300}\nwdocspar

The 1st problem I discovered was that the \texttt{\textbf{people-fields}}
used 2 more bytes than the \texttt{\textbf{PEOPLE}} database was defined
to use. While this is a common proplem with database definitions,
it is not easily solved in a generic manner. Therefore, a specific
test will have to suffice:

\nwenddocs{}\nwbegincode{301}\sublabel{NWtG4Lr-3C1YNj-1}\nwmargintag{{\nwtagstyle{}\subpageref{NWtG4Lr-3C1YNj-1}}}\moddef{fits~{\nwtagstyle{}\subpageref{NWtG4Lr-3C1YNj-1}}}\endmoddef\nwstartdeflinemarkup\nwusesondefline{\\{NWtG4Lr-Joiug-1}}\nwenddeflinemarkup
: fits ( n a -- flag )   WORKING - + B/R @ > 0= ;
\nwused{\\{NWtG4Lr-Joiug-1}}\nwendcode{}\nwbegindocs{302}\nwdocspar

The problem with the \texttt{\textbf{PERSONNEL}} data was that it
had been created with \texttt{\textbf{FALSE REVERSE !}} and  \texttt{\textbf{BIG-ENDIAN}},
which isn't how it was described in this document. The latter is easy
to test against a know file size. The former requires a way to test
text fields:

\nwenddocs{}\nwbegincode{303}\sublabel{NWtG4Lr-3PN52u-1}\nwmargintag{{\nwtagstyle{}\subpageref{NWtG4Lr-3PN52u-1}}}\moddef{text=~{\nwtagstyle{}\subpageref{NWtG4Lr-3PN52u-1}}}\endmoddef\nwstartdeflinemarkup\nwusesondefline{\\{NWtG4Lr-Joiug-1}}\nwenddeflinemarkup
: text= ( a1 n1 n2 a2 -- flag )   2DUP B@ (S.) COMPARE 0= ;
\nwused{\\{NWtG4Lr-Joiug-1}}\nwendcode{}\nwbegindocs{304}\nwdocspar

The \texttt{\textbf{CUSTOMERS}} example provided the 1st case of a
64-bit problem. This ended up being a problem that \texttt{\textbf{L@}}
returned unsigned 32-bit values and we need an \texttt{\textbf{LU@}}
to do that and sign extend the value in \texttt{\textbf{L@}}, just
as we had to do in \texttt{\textbf{N@}} when porting this code from
16-bit to 32-bit. A simple test for a negative link is needed here:

\nwenddocs{}\nwbegincode{305}\sublabel{NWtG4Lr-2WycFE-1}\nwmargintag{{\nwtagstyle{}\subpageref{NWtG4Lr-2WycFE-1}}}\moddef{-linked~{\nwtagstyle{}\subpageref{NWtG4Lr-2WycFE-1}}}\endmoddef\nwstartdeflinemarkup\nwusesondefline{\\{NWtG4Lr-Joiug-1}}\nwenddeflinemarkup
: -linked ( n -- flag )   READ LINK L@ 0< ;
\nwused{\\{NWtG4Lr-Joiug-1}}\nwendcode{}\nwbegindocs{306}\nwdocspar

The \texttt{\textbf{ACCOUNTS}} example has a similar problem with
\texttt{\textbf{D@}}, but it is not so easily solved. When this code
was converted from 16 to 32-bits, 16-bit compatibility was abandonded.
The \texttt{\textbf{LONG}} fields were added and the \texttt{\textbf{DOUBLE}}
fields were extended to cover 64-bits. However, this time the desire
is to preserve the compatibility and not add another field type. Therefore,
another solution must be developed, which can actually bring back
the support for the incompatible, 16-bit systems. To test this, we
need to make sure that stack effects don't affect the test, like this:

\nwenddocs{}\nwbegincode{307}\sublabel{NWtG4Lr-3SXF7P-1}\nwmargintag{{\nwtagstyle{}\subpageref{NWtG4Lr-3SXF7P-1}}}\moddef{double-stacked~{\nwtagstyle{}\subpageref{NWtG4Lr-3SXF7P-1}}}\endmoddef\nwstartdeflinemarkup\nwusesondefline{\\{NWtG4Lr-Joiug-1}}\nwenddeflinemarkup
: double-stacked ( a n -- flag )   READ >R DEPTH R> SWAP >R D@ 2DROP DEPTH R> - 0= ;
\nwused{\\{NWtG4Lr-Joiug-1}}\nwendcode{}

\nwixlogsorted{c}{{!label}{NWtG4Lr-3Pw6lL-1}{\nwixu{NWtG4Lr-1aUp39-1}\nwixd{NWtG4Lr-3Pw6lL-1}}}%
\nwixlogsorted{c}{{+ordered}{NWtG4Lr-xEqwf-1}{\nwixu{NWtG4Lr-2dg69x-1}\nwixd{NWtG4Lr-xEqwf-1}}}%
\nwixlogsorted{c}{{+user}{NWtG4Lr-Zsakx-1}{\nwixu{NWtG4Lr-2V5MiK-1}\nwixd{NWtG4Lr-Zsakx-1}}}%
\nwixlogsorted{c}{{,string}{NWtG4Lr-4Hk4uw-1}{\nwixu{NWtG4Lr-2V5MiK-1}\nwixd{NWtG4Lr-4Hk4uw-1}}}%
\nwixlogsorted{c}{{-binary}{NWtG4Lr-1Wzvls-1}{\nwixu{NWtG4Lr-2dg69x-1}\nwixd{NWtG4Lr-1Wzvls-1}}}%
\nwixlogsorted{c}{{-linked}{NWtG4Lr-2WycFE-1}{\nwixu{NWtG4Lr-Joiug-1}\nwixd{NWtG4Lr-2WycFE-1}}}%
\nwixlogsorted{c}{{-locate}{NWtG4Lr-3ikf9S-1}{\nwixu{NWtG4Lr-2dg69x-1}\nwixd{NWtG4Lr-3ikf9S-1}}}%
\nwixlogsorted{c}{{-next}{NWtG4Lr-49QaAb-1}{\nwixu{NWtG4Lr-2dg69x-1}\nwixd{NWtG4Lr-49QaAb-1}}}%
\nwixlogsorted{c}{{-ordered}{NWtG4Lr-2HqFpk-1}{\nwixu{NWtG4Lr-2dg69x-1}\nwixd{NWtG4Lr-2HqFpk-1}}}%
\nwixlogsorted{c}{{-spin}{NWtG4Lr-2LU99s-1}{\nwixu{NWtG4Lr-2Cbxl2-1}\nwixd{NWtG4Lr-2LU99s-1}}}%
\nwixlogsorted{c}{{-text}{NWtG4Lr-4HDS1q-1}{\nwixu{NWtG4Lr-2V5MiK-1}\nwixd{NWtG4Lr-4HDS1q-1}}}%
\nwixlogsorted{c}{{.date}{NWtG4Lr-3H8kGz-1}{\nwixu{NWtG4Lr-41VtOE-1}\nwixd{NWtG4Lr-3H8kGz-1}}}%
\nwixlogsorted{c}{{.person}{NWtG4Lr-GdxRl-1}{\nwixu{NWtG4Lr-1aUp39-1}\nwixd{NWtG4Lr-GdxRl-1}}}%
\nwixlogsorted{c}{{.phone}{NWtG4Lr-27PjQm-1}{\nwixu{NWtG4Lr-1aUp39-1}\nwixd{NWtG4Lr-27PjQm-1}}}%
\nwixlogsorted{c}{{.zip}{NWtG4Lr-QWrSM-1}{\nwixu{NWtG4Lr-1aUp39-1}\nwixd{NWtG4Lr-QWrSM-1}}}%
\nwixlogsorted{c}{{<link}{NWtG4Lr-r5XgZ-1}{\nwixu{NWtG4Lr-aGu5T-1}\nwixd{NWtG4Lr-r5XgZ-1}}}%
\nwixlogsorted{c}{{>link}{NWtG4Lr-1CisWZ-1}{\nwixu{NWtG4Lr-aGu5T-1}\nwixd{NWtG4Lr-1CisWZ-1}}}%
\nwixlogsorted{c}{{@!,link}{NWtG4Lr-41t9JJ-1}{\nwixd{NWtG4Lr-41t9JJ-1}\nwixu{NWtG4Lr-aGu5T-1}}}%
\nwixlogsorted{c}{{a.d.}{NWtG4Lr-2AFXuz-1}{\nwixu{NWtG4Lr-41VtOE-1}\nwixd{NWtG4Lr-2AFXuz-1}}}%
\nwixlogsorted{c}{{access}{NWtG4Lr-17LFpz-1}{\nwixu{NWtG4Lr-4ffLeu-1}\nwixd{NWtG4Lr-17LFpz-1}}}%
\nwixlogsorted{c}{{accounts}{NWtG4Lr-34dcAv-1}{\nwixu{NWtG4Lr-2V5MiK-1}\nwixd{NWtG4Lr-34dcAv-1}}}%
\nwixlogsorted{c}{{address}{NWtG4Lr-1pHJON-1}{\nwixu{NWtG4Lr-4ffLeu-1}\nwixd{NWtG4Lr-1pHJON-1}}}%
\nwixlogsorted{c}{{all}{NWtG4Lr-j6xjA-1}{\nwixu{NWtG4Lr-1aUp39-1}\nwixd{NWtG4Lr-j6xjA-1}}}%
\nwixlogsorted{c}{{all-people}{NWtG4Lr-36NJG5-1}{\nwixu{NWtG4Lr-3Y3Oqi-1}\nwixd{NWtG4Lr-36NJG5-1}}}%
\nwixlogsorted{c}{{alt-add}{NWtG4Lr-1M2tOz-1}{\nwixu{NWtG4Lr-i0vIO-1}\nwixd{NWtG4Lr-1M2tOz-1}}}%
\nwixlogsorted{c}{{alt-locate}{NWtG4Lr-4dMbP0-1}{\nwixu{NWtG4Lr-i0vIO-1}\nwixd{NWtG4Lr-4dMbP0-1}}}%
\nwixlogsorted{c}{{alt-serials}{NWtG4Lr-2TgUOO-1}{\nwixu{NWtG4Lr-i0vIO-1}\nwixd{NWtG4Lr-2TgUOO-1}}}%
\nwixlogsorted{c}{{app"}{NWtG4Lr-1ZZNZF-1}{\nwixu{NWtG4Lr-2V5MiK-1}\nwixd{NWtG4Lr-1ZZNZF-1}}}%
\nwixlogsorted{c}{{append}{NWtG4Lr-2ZtF2B-1}{\nwixu{NWtG4Lr-2V5MiK-1}\nwixd{NWtG4Lr-2ZtF2B-1}}}%
\nwixlogsorted{c}{{ask}{NWtG4Lr-2k8QJQ-1}{\nwixu{NWtG4Lr-4ffLeu-1}\nwixd{NWtG4Lr-2k8QJQ-1}}}%
\nwixlogsorted{c}{{at}{NWtG4Lr-100f20-1}{\nwixu{NWtG4Lr-2RYM6N-1}\nwixd{NWtG4Lr-100f20-1}}}%
\nwixlogsorted{c}{{available}{NWtG4Lr-2zefAp-1}{\nwixu{NWtG4Lr-4ffLeu-1}\nwixd{NWtG4Lr-2zefAp-1}}}%
\nwixlogsorted{c}{{binary}{NWtG4Lr-1rYrP5-1}{\nwixu{NWtG4Lr-2dg69x-1}\nwixd{NWtG4Lr-1rYrP5-1}}}%
\nwixlogsorted{c}{{bind-fields}{NWtG4Lr-2eFuO6-1}{\nwixu{NWtG4Lr-4ffLeu-1}\nwixd{NWtG4Lr-2eFuO6-1}}}%
\nwixlogsorted{c}{{block}{NWtG4Lr-4IiShX-1}{\nwixu{NWtG4Lr-4ffLeu-1}\nwixd{NWtG4Lr-4IiShX-1}}}%
\nwixlogsorted{c}{{byte}{NWtG4Lr-4CCogU-1}{\nwixu{NWtG4Lr-4ffLeu-1}\nwixd{NWtG4Lr-4CCogU-1}}}%
\nwixlogsorted{c}{{bytes}{NWtG4Lr-4T6feO-1}{\nwixu{NWtG4Lr-4ffLeu-1}\nwixd{NWtG4Lr-4T6feO-1}}}%
\nwixlogsorted{c}{{c!-be}{NWtG4Lr-1FVKR8-1}{\nwixd{NWtG4Lr-1FVKR8-1}\nwixu{NWtG4Lr-1c5Q5W-1}}}%
\nwixlogsorted{c}{{c!-le}{NWtG4Lr-1FK9pC-1}{\nwixd{NWtG4Lr-1FK9pC-1}\nwixu{NWtG4Lr-1c5Q5W-1}}}%
\nwixlogsorted{c}{{c@-be}{NWtG4Lr-1zG2w7-1}{\nwixd{NWtG4Lr-1zG2w7-1}\nwixu{NWtG4Lr-1c5Q5W-1}}}%
\nwixlogsorted{c}{{c@-le}{NWtG4Lr-1zZfnn-1}{\nwixd{NWtG4Lr-1zZfnn-1}\nwixu{NWtG4Lr-1c5Q5W-1}}}%
\nwixlogsorted{c}{{calls}{NWtG4Lr-16qzin-1}{\nwixu{NWtG4Lr-aGu5T-1}\nwixd{NWtG4Lr-16qzin-1}}}%
\nwixlogsorted{c}{{chain}{NWtG4Lr-4AREak-1}{\nwixu{NWtG4Lr-2dg69x-1}\nwixd{NWtG4Lr-4AREak-1}}}%
\nwixlogsorted{c}{{cols}{NWtG4Lr-3BC2yN-1}{\nwixu{NWtG4Lr-1HFlBt-1}\nwixd{NWtG4Lr-3BC2yN-1}}}%
\nwixlogsorted{c}{{company"}{NWtG4Lr-1Iq9I9-1}{\nwixu{NWtG4Lr-2V5MiK-1}\nwixd{NWtG4Lr-1Iq9I9-1}}}%
\nwixlogsorted{c}{{copies}{NWtG4Lr-3NRM5h-1}{\nwixu{NWtG4Lr-4ffLeu-1}\nwixd{NWtG4Lr-3NRM5h-1}}}%
\nwixlogsorted{c}{{customers}{NWtG4Lr-i0vIO-1}{\nwixu{NWtG4Lr-2V5MiK-1}\nwixd{NWtG4Lr-i0vIO-1}}}%
\nwixlogsorted{c}{{date-format}{NWtG4Lr-41VtOE-1}{\nwixu{NWtG4Lr-2V5MiK-1}\nwixd{NWtG4Lr-41VtOE-1}}}%
\nwixlogsorted{c}{{days}{NWtG4Lr-3D4RV7-1}{\nwixu{NWtG4Lr-41VtOE-1}\nwixd{NWtG4Lr-3D4RV7-1}}}%
\nwixlogsorted{c}{{dd-mmm-yyyy}{NWtG4Lr-YNC8v-1}{\nwixu{NWtG4Lr-41VtOE-1}\nwixd{NWtG4Lr-YNC8v-1}}}%
\nwixlogsorted{c}{{delete}{NWtG4Lr-3WJOq3-1}{\nwixu{NWtG4Lr-2RYM6N-1}\nwixd{NWtG4Lr-3WJOq3-1}}}%
\nwixlogsorted{c}{{digits}{NWtG4Lr-8Jj7a-1}{\nwixu{NWtG4Lr-1aUp39-1}\nwixd{NWtG4Lr-8Jj7a-1}}}%
\nwixlogsorted{c}{{display-everyone}{NWtG4Lr-2lRxl2-1}{\nwixu{NWtG4Lr-3Y3Oqi-1}\nwixd{NWtG4Lr-2lRxl2-1}}}%
\nwixlogsorted{c}{{display-person}{NWtG4Lr-2HPrsv-1}{\nwixu{NWtG4Lr-3Y3Oqi-1}\nwixd{NWtG4Lr-2HPrsv-1}}}%
\nwixlogsorted{c}{{dmy}{NWtG4Lr-46kc96-1}{\nwixu{NWtG4Lr-41VtOE-1}\nwixd{NWtG4Lr-46kc96-1}}}%
\nwixlogsorted{c}{{double}{NWtG4Lr-4Sesx5-1}{\nwixu{NWtG4Lr-4ffLeu-1}\nwixd{NWtG4Lr-4Sesx5-1}}}%
\nwixlogsorted{c}{{double-stacked}{NWtG4Lr-3SXF7P-1}{\nwixu{NWtG4Lr-Joiug-1}\nwixd{NWtG4Lr-3SXF7P-1}}}%
\nwixlogsorted{c}{{dy}{NWtG4Lr-478QOO-1}{\nwixu{NWtG4Lr-41VtOE-1}\nwixd{NWtG4Lr-478QOO-1}}}%
\nwixlogsorted{c}{{endians}{NWtG4Lr-1c5Q5W-1}{\nwixu{NWtG4Lr-2V5MiK-1}\nwixd{NWtG4Lr-1c5Q5W-1}}}%
\nwixlogsorted{c}{{enter}{NWtG4Lr-2GxEXJ-1}{\nwixu{NWtG4Lr-1aUp39-1}\nwixd{NWtG4Lr-2GxEXJ-1}}}%
\nwixlogsorted{c}{{enter-person}{NWtG4Lr-30BTWM-1}{\nwixu{NWtG4Lr-3Y3Oqi-1}\nwixd{NWtG4Lr-30BTWM-1}}}%
\nwixlogsorted{c}{{entire}{NWtG4Lr-uRyxG-1}{\nwixu{NWtG4Lr-4ffLeu-1}\nwixd{NWtG4Lr-uRyxG-1}}}%
\nwixlogsorted{c}{{f}{NWtG4Lr-1EuC3M-1}{\nwixu{NWtG4Lr-2RYM6N-1}\nwixd{NWtG4Lr-1EuC3M-1}}}%
\nwixlogsorted{c}{{fetch-date}{NWtG4Lr-1MB56x-1}{\nwixu{NWtG4Lr-41VtOE-1}\nwixd{NWtG4Lr-1MB56x-1}}}%
\nwixlogsorted{c}{{fetch-mth}{NWtG4Lr-2xfvjX-1}{\nwixu{NWtG4Lr-41VtOE-1}\nwixd{NWtG4Lr-2xfvjX-1}}}%
\nwixlogsorted{c}{{fields}{NWtG4Lr-7mIQU-1}{\nwixu{NWtG4Lr-4ffLeu-1}\nwixd{NWtG4Lr-7mIQU-1}}}%
\nwixlogsorted{c}{{file}{NWtG4Lr-4EB9lH-1}{\nwixu{NWtG4Lr-4ffLeu-1}\nwixd{NWtG4Lr-4EB9lH-1}}}%
\nwixlogsorted{c}{{filler}{NWtG4Lr-4UfOWH-1}{\nwixu{NWtG4Lr-4ffLeu-1}\nwixd{NWtG4Lr-4UfOWH-1}}}%
\nwixlogsorted{c}{{find}{NWtG4Lr-MECIZ-1}{\nwixu{NWtG4Lr-2RYM6N-1}\nwixd{NWtG4Lr-MECIZ-1}}}%
\nwixlogsorted{c}{{first}{NWtG4Lr-2i9BK7-1}{\nwixu{NWtG4Lr-2dg69x-1}\nwixd{NWtG4Lr-2i9BK7-1}}}%
\nwixlogsorted{c}{{fits}{NWtG4Lr-3C1YNj-1}{\nwixu{NWtG4Lr-Joiug-1}\nwixd{NWtG4Lr-3C1YNj-1}}}%
\nwixlogsorted{c}{{fix}{NWtG4Lr-OHhPA-1}{\nwixu{NWtG4Lr-1aUp39-1}\nwixd{NWtG4Lr-OHhPA-1}}}%
\nwixlogsorted{c}{{float}{NWtG4Lr-2mGd9n-1}{\nwixu{NWtG4Lr-4ffLeu-1}\nwixd{NWtG4Lr-2mGd9n-1}}}%
\nwixlogsorted{c}{{foot}{NWtG4Lr-W27Dj-1}{\nwixu{NWtG4Lr-1HFlBt-1}\nwixd{NWtG4Lr-W27Dj-1}}}%
\nwixlogsorted{c}{{get}{NWtG4Lr-101V9w-1}{\nwixu{NWtG4Lr-2V5MiK-1}\nwixd{NWtG4Lr-101V9w-1}}}%
\nwixlogsorted{c}{{glossary}{NWtG4Lr-2RYM6N-1}{\nwixu{NWtG4Lr-2V5MiK-1}\nwixd{NWtG4Lr-2RYM6N-1}}}%
\nwixlogsorted{c}{{glossary-file}{NWtG4Lr-1yjrvH-1}{\nwixu{NWtG4Lr-2RYM6N-1}\nwixd{NWtG4Lr-1yjrvH-1}}}%
\nwixlogsorted{c}{{grab}{NWtG4Lr-1YVM30-1}{\nwixu{NWtG4Lr-101V9w-1}\nwixd{NWtG4Lr-1YVM30-1}}}%
\nwixlogsorted{c}{{grand}{NWtG4Lr-1VMgcH-1}{\nwixu{NWtG4Lr-1HFlBt-1}\nwixd{NWtG4Lr-1VMgcH-1}}}%
\nwixlogsorted{c}{{head}{NWtG4Lr-QPhyQ-1}{\nwixu{NWtG4Lr-2dg69x-1}\nwixd{NWtG4Lr-QPhyQ-1}}}%
\nwixlogsorted{c}{{heading}{NWtG4Lr-14go8R-1}{\nwixu{NWtG4Lr-1HFlBt-1}\nwixd{NWtG4Lr-14go8R-1}}}%
\nwixlogsorted{c}{{index}{NWtG4Lr-2dg69x-1}{\nwixu{NWtG4Lr-2V5MiK-1}\nwixd{NWtG4Lr-2dg69x-1}}}%
\nwixlogsorted{c}{{indexed}{NWtG4Lr-36msie-1}{\nwixu{NWtG4Lr-4ffLeu-1}\nwixd{NWtG4Lr-36msie-1}}}%
\nwixlogsorted{c}{{initialize}{NWtG4Lr-p3Fax-1}{\nwixu{NWtG4Lr-4ffLeu-1}\nwixd{NWtG4Lr-p3Fax-1}}}%
\nwixlogsorted{c}{{keep-closed}{NWtG4Lr-4MW32q-1}{\nwixu{NWtG4Lr-2V5MiK-1}\nwixd{NWtG4Lr-4MW32q-1}}}%
\nwixlogsorted{c}{{layout}{NWtG4Lr-4JwrrS-1}{\nwixu{NWtG4Lr-1HFlBt-1}\nwixd{NWtG4Lr-4JwrrS-1}}}%
\nwixlogsorted{c}{{line}{NWtG4Lr-47jTCN-1}{\nwixu{NWtG4Lr-3BC2yN-1}\nwixd{NWtG4Lr-47jTCN-1}}}%
\nwixlogsorted{c}{{link}{NWtG4Lr-3NK3Qp-1}{\nwixu{NWtG4Lr-4ffLeu-1}\nwixd{NWtG4Lr-3NK3Qp-1}}}%
\nwixlogsorted{c}{{links}{NWtG4Lr-aGu5T-1}{\nwixu{NWtG4Lr-2V5MiK-1}\nwixd{NWtG4Lr-aGu5T-1}}}%
\nwixlogsorted{c}{{long}{NWtG4Lr-2qounD-1}{\nwixu{NWtG4Lr-4ffLeu-1}\nwixd{NWtG4Lr-2qounD-1}}}%
\nwixlogsorted{c}{{mdy}{NWtG4Lr-478cHs-1}{\nwixu{NWtG4Lr-41VtOE-1}\nwixd{NWtG4Lr-478cHs-1}}}%
\nwixlogsorted{c}{{memory}{NWtG4Lr-4K7kUG-1}{\nwixu{NWtG4Lr-2V5MiK-1}\nwixd{NWtG4Lr-4K7kUG-1}}}%
\nwixlogsorted{c}{{mmddyyyy}{NWtG4Lr-3b82bI-1}{\nwixu{NWtG4Lr-41VtOE-1}\nwixd{NWtG4Lr-3b82bI-1}}}%
\nwixlogsorted{c}{{moved}{NWtG4Lr-1nqpD2-1}{\nwixu{NWtG4Lr-2RYM6N-1}\nwixd{NWtG4Lr-1nqpD2-1}}}%
\nwixlogsorted{c}{{mth}{NWtG4Lr-UYpfI-1}{\nwixu{NWtG4Lr-41VtOE-1}\nwixd{NWtG4Lr-UYpfI-1}}}%
\nwixlogsorted{c}{{new}{NWtG4Lr-3OC4Hk-1}{\nwixu{NWtG4Lr-2RYM6N-1}\nwixd{NWtG4Lr-3OC4Hk-1}}}%
\nwixlogsorted{c}{{number-user}{NWtG4Lr-2KZt7q-1}{\nwixu{NWtG4Lr-2V5MiK-1}\nwixd{NWtG4Lr-2KZt7q-1}}}%
\nwixlogsorted{c}{{numeric}{NWtG4Lr-44RunE-1}{\nwixu{NWtG4Lr-4ffLeu-1}\nwixd{NWtG4Lr-44RunE-1}}}%
\nwixlogsorted{c}{{ordered}{NWtG4Lr-4Y6sNo-1}{\nwixu{NWtG4Lr-4ffLeu-1}\nwixd{NWtG4Lr-4Y6sNo-1}}}%
\nwixlogsorted{c}{{origin}{NWtG4Lr-2aZT6J-1}{\nwixu{NWtG4Lr-4ffLeu-1}\nwixd{NWtG4Lr-2aZT6J-1}}}%
\nwixlogsorted{c}{{output}{NWtG4Lr-2ruhRa-1}{\nwixu{NWtG4Lr-1HFlBt-1}\nwixd{NWtG4Lr-2ruhRa-1}}}%
\nwixlogsorted{c}{{p}{NWtG4Lr-CgRsG-1}{\nwixu{NWtG4Lr-2RYM6N-1}\nwixd{NWtG4Lr-CgRsG-1}}}%
\nwixlogsorted{c}{{page}{NWtG4Lr-4GHBLz-1}{\nwixu{NWtG4Lr-1HFlBt-1}\nwixd{NWtG4Lr-4GHBLz-1}}}%
\nwixlogsorted{c}{{people}{NWtG4Lr-3Y3Oqi-1}{\nwixu{NWtG4Lr-2V5MiK-1}\nwixd{NWtG4Lr-3Y3Oqi-1}}}%
\nwixlogsorted{c}{{people-fields}{NWtG4Lr-1tBGdQ-1}{\nwixu{NWtG4Lr-3Y3Oqi-1}\nwixd{NWtG4Lr-1tBGdQ-1}}}%
\nwixlogsorted{c}{{people-file}{NWtG4Lr-2XK40H-1}{\nwixu{NWtG4Lr-3Y3Oqi-1}\nwixd{NWtG4Lr-2XK40H-1}}}%
\nwixlogsorted{c}{{people-initialize}{NWtG4Lr-4XCuIG-1}{\nwixu{NWtG4Lr-3Y3Oqi-1}\nwixd{NWtG4Lr-4XCuIG-1}}}%
\nwixlogsorted{c}{{person}{NWtG4Lr-1dYgi5-1}{\nwixu{NWtG4Lr-1aUp39-1}\nwixd{NWtG4Lr-1dYgi5-1}}}%
\nwixlogsorted{c}{{personnel}{NWtG4Lr-1aUp39-1}{\nwixu{NWtG4Lr-2V5MiK-1}\nwixd{NWtG4Lr-1aUp39-1}}}%
\nwixlogsorted{c}{{personnel-layout}{NWtG4Lr-3Rjkip-1}{\nwixu{NWtG4Lr-1aUp39-1}\nwixd{NWtG4Lr-3Rjkip-1}}}%
\nwixlogsorted{c}{{pFDatabase.fth}{NWtG4Lr-2V5MiK-1}{\nwixd{NWtG4Lr-2V5MiK-1}}}%
\nwixlogsorted{c}{{place}{NWtG4Lr-24CRHn-1}{\nwixu{NWtG4Lr-2V5MiK-1}\nwixd{NWtG4Lr-24CRHn-1}}}%
\nwixlogsorted{c}{{put}{NWtG4Lr-105aWa-1}{\nwixu{NWtG4Lr-4ffLeu-1}\nwixd{NWtG4Lr-105aWa-1}}}%
\nwixlogsorted{c}{{read}{NWtG4Lr-WHmjl-1}{\nwixu{NWtG4Lr-4ffLeu-1}\nwixd{NWtG4Lr-WHmjl-1}}}%
\nwixlogsorted{c}{{records}{NWtG4Lr-1FNAaT-1}{\nwixu{NWtG4Lr-4ffLeu-1}\nwixd{NWtG4Lr-1FNAaT-1}}}%
\nwixlogsorted{c}{{register}{NWtG4Lr-8Tw6T-1}{\nwixu{NWtG4Lr-1HFlBt-1}\nwixd{NWtG4Lr-8Tw6T-1}}}%
\nwixlogsorted{c}{{release}{NWtG4Lr-qFr4D-1}{\nwixu{NWtG4Lr-101V9w-1}\nwixd{NWtG4Lr-qFr4D-1}}}%
\nwixlogsorted{c}{{remove}{NWtG4Lr-3W96CK-1}{\nwixu{NWtG4Lr-1aUp39-1}\nwixd{NWtG4Lr-3W96CK-1}}}%
\nwixlogsorted{c}{{reports}{NWtG4Lr-1HFlBt-1}{\nwixu{NWtG4Lr-2V5MiK-1}\nwixd{NWtG4Lr-1HFlBt-1}}}%
\nwixlogsorted{c}{{reverse}{NWtG4Lr-3QgOwL-1}{\nwixu{NWtG4Lr-2V5MiK-1}\nwixd{NWtG4Lr-3QgOwL-1}}}%
\nwixlogsorted{c}{{rlc}{NWtG4Lr-2YnRrI-1}{\nwixu{NWtG4Lr-1HFlBt-1}\nwixd{NWtG4Lr-2YnRrI-1}}}%
\nwixlogsorted{c}{{save}{NWtG4Lr-4HTvUe-1}{\nwixu{NWtG4Lr-4ffLeu-1}\nwixd{NWtG4Lr-4HTvUe-1}}}%
\nwixlogsorted{c}{{slot}{NWtG4Lr-M1coC-1}{\nwixu{NWtG4Lr-4ffLeu-1}\nwixd{NWtG4Lr-M1coC-1}}}%
\nwixlogsorted{c}{{snatch}{NWtG4Lr-4GTjjD-1}{\nwixu{NWtG4Lr-2dg69x-1}\nwixd{NWtG4Lr-4GTjjD-1}}}%
\nwixlogsorted{c}{{spin}{NWtG4Lr-2Cbxl2-1}{\nwixu{NWtG4Lr-2V5MiK-1}\nwixd{NWtG4Lr-2Cbxl2-1}}}%
\nwixlogsorted{c}{{stacks}{NWtG4Lr-4Agpfb-1}{\nwixu{NWtG4Lr-2RYM6N-1}\nwixd{NWtG4Lr-4Agpfb-1}}}%
\nwixlogsorted{c}{{string-c,}{NWtG4Lr-b9pLA-1}{\nwixu{NWtG4Lr-2V5MiK-1}\nwixd{NWtG4Lr-b9pLA-1}}}%
\nwixlogsorted{c}{{strings}{NWtG4Lr-jUK2q-1}{\nwixu{NWtG4Lr-4ffLeu-1}\nwixd{NWtG4Lr-jUK2q-1}}}%
\nwixlogsorted{c}{{structure}{NWtG4Lr-2dJGUc-1}{\nwixu{NWtG4Lr-2V5MiK-1}\nwixd{NWtG4Lr-2dJGUc-1}}}%
\nwixlogsorted{c}{{sub}{NWtG4Lr-2JLP8G-1}{\nwixu{NWtG4Lr-1HFlBt-1}\nwixd{NWtG4Lr-2JLP8G-1}}}%
\nwixlogsorted{c}{{sum}{NWtG4Lr-3qrQOy-1}{\nwixu{NWtG4Lr-1HFlBt-1}\nwixd{NWtG4Lr-3qrQOy-1}}}%
\nwixlogsorted{c}{{summary}{NWtG4Lr-3d0IxR-1}{\nwixu{NWtG4Lr-2RYM6N-1}\nwixd{NWtG4Lr-3d0IxR-1}}}%
\nwixlogsorted{c}{{support}{NWtG4Lr-4ffLeu-1}{\nwixd{NWtG4Lr-4ffLeu-1}\nwixu{NWtG4Lr-2V5MiK-1}}}%
\nwixlogsorted{c}{{t}{NWtG4Lr-zn3NQ-1}{\nwixu{NWtG4Lr-2RYM6N-1}\nwixd{NWtG4Lr-zn3NQ-1}}}%
\nwixlogsorted{c}{{test}{NWtG4Lr-Joiug-1}{\nwixu{NWtG4Lr-2V5MiK-1}\nwixd{NWtG4Lr-Joiug-1}}}%
\nwixlogsorted{c}{{text}{NWtG4Lr-KJlJC-1}{\nwixu{NWtG4Lr-2V5MiK-1}\nwixd{NWtG4Lr-KJlJC-1}}}%
\nwixlogsorted{c}{{text=}{NWtG4Lr-3PN52u-1}{\nwixu{NWtG4Lr-Joiug-1}\nwixd{NWtG4Lr-3PN52u-1}}}%
\nwixlogsorted{c}{{time}{NWtG4Lr-4JKPYw-1}{\nwixu{NWtG4Lr-2V5MiK-1}\nwixd{NWtG4Lr-4JKPYw-1}}}%
\nwixlogsorted{c}{{title}{NWtG4Lr-3B2QOB-1}{\nwixu{NWtG4Lr-1HFlBt-1}\nwixd{NWtG4Lr-3B2QOB-1}}}%
\nwixlogsorted{c}{{title"}{NWtG4Lr-4SxyK6-1}{\nwixu{NWtG4Lr-2V5MiK-1}\nwixd{NWtG4Lr-4SxyK6-1}}}%
\nwixlogsorted{c}{{title-heading}{NWtG4Lr-3WJ4qF-1}{\nwixu{NWtG4Lr-1HFlBt-1}\nwixd{NWtG4Lr-3WJ4qF-1}}}%
\nwixlogsorted{c}{{total}{NWtG4Lr-1G4TAa-1}{\nwixu{NWtG4Lr-1HFlBt-1}\nwixd{NWtG4Lr-1G4TAa-1}}}%
\nwixlogsorted{c}{{totals}{NWtG4Lr-3QzPba-1}{\nwixu{NWtG4Lr-1HFlBt-1}\nwixd{NWtG4Lr-3QzPba-1}}}%
\nwixlogsorted{c}{{u}{NWtG4Lr-3Zsojo-1}{\nwixu{NWtG4Lr-2RYM6N-1}\nwixd{NWtG4Lr-3Zsojo-1}}}%
\nwixlogsorted{c}{{unchain}{NWtG4Lr-3m1wNz-1}{\nwixu{NWtG4Lr-2dg69x-1}\nwixd{NWtG4Lr-3m1wNz-1}}}%
\nwixlogsorted{c}{{unlink}{NWtG4Lr-1nyRGB-1}{\nwixu{NWtG4Lr-aGu5T-1}\nwixd{NWtG4Lr-1nyRGB-1}}}%
\nwixlogsorted{c}{{user}{NWtG4Lr-1alVLr-1}{\nwixu{NWtG4Lr-4ffLeu-1}\nwixd{NWtG4Lr-1alVLr-1}}}%
\nwixlogsorted{c}{{view}{NWtG4Lr-2uul1M-1}{\nwixu{NWtG4Lr-1aUp39-1}\nwixd{NWtG4Lr-2uul1M-1}}}%
\nwixlogsorted{c}{{vocab}{NWtG4Lr-26DLU4-1}{\nwixu{NWtG4Lr-2RYM6N-1}\nwixd{NWtG4Lr-26DLU4-1}}}%
\nwixlogsorted{c}{{vocabulary}{NWtG4Lr-424Fqv-1}{\nwixu{NWtG4Lr-2RYM6N-1}\nwixd{NWtG4Lr-424Fqv-1}}}%
\nwixlogsorted{c}{{wines}{NWtG4Lr-1KI31P-1}{\nwixu{NWtG4Lr-2V5MiK-1}\nwixd{NWtG4Lr-1KI31P-1}}}%
\nwixlogsorted{c}{{working}{NWtG4Lr-nM8ar-1}{\nwixu{NWtG4Lr-4ffLeu-1}\nwixd{NWtG4Lr-nM8ar-1}}}%
\nwixlogsorted{c}{{x}{NWtG4Lr-OSnlw-1}{\nwixu{NWtG4Lr-2RYM6N-1}\nwixd{NWtG4Lr-OSnlw-1}}}%
\nwixlogsorted{c}{{y-dd-dm}{NWtG4Lr-2JA1Oz-1}{\nwixu{NWtG4Lr-41VtOE-1}\nwixd{NWtG4Lr-2JA1Oz-1}}}%
\nwbegindocs{308}\nwdocspar

The WINES example works fine at this point, as does the GLOSSARY,
so we are done for now. However, it's is worth noting, in summary,
that I did not test everything, which is appropriate for legacy applications,
such as this. Tests can be added as needed, when problems are found.
\end{document}
\nwenddocs{}
